This chapter details and validates the procedure for calculating the theoretical average transmission to accurately fit parameters to self-shielded transmission measurements in the URR\cite{Bahran2015}. As shown in \autoref{sec:resonance-self-shielding}, the theoretical average transmission can be calculated as the product of the transmission of the average cross section, $e^{-n\langle \sigma \rangle}$, and a transmission correction factor $C_T$. SESH uses Monte Carlo integration in order to determine these energy-averaged quantities\cite{sesh}.

SESH is used here as the transport engine that computes the self-shielding transmission correction factor $C_T$, while SAMMY performs the parameter fitting. The goal of this chapter is to validate that $C_T$ is calculated correctly and that applying it within SAMMY leads to consistent transmission predictions and stable fits.
\section{Transmission Correction for a Mono-Isotopic Sample}
\label{sec:tc-monotopic}

The calculation of the transmission correction factor is centered around a Monte Carlo simulation that generates a statistical population of cross-section realizations. This process, described in detail in Chapters \ref{chap:theory} and \ref{chap:implementation}, simulates the inherent fluctuations of the cross-section in the URR. For each Monte Carlo history, a ``resonance ladder'' is constructed for each channel by sampling resonance widths from the Porter-Thomas\cite{Porter1956} distribution and level spacings from a Wigner distribution\cite{Wigner1951}. These stochastically generated parameters are then used within the SLBW formalism\cite{Lane1958} to produce a single realization of the total microscopic cross-section, $\sigma_i$.

For each sampled cross-section, a corresponding transmission value is calculated, $T_{i} = e^{-n\sigma_{i}}$, where $n$ is the sample thickness. By repeating this for a large number of histories, a set of sampled cross-sections $\left\{ \sigma_{1}, ..., \sigma_{N} \right\}$ and a corresponding set of transmissions $\left\{ T_{1}, ..., T_{N}\right\}$ are obtained. From these sets, their respective averages are computed:
\begin{equation}
    \langle \sigma \rangle = \frac{1}{N} \sum_{i=1}^{N} \sigma_{i}
\end{equation}
and
\begin{equation}
    \langle T \rangle = \frac{1}{N} \sum_{i=1}^{N} T_{i}
\end{equation}

From these averages, the correction factor defined in \autoref{sec:resonance-self-shielding} can be approximated as:
\begin{equation}
    C_T \approx \frac{\langle T \rangle}{e^{-n\langle \sigma \rangle}}
\label{eq:CT_mc}
\end{equation}

The accuracy of this method will be validated using $^{181}$Ta, for which a recent evaluation provides new resonance parameters\cite{Brown2024}. The validation is twofold. First, the theoretical model is compared against MCNP simulations to provide a detailed characterization of its performance. As the MCNP model utilizes the exact best-fit parameters from the recent $^{181}$Ta evaluation, it serves as an useful reference case to investigate the effects of physical parameters such as sample thickness and temperature. Second, the calculated average transmission is compared against the experimental data that was used in the evaluation. This comparison serves to validate the theoretical model's overall performance for a single sample thickness, serving as a final integration test for SESH’s theoretical self-shielded transmission calculation.

\section{Verification of the Correction Factor Model}
Before integrating the transmission correction into the fitting procedure, it is essential to verify that SESH accurately calculates the correction factor itself. The verification was performed by comparing the results from SESH to those from a high-fidelity model of the experiment constructed using MCNP \cite{mcnp}. This MCNP model serves as a computational benchmark\cite{Humbert2014}, providing a a high-fidelity reference against which the statistical methods in SESH can be compared. Three key physical parameters that could influence the correction factor were investigated: the number of external resonance pairs used in the calculation, the physical thickness of the sample, and the temperature of the sample.

\subsection{MCNP Reference Model}
\label{ssec:mcnp-benchmark-model}

MCNP is used here as a transport reference: both MCNP and SAMMY/SESH use the same evaluated nuclear data, so the comparison primarily tests the correction method and its implementation rather than differences in input cross sections.
The MCNP model was configured to simulate neutron transmission through a $^{181}$Ta sample. The cross-section data for the simulation was prepared from the ENDF/B-VIII.1 evaluation \cite{endf-manual} and processed into a pointwise format using NJOY \cite{njoy}. By using the exact, finely-resolved cross sections from a formal evaluation, the MCNP simulation can directly calculate the true average transmission, $\langle T \rangle$, and the transmission of the average cross section, $e^{-n\langle\sigma\rangle}$, over a given energy group. The benchmark correction factor is then calculated as:
\begin{equation}
    C_{T, \text{MCNP}} = \frac{\langle T \rangle_{\text{P-Table=ON}}}{\langle T \rangle_{\text{P-Table=OFF}}}
\end{equation}
This provides an ideal reference value, free from the statistical approximations inherent to the SESH methodology, against which SESH's calculations can be validated.

\subsection{Monte Carlo Convergence}
\label{ssec:mc-convergence}

The preceding subsections validated the accuracy of the SESH correction factor against an MCNP benchmark, but the Monte Carlo procedure introduces a separate concern: statistical precision. Because $C_T$ is estimated from a finite sample of resonance ladder realizations, each evaluation carries a random fluctuation whose magnitude depends on both the number of histories $N$ and the physical conditions of the calculation. In a standalone comparison to data this fluctuation simply widens the scatter in the theoretical prediction. In the context of iterative fitting, however, an imprecise $C_T$ corrupts the derivative information used by the Bayesian update, causing the fit to oscillate around the minimum rather than converge smoothly.

To characterize the convergence behavior, the correction factor was computed for a 4~mm $^{181}$Ta sample at 300~K using four external resonance pairs at three representative energies spanning the URR: 5~keV, where self-shielding is strongest ($C_T \approx 1.043$); 20~keV, where the correction is moderate ($C_T \approx 1.006$); and 80~keV, where it is negligible ($C_T \approx 1.0005$). At each energy, the number of Monte Carlo histories was varied from $N = 25$ to $N = 100{,}000$, with 30 independent trials performed at each value of $N$ to estimate the standard deviation of the $C_T$ estimator.

\autoref{fig:ct-convergence-means} shows convergence of the computed correction factor as a function of $N$ at each energy. The mean value is stable across all $N$, confirming that the Monte Carlo estimator is unbiased. The error bars contract monotonically with increasing $N$, and the effect is most pronounced at 5~keV where the cross-section fluctuations are largest. At 80~keV, where the level density is high and individual resonances overlap strongly, even $N = 25$ histories produce a correction factor that is effectively converged.

The convergence rate is quantified more precisely in \autoref{fig:ct-convergence-precision}, which plots the relative standard deviation of $C_T$ against $N$. All three energies follow the expected $1/\sqrt{N}$ scaling of a Monte Carlo estimator, confirming that the variance arises from the stochastic sampling of resonance ladders and not from any systematic instability in the algorithm. The vertical separation between the curves reflects the magnitude of the self-shielding correction: larger corrections produce wider cross-section distributions, which require more histories to average precisely.

For the purpose of evaluating resonance parameters, the precision requirement on $C_T$ is more stringent than might be suggested by the correction factor alone. During fitting, SAMMY updates the average resonance parameters iteratively, with each step relying on the partial derivatives of the theoretical transmission with respect to the fitted quantities. When $C_T$ carries excessive statistical noise, these derivatives are contaminated, and the fit oscillates near the minimum rather than converging to it. In practice, this manifests as a final $\chi^2$ that fluctuates between iterations without settling, particularly for the strength functions whose effect on $C_T$ is small but physically meaningful. Empirically, stable convergence of the fitting procedure requires the relative precision on $C_T$ to be well below the sensitivity of the transmission to a single parameter step, which for typical URR evaluations corresponds to a relative standard deviation on the order of 0.1\% or better.

From \autoref{fig:ct-convergence-precision}, this threshold is reached at approximately $N = 50{,}000$ for the most heavily self-shielded case at 5~keV, where the relative standard deviation falls to 0.07\%. At 20~keV the same precision is achieved near $N = 10{,}000$, and at 80~keV it is reached with only a few hundred histories. In general, the required number of histories scales with the magnitude of the self-shielding correction: thicker samples of isotopes with wider resonance-width distributions and lower level densities will require more histories to achieve the same statistical precision. For the evaluations presented in \autoref{chap:evaluation}, $N = 50{,}000$ histories were used at all energies to ensure that the Monte Carlo noise in $C_T$ does not limit the fitting precision.

\begin{figure}[H]
    \centering
    \includegraphics[width=\textwidth]{Transmission Correction/Figures/ct_convergence_means.pdf}
    \caption{Convergence of the transmission correction factor for a 4~mm $^{181}$Ta sample at 300~K at three representative URR energies. Each point is the mean of 30 independent calculations; error bars show $\pm 1\sigma$. The dashed lines indicate the converged value at $N = 100{,}000$. The mean is unbiased at all $N$; only the statistical spread decreases with increasing history count.}
    \label{fig:ct-convergence-means}
\end{figure}

\begin{figure}[H]
    \centering
    \includegraphics[width=0.75\textwidth]{Transmission Correction/Figures/ct_convergence_precision.pdf}
    \caption{Relative standard deviation of the correction factor as a function of the number of Monte Carlo histories. The dashed line shows the expected $1/\sqrt{N}$ scaling. At 5~keV, where self-shielding is strongest, approximately $50{,}000$ histories are needed to reduce the relative uncertainty below 0.1\%. At higher energies where the correction is smaller, fewer histories suffice.}
    \label{fig:ct-convergence-precision}
\end{figure}

\subsection{Influence of External Resonance Pairs}
\label{ssec:resonance-pairs}

The Monte Carlo procedure in SESH constructs a set of resonance ladders within each energy bin by sampling level spacings and widths from their respective statistical distributions. In practice, the cross section at any point within the bin is not determined solely by the resonances inside it --- the tails of nearby resonances outside the bin also contribute. The question is how many of these external resonances need to be included to produce a statistically representative sample of the cross-section fluctuations.

When resonance widths are much larger than the mean level spacing ($\Gamma \gg D$), the cross section at a given energy is the incoherent sum of many overlapping resonance contributions, and the fine details of any individual resonance placement become unimportant\cite{ERICSON1963390}. In this regime, even a modest number of external resonances captures the essential behavior. However, $\Gamma/D$ is not uniformly large across the URR. At lower energies where level densities are smaller and resonances are more isolated, the cross section within a bin can be sensitive to whether a particular external resonance happens to sit just outside the bin edge. Monahan and Elwyn\cite{MONAHAN1967683} showed that fluctuations in the averaged compound cross section depend on both the width and spacing distributions, meaning that truncating the resonance ladder too aggressively can bias the sampled cross-section distribution and, in turn, the correction factor.

The SESH code addresses this by allowing the user to specify a number of external resonance pairs: resonances placed above and below the bin boundaries spaced by sampling from the Wigner Distribution\cite{Wigner1951}. Adding more pairs extends the sampled ladder further from the bin edges, ensuring that the tails of distant resonances are properly included in the cross-section calculation. To quantify the impact of this setting, the correction factor was calculated for a 4~mm $^{181}$Ta sample at 300~K using one through four external resonance pairs.

The results, shown in \autoref{fig:tc-resonance-pairs}, demonstrate that the inclusion of external resonance pairs has a slight but noticeable influence on the calculated correction factor. As more pairs are added, the SESH calculation converges toward the MCNP benchmark value. The effect is most visible at lower energies where the resonances are less overlapping and the bin-edge truncation matters more. By four external pairs the improvement saturates, and additional pairs produce no further change within the statistical precision of the calculation. For the remainder of this work, four external resonance pairs were used in all SESH calculations.

\begin{figure}[H]
    \centering
    \begin{subfigure}[b]{0.48\textwidth}
        \centering
        \includegraphics[width=\textwidth]{Transmission Correction/Figures/resonancepair_CT.pdf}
        \caption{SESH correction factor compared to MCNP (solid black line) for increasing numbers of external resonance pairs.}
        \label{fig:tc-resonance-pairs-data}
    \end{subfigure}
    \hfill
    \begin{subfigure}[b]{0.48\textwidth}
        \centering
        \includegraphics[width=\textwidth]{Transmission Correction/Figures/resonancepair_convergence.pdf}
        \caption{Convergence of $C_T$ toward the MCNP reference (dotted lines) as a function of the number of external pairs at selected energies.}
        \label{fig:tc-resonance-pairs-convergence}
    \end{subfigure}
    \caption{Effect of external resonance pairs on the transmission correction factor for a 4~mm $^{181}$Ta sample at 300~K. The correction converges by approximately four external pairs at all energies, with the effect most pronounced at lower energies where resonance overlap is weakest.}
    \label{fig:tc-resonance-pairs}
\end{figure}


\subsection{Sample Thickness Dependence}
\label{ssec:thickness}

Of the parameters investigated in this chapter, the sample thickness exerts the strongest influence on the transmission correction factor. This sensitivity follows directly from the physics of self-shielding: a thicker sample amplifies the nonlinear relationship between the cross-section fluctuations and the transmitted intensity. As presented in \autoref{eq:taylor-series-expansion},  the correction factor depends on the cross-section variance as $C_T \approx 1 + \frac{1}{2}n^2 \mathrm{Var}(\sigma)$, where $n$ is the sample thickness in atoms per barn. For thin samples the quadratic term is small and $C_T$ remains close to unity. But as $n$ increases the correction grows rapidly and the shape of the cross-section distribution become increasingly important.

To characterize this behavior and determine the range of applicability for SESH, the correction factor was computed for 2~mm, 4~mm, 8~mm, and 12~mm samples of $^{181}$Ta at 300~K and compared against the MCNP benchmark. The results are shown in \autoref{fig:tc-thickness}. For the thinner samples (2--4~mm), the agreement between SESH and MCNP is excellent, with relative errors below 0.1\%. For the thicker samples (8--12~mm), a systematic positive bias emerges in which SESH predicts a larger correction factor than MCNP, with the disagreement growing at lower energies where self-shielding is strongest.

\begin{figure}[H]
    \centering
    \begin{subfigure}[b]{0.48\textwidth}
        \centering
        \includegraphics[width=\textwidth]{Transmission Correction/Figures/thickness_CT.pdf}
        \caption{SESH correction factor (solid lines) compared to MCNP (open circles) for each sample thickness.}
        \label{fig:tc-thickness-data}
    \end{subfigure}
    \hfill
    \begin{subfigure}[b]{0.48\textwidth}
        \centering
        \includegraphics[width=\textwidth]{Transmission Correction/Figures/thickness_error.pdf}
        \caption{Relative deviation of SESH from MCNP as a function of incident energy, showing increasing disagreement for thicker samples at lower energies.}
        \label{fig:tc-thickness-error}
    \end{subfigure}
    \caption{Transmission correction factor for various thicknesses of a $^{181}$Ta sample at 300~K. The two methods agree to within 0.1\% for samples up to 4~mm, while thicker samples exhibit a systematic positive bias in SESH relative to MCNP.}
    \label{fig:tc-thickness}
\end{figure}

The relative error in \autoref{fig:tc-thickness-error} appears to correlate more closely with the magnitude of the correction than with the sample thickness alone. This is confirmed in \autoref{fig:tc-thickness-vs-ct}, which collapses the data from all four thicknesses onto a single trend when plotted against $C_T$ rather than energy. The correlation demonstrates that the discrepancy is not a thickness-specific artifact but rather grows systematically with the strength of the self-shielding correction itself.

\begin{figure}[H]
    \centering
    \includegraphics[width=0.75\textwidth]{Transmission Correction/Figures/thickness_error_vs_ct.pdf}
    \caption{Relative deviation of SESH from MCNP as a function of the MCNP correction factor magnitude. Data from all four sample thicknesses collapse onto a common trend, confirming that the discrepancy scales with the strength of the self-shielding correction rather than with thickness independently.}
    \label{fig:tc-thickness-vs-ct}
\end{figure}

That the disagreement is a modeling difference rather than a numerical precision issue is confirmed by \autoref{fig:tc-thickness-residual}, which expresses the residual in units of the SESH Monte Carlo statistical uncertainty. At all energies and thicknesses, the residuals far exceed the $\pm 2\sigma$ band of the SESH calculation, demonstrating that the SESH estimator has fully converged and that the observed discrepancy reflects a genuine difference in the self-shielding predictions of the two methods.

\begin{figure}[H]
    \centering
    \includegraphics[width=0.75\textwidth]{Transmission Correction/Figures/thickness_residual_sigma.pdf}
    \caption{Residual between SESH and MCNP correction factors expressed in units of the SESH Monte Carlo statistical uncertainty ($\sigma$). The shaded band indicates $\pm 2\sigma$. The residuals exceed the statistical precision of the SESH calculation at low energies, confirming that the disagreement is a modeling difference rather than numerical noise.}
    \label{fig:tc-thickness-residual}
\end{figure}

This behavior points to a modeling difference between SESH and the MCNP benchmark rather than an error in either code. SESH samples cross sections directly from the resonance parameter distributions, while the MCNP benchmark represents the URR cross-section distribution through pre-computed probability tables generated by NJOY. These two representations need not produce identical self-shielding corrections, particularly as the correction grows large and the result becomes more sensitive to the details of the sampled cross-section distribution. The systematic nature of the disagreement (always in the same direction, and scaling with $C_T$) is consistent with a difference in how the two methods represent the underlying cross-section fluctuations.

This level of disagreement between independent URR self-shielding methods is not unexpected. The probability table approach involves several layers of approximation. Each of which can introduce subtle distortions that compound as the self-shielding correction grows. Differences between probability table implementations and direct Monte Carlo sampling have been documented across multiple independent studies \cite{Sublet2009_CEAR6227_URRPT_SelfShield, Cullen2010_ENDF369_URRHistory, Holcomb2017_NewURRProbTables, Brown2024_URRProbTableTests}. These investigations have identified various sources of discrepancy, including sensitivity to the number of probability table bins, ambiguities in the ENDF URR format conventions, and limitations of the underlying single-level Breit-Wigner approximation used in table generation. The sub-percent disagreement observed here between SESH and the NJOY-generated probability tables used in MCNP is consistent with the range of inter-code variability reported in these studies.

For the purposes of validation in this chapter, the key conclusion is that the SESH and MCNP methods agree well within the regime most relevant to typical transmission experiments ($C_T < 1.5$), and that the divergence at larger corrections reflects a difference in the underlying cross-section representations rather than a failure of the Monte Carlo sampling procedure. The probability table methodology used by MCNP is investigated further in \autoref{chap:multiiso-transmission-correction}.


\subsection{Verification of Sample Temperature Dependence}
The final parameter investigated was the sample temperature, which influences the Doppler broadening of the resonances. The $^{181}$Ta evaluation was processed at five different temperatures: 100~K, 200~K, 300~K, 400~K, and 500~K. These cross sections were used to calculate the correction factor in MCNP for a 4~mm thick sample, which was then compared to the SESH results at the same temperatures.

The results, presented in \autoref{fig:tc-temperature}, show a very strong agreement between SESH and MCNP across all examined temperatures. The relative error does not exhibit any significant bias with temperature, indicating that the Doppler broadening and other temperature-dependent physics are correctly modeled in SESH for the purpose of calculating the transmission correction factor.

 \begin{figure}[H]
     \centering
     \begin{subfigure}[b]{0.48\textwidth}
         \centering
         \includegraphics[width=\textwidth]{Transmission Correction/Figures/temperature_CT.pdf}
         \caption{SESH correction factor (solid lines) compared to MCNP (open circles) at each temperature.}
     \end{subfigure}
     \begin{subfigure}[b]{0.48\textwidth}
         \centering
         \includegraphics[width=\textwidth]{Transmission Correction/Figures/temperature_error.pdf}
         \caption{Relative deviation of SESH from MCNP as a function of incident energy at each temperature.}
     \end{subfigure}
     \caption{Transmission correction factor for a 4~mm $^{181}$Ta sample at temperatures from 100~K to 500~K. The relative deviation remains below 0.125\% across all temperatures and energies, with no systematic temperature-dependent bias.}
     \label{fig:tc-temperature}
 \end{figure}

\section{Integration with the Parameter Fitting Workflow}
After verifying the standalone accuracy of the SESH model, the next crucial step is to integrate it into the SAMMY resonance parameter fitting code. This integration transforms the correction factor from a simple calculation into a dynamic component of the fitting process, allowing experimental transmission data to be fit directly. This requires not only the correction factor itself, but also its derivative with respect to the parameters being adjusted.

\subsection{Numerical Estimation of the Correction Factor Derivative}
The Bayesian fitting algorithm in SAMMY, like many optimization routines, relies on the partial derivative of the theoretical model with respect to each fitting parameter, $u$. For the theoretical transmission, $\langle T \rangle = C_T e^{-n\langle\sigma\rangle}$, the derivative is given by the product rule:
\begin{equation}
    \frac{\partial \langle T \rangle}{\partial u} = e^{-n\langle\sigma\rangle} \frac{\partial C_T}{\partial u} + C_T \frac{\partial e^{-n\langle\sigma\rangle}}{\partial u}
\end{equation}
While the derivative of the Hauser-Feshbach term, $\partial e^{-n\langle\sigma\rangle}/\partial u$, is calculated analytically within SAMMY, the derivative of the correction factor, $\partial C_T / \partial u$, is not straightforward as $C_T$ is the output of a Monte Carlo process.

To solve this, a numerical approach was implemented. For a given parameter $u$ (such as the s-wave strength function, $S_0$), SESH samples the parameter from a narrow distribution around its central value. The resulting distribution of $C_T$ values is then plotted against the sampled parameter values, as shown in \autoref{fig:ct-derivative-regression}. A linear regression is performed on these points, and the slope of the best-fit line is taken as the estimate of the derivative $\partial C_T / \partial u$. This linear regression method was found to be more statistically efficient than a standard finite difference approach, providing a much lower variance in the derivative estimate for the same number of particle histories, as illustrated in \autoref{fig:ct-derivative-variance}.

\begin{figure}[H]
    \centering
    \includegraphics[width=0.75\linewidth]{Transmission Correction/Figures/ct_vs_u_replotted_with_S0_topaxis.pdf}
    \caption{Estimation of $\partial C_T / \partial u$ for $S_0$ at 3~keV using linear regression on the jointly sampled $(S_0, C_T)$ pairs. The slope of the best-fit line provides the derivative estimate.}
    \label{fig:ct-derivative-regression}
\end{figure}

The advantage of the regression approach over finite differences is quantified in \autoref{fig:ct-derivative-variance} and \autoref{fig:derivative-variance-reduction}. At a fixed number of histories, the regression estimate exhibits substantially lower variance than the finite difference estimate, because it exploits the correlation structure across the full set of sampled parameter values rather than relying on the difference between only two evaluations. \autoref{fig:derivative-variance-reduction} confirms that both methods converge as $N$ increases, but the regression estimator reaches a given precision threshold with significantly fewer histories. In practice, this means that the regression derivative can be computed at negligible additional cost during the same Monte Carlo run already required for $C_T$ itself.

\begin{figure}[H]
    \centering
    \includegraphics[width=0.75\linewidth]{Transmission Correction/Figures/DerivativeEstimation.pdf}
    \caption{Comparison of derivative estimate distributions from the linear regression and finite difference methods at 3~keV for \textsuperscript{181}Ta, illustrating the lower variance achieved by the regression approach for the same number of Monte Carlo histories.}
    \label{fig:ct-derivative-variance}
\end{figure}

\begin{figure}[H]
    \centering
    \includegraphics[width=0.75\linewidth]{Transmission Correction/Figures/DerivativeMethod.pdf}
    \caption{Variance of the derivative estimate as a function of the total number of Monte Carlo histories for both the linear regression and finite difference methods. The regression method converges more rapidly, achieving the same precision with fewer histories.}
    \label{fig:derivative-variance-reduction}
\end{figure}


\subsection{Verification of the Integrated Fitting Procedure}

With a method to calculate both $C_T$ and its derivative established, the fully integrated SESH+SAMMY workflow can be verified end-to-end. The goal of this verification is to demonstrate that the fitting procedure can recover known resonance parameters from synthetic data, analogous to the verification tests used for the capture yield correction in \autoref{chap:capture-yield-correction} and the multi-isotope extension in \autoref{chap:multiiso-transmission-correction}.

Synthetic transmission data for multiple sample thicknesses of $^{181}$Ta were generated using MCNP with the ENDF/B-VIII.1 evaluation parameters as the ground truth. The starting parameters for the fit were drawn from the Atlas of Neutron Resonances \cite{atlas}, which provides an older, independently compiled set of average resonance parameters for $^{181}$Ta. This choice ensures that the initial guess is physically reasonable but deliberately offset from the true values, so that the fit must traverse a meaningful region of parameter space to converge. Only the neutron strength functions ($S_0$, $S_1$, $S_2$) were varied during the fit; all other parameters, including the distant-level contribution and average radiative widths, were held at their evaluated values.

The fitting procedure was executed twice under identical conditions: once with the numerically estimated derivative $\partial C_T / \partial u$ included in the Jacobian, and once with the assumption $\partial C_T / \partial u = 0$. The latter case corresponds to the implicit approximation used in all previous self-shielding correction workflows, in which the correction factor is recomputed at each iteration but treated as independent of the fitted parameters when forming the update step.

The fitted transmission curves for both cases are compared against the synthetic data in \autoref{fig:fitting-comparison}. Both fits converge to final predictions that visually reproduce the synthetic data, indicating that the integrated workflow functions correctly in either mode.

\begin{figure}[H]
    \centering
    \includegraphics[width=0.75\linewidth]{Transmission Correction/Figures/DCtFittingPerformance.pdf}
    \caption{Fitted self-shielded transmission compared to the synthetic MCNP data for $^{181}$Ta. Results are shown for both the derivative-enabled ($\partial C_T / \partial u \neq 0$) and derivative-free ($\partial C_T / \partial u = 0$) fitting modes.}
    \label{fig:fitting-comparison}
\end{figure}

The quantitative comparison is summarized in Tables~\ref{tab:fitting-error} and \ref{tab:fitting-values}. Enabling the derivative calculation yields a modestly lower reduced chi-square ($\chi^2/N = 1.981$ versus $2.444$), confirming that the additional Jacobian information does improve the fit. More importantly, the recovered strength function values in \autoref{tab:fitting-values} are close to the ENDF/B-VIII.1 truth values in both cases, demonstrating that the integrated workflow successfully recovers the parameters used to generate the synthetic data regardless of whether the derivative is included.

\begin{table}[H]
    \centering
    \caption{Final reduced chi-square ($\chi^2/N$) at convergence for the two fitting modes applied to synthetic $^{181}$Ta transmission data.}
    \begin{tabular}{| c | c c |}
        \hline
                               & $\partial C_T / \partial u$=ON  & $\partial C_T / \partial u$=OFF \\ 
                               \hline
        $\chi^2/N$ & 1.981  & 2.444 \\ 
        \hline
    \end{tabular}
    \label{tab:fitting-error}
\end{table}


\begin{table}[H]
    \centering
    \caption{Recovered $^{181}$Ta neutron strength functions from the synthetic data fit. The ENDF/B-VIII.1 column gives the true values used to generate the data; the two fitting modes are compared.}
    \begin{tabular}{|c|ccc|}
        \hline
        Parameter & ENDF-8.1  & $\partial C_T / \partial u =$ON  & $\partial C_T / \partial u =$OFF \\
        \hline
        $S_0 \times 10^{-4}$   & $1.740 \pm 0.03$ & 1.807 & 1.752\\
        $S_1 \times 10^{-4}$   & $0.800 \pm 0.07$ & 0.620 & 0.751\\
        $S_2 \times 10^{-4}$   & $1.690 \pm 0.18$ & 1.486 & 1.499\\
        \hline
    \end{tabular}
    \label{tab:fitting-values}
\end{table}

The residual differences between the recovered and true strength functions in \autoref{tab:fitting-values} merit brief comment. The $S_0$ and $S_2$ values are recovered to within approximately 5--10\% of the evaluation in both modes, while $S_1$ shows a larger residual offset in the derivative-enabled case. These offsets are not unexpected for a fit to synthetic transmission data alone: transmission measurements are most sensitive to the total cross section, which in the URR is dominated by $s$-wave and $d$-wave contributions. The $p$-wave strength function has a comparatively weaker effect on the total cross section at these energies, making it inherently more difficult to constrain from transmission data in isolation. The key observation is that both fitting modes converge to physically reasonable values from a deliberately incorrect starting point, confirming that the integrated workflow is functioning correctly.


\section{Summary of Monotopic Transmission Correction Validation}
\label{ssec:tc-monotopic-summary}

The verification exercises detailed in this chapter confirm that the SESH methodology for calculating the transmission correction factor, $C_T$, is accurate for a range of experimental conditions. The model shows strong agreement with high-fidelity MCNP reference models across various sample temperatures and for correction factors up to approximately $C_T \approx 1.5$. At larger correction factors a systematic discrepancy emerges that scales with the magnitude of $C_T$ itself, consistent with known differences between direct Monte Carlo sampling and probability table representations of URR cross-section fluctuations.

A key investigation was the integration of the correction factor's derivative, $\partial C_T / \partial u$, into the fitting procedure. While the numerical estimation of this derivative via linear regression was shown to be statistically efficient, its impact on the final fitting results was found to be modest. As shown in \autoref{tab:fitting-error}, enabling the derivative yields a lower final $\chi^2/N$ ($1.981$ versus $2.444$), but the recovered parameters are comparable in both cases and both fits converge from a deliberately incorrect starting point. The improvement does not justify the significant increase in computational cost, particularly given that all previous self-shielding correction workflows have implicitly operated under the assumption $\partial C_T / \partial u = 0$.

Given the minor improvement in fit quality versus the computational cost, the most practical approach for future evaluations is to proceed with $\partial C_T / \partial u = 0$ while fully recomputing $C_T$ at each iteration. This simplification is consistent with established methodologies and produces results that are sufficiently accurate for parameter recovery, as further demonstrated in the capture yield fitting verification (\autoref{chap:capture-yield-correction}) and the evaluation presented in \autoref{chap:evaluation}.
