This chapter details and validates the procedure for calculating the theoretical average transmission to accurately fit parameters to self-shielded transmission measurements in the URR. As shown in \autoref{sec:correction-factor}, the theoretical average transmission is calculated as the product of the transmission of the average cross section, $e^{-n\langle \sigma \rangle}$, and a transmission correction factor $C_T$. While $C_T$ was previously defined in \autoref{eq:transmission-correction-factor} with an analytical form over an energy range, this procedure is not feasible in the URR where the true cross-section is unknown. Instead, SESH uses Monte Carlo integration in order to determine these energy-averaged quantities.

The accuracy of this method will be validated using $^{181}$Ta, for which a recent evaluation provides new resonance parameters. The validation is twofold. First, the theoretical model is compared against MCNP simulations to provide a detailed characterization of its performance. As the MCNP model utilizes the exact best-fit parameters from the recent $^{181}$Ta evaluation, it serves as an ideal benchmark to investigate the effects of physical parameters such as sample thickness and temperature. Second, the calculated average transmission is compared against the experimental data that was used in the evaluation. This comparison serves to validate the theoretical model's overall performance for a single sample thickness, serving as a final integration test for SESH’s theoretical self-shielded transmission calculation.
