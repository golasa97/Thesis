The preceding chapters established the Monte Carlo procedure for computing the transmission correction factor $C_T$ from average resonance parameters (\autoref{sec:sampling-procedure}) and described its implementation within SESH (\autoref{chap:implementation}). This chapter validates that implementation against an independent computational benchmark and demonstrates its integration into SAMMY's parameter fitting workflow. The validation uses $^{181}$Ta, for which a recent evaluation provides well-characterized resonance parameters\cite{Brown2024}, and proceeds in two stages. First, the standalone accuracy of the $C_T$ calculation is verified by comparison to high-fidelity MCNP\cite{mcnp} simulations across a range of sample thicknesses, temperatures, and model settings. Second, the correction factor is embedded in SAMMY's fitting loop and tested on synthetic data to confirm that the integrated code can recover known parameters from self-shielded transmission measurements.

\section{MCNP Reference Model}
\label{sec:mcnp-benchmark-model}

Validating the SESH correction factor requires a reference calculation that is independent of the statistical sampling approximations inherent to the SESH methodology. MCNP\cite{mcnp} provides this reference: both MCNP and SAMMY/SESH use the same evaluated nuclear data, so the comparison primarily tests the correction method and its implementation rather than differences in input cross sections.

The MCNP model was configured to simulate neutron transmission through a $^{181}$Ta sample. The cross-section data for the simulation was prepared from the ENDF/B-VIII.1 evaluation\cite{endf-manual} and processed into a pointwise format using NJOY\cite{njoy}. By using the exact, finely-resolved cross sections from a formal evaluation, the MCNP simulation can directly calculate the true average transmission, $\langle T \rangle$, and the transmission of the average cross section, $e^{-n\langle\sigma\rangle}$, over a given energy group. The benchmark correction factor is then calculated as:
\begin{equation}
    C_{T, \text{MCNP}} = \frac{\langle T \rangle_{\text{P-Table=ON}}}{\langle T \rangle_{\text{P-Table=OFF}}}
\end{equation}
This provides an ideal reference value, free from the statistical approximations inherent to the SESH methodology, against which SESH's calculations can be validated. The following sections use this MCNP benchmark to systematically investigate the sensitivity of the SESH-calculated $C_T$ to three key physical parameters: the number of external resonance pairs used in the calculation, the physical thickness of the sample, and the temperature of the sample.

\section{Verification of the Correction Factor}
\label{sec:ct-verification}

\subsection{Influence of External Resonance Pairs}
\label{ssec:resonance-pairs}

The SESH code includes a feature to add resonance pairs at energies outside of the specific energy bin being analyzed. This is intended to account for the contributions from the tails of distant resonances, which can still influence the cross section within the bin. To quantify the impact of this feature, the correction factor was calculated for a 4~mm $^{181}$Ta sample at 300~K using an increasing number of external resonance pairs.

The results, shown in \autoref{fig:tc-resonance-pairs}, demonstrate that the inclusion of external resonance pairs has a slight but noticeable influence on the calculated correction factor. As more pairs are added, the SESH calculation converges toward the MCNP benchmark value, with the relative error decreasing correspondingly. For the remainder of this work, four external resonance pairs were used in all SESH calculations to ensure a balance between accuracy and computational efficiency.

\begin{figure}[H]
    \centering
    \begin{subfigure}[b]{0.48\textwidth}
        \centering
        \includegraphics[width=\textwidth]{Transmission Correction/Figures/ResonancePairInfluenceOnCT.png}
        \caption{Correction factor comparison}
        \label{fig:tc-resonance-pairs-data}
    \end{subfigure}
    \hfill
    \begin{subfigure}[b]{0.48\textwidth}
        \centering
        \includegraphics[width=\textwidth]{Transmission Correction/Figures/ResonancePairError.png}
        \caption{Relative error}
        \label{fig:tc-resonance-pairs-error}
    \end{subfigure}
    \caption{Transmission correction factor for a 4~mm $^{181}$Ta sample at 300~K as a function of the number of external resonance pairs. (a) SESH calculations compared to the MCNP benchmark. (b) Relative error between SESH and MCNP.}
    \label{fig:tc-resonance-pairs}
\end{figure}


\subsection{Sample Thickness Dependence}
\label{ssec:thickness}

A critical parameter in self-shielding experiments is the thickness of the sample\cite{Dresner1960}. To determine the range of applicability for SESH, various thicknesses of $^{181}$Ta were simulated, specifically 2~mm, 4~mm, 8~mm, and 12~mm. The correction factors calculated by SESH were then compared against the MCNP benchmark for each thickness.

As shown in \autoref{fig:tc-thickness}, there is very strong agreement between SESH and MCNP for the thinner samples. However, as the sample thickness increases, SESH begins to under-predict the correction factor. This deviation becomes more pronounced for thicker samples, particularly where the correction factor $C_T$ exceeds a value of approximately 1.5. To maintain a disagreement of less than 1\%, the use of this method should be limited to samples where the calculated correction factor remains below this threshold. As a practical guideline, experimental designs that keep $C_T \leq 1.5$ avoid the edge of applicability observed in this benchmark.

\begin{figure}[H]
    \centering
    \begin{subfigure}[b]{0.48\textwidth}
        \centering
        \includegraphics[width=\textwidth]{Transmission Correction/Figures/ct-verification.png}
        \caption{Correction factor comparison}
        \label{fig:tc-thickness-data}
    \end{subfigure}
    \hfill
    \begin{subfigure}[b]{0.48\textwidth}
        \centering
        \includegraphics[width=\textwidth]{Transmission Correction/Figures/ct-verification-err.png}
        \caption{Relative error}
        \label{fig:tc-thickness-error}
    \end{subfigure}
    \caption{Transmission correction factor for various thicknesses of a $^{181}$Ta sample at 300~K. (a) SESH calculations compared to the MCNP benchmark. (b) Relative error between SESH and MCNP, showing increasing deviation for thicker samples where $C_T$ exceeds 1.5.}
    \label{fig:tc-thickness}
\end{figure}


\subsection{Sample Temperature Dependence}
\label{ssec:temperature}

The final parameter investigated was the sample temperature, which influences the Doppler broadening of the resonances. The $^{181}$Ta evaluation was processed at five different temperatures: 100~K, 200~K, 300~K, 400~K, and 500~K. These cross sections were used to calculate the correction factor in MCNP for a 4~mm thick sample, which was then compared to the SESH results at the same temperatures.

The results, presented in \autoref{fig:tc-temperature}, show very strong agreement between SESH and MCNP across all examined temperatures. The relative error does not exhibit any significant bias with temperature, indicating that the Doppler broadening and other temperature-dependent physics are correctly modeled in SESH for the purpose of calculating the transmission correction factor.

\begin{figure}[H]
    \centering
    \begin{subfigure}[b]{0.48\textwidth}
        \centering
        \includegraphics[width=\textwidth]{Transmission Correction/Figures/SESH_MCNP_Temperature.png}
        \caption{Correction factor comparison}
        \label{fig:tc-temperature-data}
    \end{subfigure}
    \hfill
    \begin{subfigure}[b]{0.48\textwidth}
        \centering
        \includegraphics[width=\textwidth]{Transmission Correction/Figures/ct_temperature_err.png}
        \caption{Relative error}
        \label{fig:tc-temperature-error}
    \end{subfigure}
    \caption{Transmission correction factor at various temperatures for a 4~mm $^{181}$Ta sample. (a) SESH calculations compared to the MCNP benchmark. (b) Relative error between SESH and MCNP, showing no systematic temperature dependence.}
    \label{fig:tc-temperature}
\end{figure}


\section{Integration with the Parameter Fitting Workflow}
\label{sec:fitting-integration}

After verifying the standalone accuracy of the SESH model, the next step is to integrate it into the SAMMY resonance parameter fitting code. This integration transforms the correction factor from a standalone calculation into a dynamic component of the fitting process, allowing experimental transmission data to be fit directly.

\subsection{Numerical Estimation of the Correction Factor Derivative}
\label{ssec:derivative-estimation}

The Bayesian fitting algorithm in SAMMY, like many optimization routines, relies on the partial derivative of the theoretical model with respect to each fitting parameter, $u$. For the theoretical transmission, $\langle T \rangle = C_T e^{-n\langle\sigma\rangle}$, the derivative is given by the product rule:
\begin{equation}
    \frac{\partial \langle T \rangle}{\partial u} = e^{-n\langle\sigma\rangle} \frac{\partial C_T}{\partial u} + C_T \frac{\partial e^{-n\langle\sigma\rangle}}{\partial u}
\end{equation}
While the derivative of the Hauser-Feshbach term, $\partial e^{-n\langle\sigma\rangle}/\partial u$, is calculated analytically within SAMMY, the derivative of the correction factor, $\partial C_T / \partial u$, is not straightforward as $C_T$ is the output of a Monte Carlo process.

To solve this, a numerical approach was implemented. For a given parameter $u$ (such as the s-wave strength function, $S_0$), SESH samples the parameter from a narrow distribution around its central value. The resulting distribution of $C_T$ values is then plotted against the sampled parameter values, and a linear regression is performed on these points. The slope of the best-fit line is taken as the estimate of the derivative $\partial C_T / \partial u$. An example of this procedure is shown in \autoref{fig:ct-derivative-regression}.

\begin{figure}[H]
    \centering
    \includegraphics[width=0.75\linewidth]{Transmission Correction/Figures/DerivativeEstimationExample.png}
    \caption{Estimation of $\partial C_T / \partial u$ for $S_0$ at 3~keV using linear regression. Each point represents a single Monte Carlo calculation at a perturbed parameter value; the slope of the regression line provides the derivative estimate.}
    \label{fig:ct-derivative-regression}
\end{figure}

This linear regression method was found to be more statistically efficient than a standard finite difference approach. \autoref{fig:ct-derivative-variance} compares the two methods: the left panel shows the distribution of derivative estimates from repeated trials at a single energy, illustrating the substantially lower spread of the regression method, while the right panel shows that the regression approach achieves lower variance at every level of computational effort, converging more rapidly with the number of Monte Carlo histories.

\begin{figure}[H]
    \centering
    \begin{subfigure}[b]{0.48\textwidth}
        \centering
        \includegraphics[width=\textwidth]{Transmission Correction/Figures/DerivativeEstimation.png}
        \caption{Variance comparison at 3~keV}
        \label{fig:ct-derivative-variance-comparison}
    \end{subfigure}
    \hfill
    \begin{subfigure}[b]{0.48\textwidth}
        \centering
        \includegraphics[width=\textwidth]{Transmission Correction/Figures/DerivativeMethod.png}
        \caption{Variance vs.\ particle histories}
        \label{fig:ct-derivative-variance-scaling}
    \end{subfigure}
    \caption{Statistical efficiency of the linear regression derivative estimation method compared to finite differences for \textsuperscript{181}Ta. (a) Distribution of derivative estimates at 3~keV. (b) Variance as a function of total particle histories, demonstrating faster convergence of the regression approach.}
    \label{fig:ct-derivative-variance}
\end{figure}


\subsection{Verification of the Integrated Fitting Procedure}
\label{ssec:fitting-verification}

With a method to calculate both $C_T$ and its derivative, the fully integrated SESH+SAMMY workflow could be verified. The primary goal of this verification is to demonstrate that the fitting procedure can accurately recover known resonance parameters from synthetic experimental data.

The verification test was structured as follows:
\begin{enumerate}[noitemsep]
    \item Synthetic transmission data for multiple sample thicknesses was generated using MCNP with the known $^{181}$Ta resonance parameters from the ENDF/B-VIII.1 evaluation.
    \item An initial set of incorrect resonance parameters was chosen from an older compilation, the Atlas of Neutron Resonances \cite{atlas}.
    \item The SESH+SAMMY code was used to fit the synthetic MCNP data, starting from the incorrect parameters.
\end{enumerate}

The fitting process was performed twice: once with the numerical derivative calculation for $\partial C_T / \partial u$ enabled, and once assuming $\partial C_T / \partial u = 0$. The latter case represents the implicit assumption made in previous, manual correction workflows. The results, shown in \autoref{fig:fitting-comparison}, demonstrate that both fits converge to similar values. However, as detailed in \autoref{tab:fitting-error}, enabling the derivative calculation produces a more accurate fit with a lower final $\chi^2/N$. The final converged strength function values in \autoref{tab:fitting-values} confirm that the new, fully integrated workflow successfully recovers the true parameters used to generate the synthetic data, verifying that the integration is correct and the derivative estimation is sufficiently accurate for use in the fitting procedure.

\begin{figure}[H]
    \centering
    \includegraphics[width=0.75\linewidth]{Transmission Correction/Figures/DCtFittingPerformance.png}
    \caption{Fitting result comparison between numerically estimating the derivative $\partial C_T/\partial u$ versus the assumption $\partial C_T/\partial u=0$ for all parameters.}
    \label{fig:fitting-comparison}
\end{figure}


\begin{table}[H]
    \centering
    \caption{Final reduced chi-square ($\chi^2/N$) at convergence for the two derivative treatments.}
    \begin{tabular}{l c c}
        \hline
                               & $\partial C_T / \partial u$ enabled  & $\partial C_T / \partial u = 0$ \\ 
                               \hline
        $\chi^2/N$ & 1.981  & 2.444 \\ 
        \hline
    \end{tabular}
    \label{tab:fitting-error}
\end{table}


\begin{table}[H]
    \centering
    \caption{Final converged strength function values compared to the ENDF/B-VIII.1 reference parameters.}
    \begin{tabular}{l c c c}
        \hline
        Parameter & ENDF-8.1  & $\partial C_T / \partial u$ enabled  & $\partial C_T / \partial u = 0$ \\
        \hline
        $S_0 \times 10^{-4}$   & $1.740 \pm 0.03$ & 1.807 & 1.752\\
        $S_1 \times 10^{-4}$   & $0.800 \pm 0.07$ & 0.620 & 0.751\\
        $S_2 \times 10^{-4}$   & $1.690 \pm 0.18$ & 1.486 & 1.499\\
        \hline
    \end{tabular}
    \label{tab:fitting-values}
\end{table}


\section{Summary}
\label{sec:tc-summary}

The verification exercises in this chapter confirm that the SESH methodology for calculating the transmission correction factor is accurate across a range of experimental conditions. The model shows strong agreement with high-fidelity MCNP reference calculations across sample temperatures from 100--500~K and for sample thicknesses where $C_T$ remains below approximately 1.5, which covers the regime relevant to practical transmission measurements. The investigation of the correction factor derivative showed that while the linear regression approach provides a statistically efficient estimate of $\partial C_T / \partial u$, the improvement in fitting performance is modest (\autoref{tab:fitting-error}). Given the significant computational cost and the fact that historical workflows have implicitly operated under the assumption $\partial C_T / \partial u = 0$, this simplification is adopted for all subsequent evaluations.
