\section{Introduction}
\label{sec:capture-yield-introduction}

The self-shielding correction framework developed and validated for transmission measurements in \autoref{chap:transmission-correction} and \autoref{chap:multiiso-transmission-correction} is now extended to another critical experimental observable: capture yield. Accurately modeling capture yield in the unresolved resonance region (URR) requires accounting for two distinct physical phenomena. The first is the resonance self-shielding effect, driven by the statistical fluctuations in the microscopic cross section, which has been the primary focus of the preceding chapters. The second, which is particularly pronounced in capture experiments, is the effect of neutron multiple scattering within the measurement sample.

This chapter will not repeat the foundational discussion on the underlying physics of Monte Carlo cross-section generation and the statistical nature of resonance self-shielding, as these topics have been thoroughly described in \autoref{chap:implementation}, \autoref{chap:transmission-correction}, and \autoref{chap:multiiso-transmission-correction}. Instead, it will focus on the theoretical and implementation aspects unique to the capture yield correction. Specifically, this chapter details the physics of the multiple scattering correction, presents a verification of the physics model against a high-fidelity MCNP benchmark, and validates the integrated fitting procedure using synthetic data.


\section{Multiple Scattering in a Cylindrical Sample}
\label{sec:multiple-scattering}

Multiple scattering is an effect which occurs in which a neutron is scattered one or more times before being captured in a sample. As the thickness of the sample increases, the probability of a scattered neutron experiencing a subsequent reaction (that being a capture or additional scattering events) also increases. Therefore, neutrons which undergo scattering events can end up miscounted as captures, artificially increasing the measured yield.
\begin{figure}[H]
    \centering
    \includegraphics[width=0.75\linewidth]{Capture Yield/Figures/multiplescattering.png}
    \caption{Illustration of a neutron undergoing multiple scattering events, and getting miscounted as a capture event.}
    \label{fig:multiple-scattering-example}
\end{figure}
This is illustrated in \autoref{fig:multiple-scattering-example}, in which a neutron scatters twice before being captured.

This correction is made with reference to the thin sample approximation,
\begin{equation}
    \label{eq:thin-sample-approximation}
    \left\langle Y \right\rangle \approx n \left\langle \sigma_\gamma \right\rangle 
\end{equation}
where $\left\langle Y \right\rangle$ is the average yield, $n$ is the sample thickness, and $\left\langle \sigma_\gamma \right\rangle$ is the average capture cross section. The capture correction $C_C$ corrects the thin sample approximation such that
\begin{equation}
    \label{eq:correction-thin-sample-approximation}
    \left\langle Y \right\rangle = n \left\langle \sigma_\gamma \right\rangle C_C
\end{equation}

In order to calculate the correction, a series of Monte Carlo simulations are used to simulate a neutron undergoing a series of capture or scattering events in a cylindrical disc target. The target has a radius $R$, is parallel to the $z$ axis, and with front and rear faces located at $z=0$ and $z=n$, respectively, which are centered at $(x,y)=(0,0)$. A neutron is born at the energy of interest, $E^0$, and is incident on the front face of the target, with the location
\begin{equation}
    \label{eq:incident-neutron-location}
    \overrightarrow{x}^0 = \begin{bmatrix}
        x^0 \\
        y^0 \\
        z^0
    \end{bmatrix} =
    \begin{bmatrix} 
        r_0\cos{\left( \theta_0 \right)} \\
        r_0\sin{\left( \theta_0 \right)} \\
        0
    \end{bmatrix}
\end{equation}
where
\begin{align*}
    r_0 &= R\sqrt{\zeta}, \\
    \theta_0 &= 2\pi\zeta,
\end{align*}
and $\zeta$ is a uniformly distributed random value between 0 and 1. The neutron is also traveling along the $z$-axis, and therefore is born with the direction vector
\begin{equation}
    \label{eq:incident-neutron-direction}
    \overrightarrow{\Omega}^0 = \begin{bmatrix}
        \Omega^0_u \\
        \Omega^0_v \\
        \Omega^0_w
    \end{bmatrix} =
    \begin{bmatrix}
        0 \\
        0 \\
        1
    \end{bmatrix}
\end{equation}

Cross sections are sampled at the energy $E^0$ according to the procedure described in \autoref{sec:mc-sampling-from-average-parameters}. These values are then used to produce the average cross-sections,
\begin{align}
    \label{eq:average-total-cross-section}
    \overline{ \sigma_{t} }^{0} &= \sum_{j} \sigma_{t,j}^{0} \delta_{j} \\
    \label{eq:average-capture-cross-section}
    \overline{ \sigma_{\gamma} }^{0} &= \sum_{j} \sigma_{\gamma,j}^{0} \delta_{j} \\
    \label{eq:average-scattering-cross-section}
    \overline{ \sigma_{sc} }^{0} &= \sum_{j} \sigma_{sc,j}^{0} \delta_{j}
\end{align}
where $\overline{ \sigma_{t} }$, $\overline{ \sigma_{\gamma} }$, and $\overline{ \sigma_{sc} }$  are the weighted average total, capture, and scattering cross-sections respectively. The $0$ superscript indicates that the cross-sections are taken at the energy $E^{0}$, i.e., before any scattering events have occurred. The terms $\sigma_{j}^0$ and $\delta_{j}$ are the microscopic cross-section and relative abundance of the $j^{th}$ isotope, respectively.

\section{Scattering Events}

\subsection{Sampling Location}
\label{ssec:sampling-location-ms}
The distance to leave the sample is then calculated according to the shortest path to intersect with one of the surfaces in the direction in which the neutron is traveling. For the initial neutron, that is, before any scattering events have occurred, this is just the thickness of the sample $n$. However, for a scattering event this must be solved generally for a neutron that undergone $k$ scattering events. In this case, the neutron would have position vector $\overrightarrow{x}^k$ and direction $\overrightarrow{\Omega}^k$. The quantity that must be determined is the shortest path for it to leave the sample. In the case of a cylindrical target, there are three separate surfaces in which the neutron could escape:
\begin{enumerate}
    \item The front surface at $z=0$,
    \item The rear surface at $z=n$, and
    \item The cylindrical surface at $x^2 + y^2 = R^2$.
\end{enumerate}
Therefore, there are three distances that must be calculated:
\begin{align*}
    d_{front}   &\equiv \text{Distance neutron must travel to intersect with front face} \\
    d_{back}    &\equiv \text{Distance neutron must travel to intersect with back face} \\
    d_{cyl}     &\equiv \text{Distance neutron must travel to intersect with cylindrical surface}
\end{align*}
\begin{figure}[h]
    \centering
    \includegraphics[width=0.75\linewidth]{Capture Yield/Figures/InteractionSideDemo.pdf}
    \caption{Example of the face intersection scheme, resulting in $d_{front}$ being selected as the shortest face until the particle exits.}
    \label{fig:ms-face-intersection}
\end{figure}
The distances to the planar front and back faces ($d_{front}$ and $d_{back}$) are determined. A finite, positive distance is calculated only if the neutron's path is directed towards the face in question; otherwise, the distance is treated as infinite. Using $\Omega_w^k$ to denote the z-component of the direction vector $\overrightarrow{\Omega}^k$, the distances are:
\begin{align}
    d_{front} &= \begin{cases} -z^k / \Omega_w^k, & \text{if } \Omega_w^k < 0 \\ \infty, & \text{otherwise} \end{cases} \\
    d_{back} &= \begin{cases} (n - z^k) / \Omega_w^k, & \text{if } \Omega_w^k > 0 \\ \infty, & \text{otherwise} \end{cases}
\end{align}
while $d_{cyl}$ is is determined as
\begin{equation}
    \label{eq:cyl-intersection-distance}
    d_{cyl} = \frac{\sqrt{b^2 + ac} - b}{a}
\end{equation}
where
\begin{align}
    a &= \left( \Omega_{u}^{k} \right)^2 + \left( \Omega_{v}^{k} \right)^2\\
    b &= x^k \Omega_{u}^{k} + y^k\Omega_{v}^{k} \\
    c &= R^2 - \left( x^k \right)^2 - \left( y^k \right)^2
\end{align}

The path length until the neutron escapes the sample, $d^k$, is the shortest of these three potential path lengths:
\begin{equation}
    d^k = \min{\left(d_{front}, d_{back}, d_{cyl}\right)}
\end{equation}

Next, a new set of total, capture, and elastic cross sections are sampled at energy $E^k$, which are then used to determine the sampled distance until the next collision,
\begin{equation}
    \label{eq:free-path-sampling}
    s^k = -\frac{1}{\overline{\sigma_{t} }^{k} }  \ln{\left\{
                1 - \zeta \left[ 1 - \exp{\left( \overline{\sigma_{t}}^{k} d^k \right)} \right] 
    \right\}}
\end{equation}
where $\zeta$ is a randomly selected value between 0 and 1. This distance term $s^{k}$ is then used to calculate the location of the next scattering event,
\begin{equation}
    \label{eq:sampling-new-location}
    \overrightarrow{x}^{k+1} = \overrightarrow{x}^{k} + \Omega^{k}s^{k}
\end{equation}

\subsection{Sampling Energy}
\label{ssec:sampling-energy-ms}
The energy in which the neutron leaves the scattering event, i.e., $E^{k+1}$, must be determined. This introduces its own complication, as $E^{k+1}$ is dependent on the mass of the isotope it interacts with, along with its scattering angle. 

First, addressing the mass dependency. This quantity must be sampled according to each isotopes calculated scattering cross section and their given abundances, as
\begin{equation}
    w_{j} = \frac{\delta_j \sigma_{j,sc}^{k} }{\overline{\sigma_{sc}}^{k}}
\end{equation}
This quantity is then used to calculate a cumulative weight, such that
\begin{equation}
    W_j = \sum_{j} w_j
\end{equation}
This defines the window of probability for the scattering event occurring on isotope $j$ as being between $(W_{j-1}, W_{j}]$.
A random number $\zeta$ is sampled between 0 and 1, is used such that
\begin{equation}
    W_{j-1} < \zeta \leq W_{j}
\end{equation}
determines that the neutron will be sampled as scattering off a nucleus with the mass $A_j$.

Next, addressing the angle dependency. The scattering angle is assumed to be isotropic, therefore can scatter with the angle $\phi$, determined as
\begin{equation}
    \phi^k = 2\pi\zeta
\end{equation}
Therefore, the energy $E^{k+1}$ is determined as
\begin{equation}
    E^{k+1} = E^{k} \frac{A_j^{2} + 2A_{j}\cos{\left(\phi^{k}\right) + 1}}{\left( A_{j} + 1 \right)^2}
\end{equation}

\subsection{Sampling Direction}
\label{ssec:sampling-direction}
The final component of determining the new parameters of a scattering event include calculating the direction the neutron is traveling, $\overrightarrow{\Omega}^{k+1}$. Similar to \autoref{ssec:sampling-energy-ms}, an angle $\theta^k$ must be sampled uniformly in the range $[0,2\pi]$. However $\theta$ must be converted from center of mass scale to lab scale, where
\begin{align}
    \cos{\theta'} &= \frac{1 + A_j \cos{\left( \theta^k \right)}}{\sqrt{1 + \left( A_j\right)^2 + 2A_j\cos{\left(\theta^k \right)}}} \\
    \sin{\theta'} &= \sqrt{1 - \left( \cos{\theta'} \right)^2}
\end{align}
while $\phi=\phi'$.
Determining $\overrightarrow{\Omega}^{k+1}$ from $\overrightarrow{\Omega}^{k}$, $\phi'$, and $\theta'$ proceed according to
\begin{equation}
    \label{eq:direction-calculation-ms}
    \overrightarrow{\Omega}^{k+1} = \begin{bmatrix}
        \Omega^{k+1}_u \\[8pt]
        \Omega^{k+1}_v \\[8pt]
        \Omega^{k+1}_w
    \end{bmatrix}
    = \begin{bmatrix}
        \frac{\Omega_{u}^{k} \Omega_{v}^{k}} { \sqrt{1 - \left(\Omega_{w}^{k}\right)^2 }} &
        \frac{-\Omega_v^k} { \sqrt{1 - \left(\Omega_{w}^{k}\right)^2 }} &
        \Omega_u \\[10pt]
        \frac{\Omega_{v}^{k} \Omega_{w}^{k}} { \sqrt{1 - \left(\Omega_{w}^{k}\right)^2 }} &
        \frac{-\Omega_u^k} { \sqrt{1 - \left(\Omega_{w}^{k}\right)^2 }} &
        \Omega_v \\[10pt]
        \frac{-1} { \sqrt{1 - \left(\Omega_{w}^{k}\right)^2 }} &
        0 &
        \Omega_w
    \end{bmatrix} \times
    \begin{bmatrix}
        \sin{\theta'}\cos{\phi'} \\[8pt]
        \sin{\theta'}\sin{\phi'} \\[8pt]
        \cos{\theta'}
    \end{bmatrix}
\end{equation}
However, in the case where $\left(\Omega_w^k\right)^2 = 1$, \autoref{eq:direction-calculation-ms} cannot be used as it would divide by 0. Instead, the operation
\begin{equation}
        \overrightarrow{\Omega}^{k+1} = \begin{bmatrix}
        \Omega^{k+1}_u \\[8pt]
        \Omega^{k+1}_v \\[8pt]
        \Omega^{k+1}_w
    \end{bmatrix} = \Omega_w^k     \begin{bmatrix}
        \sin{\theta'}\cos{\phi'} \\[8pt]
        \sin{\theta'}\sin{\phi'} \\[8pt]
        \cos{\theta'}
    \end{bmatrix}
\end{equation}
is used instead. Finally, all the components required to sample the subsequent scattering event are obtained: $\overrightarrow{x}^{k+1}$, $\overrightarrow{\Omega}^{k+1}$, and $E^{k+1}$.

\subsection{Weighting Neutrons}
\label{ssec:killing-neutrons-ms}
A calculation must be made to ensure that sufficient scattering events are being accounted for, but not so many scattering events that it is too computationally expensive, or unrealistic collision events bias the yield in some way.

A weighted importance is used to determine the contribution of a particle to capture yield and when to consider the neutron sufficiently unimportant. This is the same calculation performed by default with MCNP\cite{mcnp}. 

Given the sampled distance $s^{k}$,
\begin{align}
    \label{eq:tot-frac}
    \gamma_{tot}^{k} &=  1 - \exp{ \left(-s^{k} \overline{\sigma_{t}}^{k} \right)} \\
    \label{eq:el-frac}
    \gamma_{sc}^{k} &= \gamma_{tot}^{k} \frac{\overline{\sigma_{sc}}^{k}} {\overline{\sigma_{t}}^{k}} \\
    \label{eq:cap-frac}
    \gamma_{cap}^{k} &= \gamma_{tot}^{k} \frac{\overline{\sigma_{\gamma}}^{k}}{\overline{\sigma_{t}}^{k}}
\end{align}

The simulation employs an implicit capture (or survival biasing) scheme. At each collision, the neutron is forced to scatter, and its statistical weight is reduced by the scattering probability. The "lost" weight is tallied as the capture contribution. Starting with an initial weight $w^{0}=1$, the weight after the $k^{th}$ collision is:
\begin{equation}
    \label{eq:neutron-weight}
    w^{k} = w^{k-1}\gamma_{sc}^{k}
\end{equation}
where $\gamma_{sc}^{k}$ is the scattering interaction fraction from \autoref{eq:el-frac}.

This quantity is used in two ways; first, the factor is used to determine the total number of collisions $K$ that a neutron will experience before it is no longer statistically significant. If the weight ends up less than some cutoff factor, $\varepsilon$, the total number of collisions is reached,
\begin{equation}
    \label{eq:collision-counter}
    K = \min{ \left\{ k : w^{k} \leq \varepsilon \right\}}
\end{equation}

The second purpose of the weighting factor is to determine the total probability of capture after $K$ collisions for a single neutron history. The total probability of a neutron being captured is the sum of all weighted capture events until the collision $K$. The probability that a neutron will be captured at the $k^{th}$ collision is the product of the neutron's surviving weight from the previous collision times the capture interacting fraction $\gamma_{cap}^{k}$. This is given explicitly as
\begin{equation}
    \label{eq:capture-total-prob}
     p = \sum_{k=1}^{K} w^{k-1}\gamma_{cap}^{k}
\end{equation}

\section{Calculating The Capture Correction Factor}
\label{sec:capture-correction-factor}

The two additional quantities required to calculate the capture correction factor are the energy averaged capture cross-section, $\langle \overline{\sigma_{\gamma}} \rangle$, and energy averaged capture probability, $\langle p \rangle$. These are obtained using the $p$ term from \autoref{eq:capture-total-prob}, and $\overline{\sigma_{\gamma}}^{0}$ from \autoref{eq:average-capture-cross-section}, as
\begin{equation}
    \label{eq:avg-capture-prob}
    \langle p \rangle = \frac{1}{N} \sum_{i=1}^{N} p_{i}
\end{equation}
and
\begin{equation}
    \label{eq:avg-capture-xs}
    \langle \overline{\sigma_{\gamma}} \rangle = \frac{1}{N} \sum_{i=1}^{N} \overline{\sigma_{\gamma}}^{0}_{i}
\end{equation}
where the sum is over $N$ total neutron histories. $p$ and $\overline{\sigma_{\gamma}}$ recieve the $i$ subscript to denote that their quantities refer to the $i^{th}$ neutron history.

These quantities are then used to calculate the capture correction factor,
\begin{equation}
    \label{eq:capture-correction-factor-calc}
    C_{C} = \frac{\langle p \rangle }{\langle \overline{ \sigma_{\gamma} } \rangle n}
\end{equation}
which can then be used in \autoref{eq:correction-thin-sample-approximation} to estimate the energy averaged capture yield $\langle Y \rangle$. 

It should be noted that the average capture cross section $\langle \overline{\sigma_{\gamma}} \rangle$ calculated from \autoref{eq:avg-capture-xs} and $\langle \sigma_{\gamma} \rangle$ are not necessarily the same quantity. The $\langle \overline{\sigma_{\gamma}} \rangle$ is computed via Monte Carlo simulation, while $\langle \sigma_{\gamma} \rangle$ from \autoref{eq:correction-thin-sample-approximation} is calculated analytically.

\section{SESH Improvements}
    \subsection{Energy Deposition Improvement}
        Originally in SESH, there was a very simple approximation for sampling what energy an interaction occurred at. It was assumed that every neutron lost the average scattering energy per collision, independent of scattering angle, i.e.,
        \begin{equation}
            E' = E \left[ 1 - \frac{2A}{\left( 1 + A \right)^2} \right]
        \end{equation}
        for all post-collision neutrons. This enabled all of the energy dependent parameters that did not get sampled using the Monte Carlo method to be pre-calculated efficiently. However, it was assumed this would be an insufficient approximation for thick samples in which many collisions were expected.
        
        This was substituted with an angle-dependent energy sampling procedure. The scattering angle of each post collision neutron would be sampled assuming an isotropic scattering distribution, and the post collision energy would be calculated as a function of the scattering angle,
        \begin{equation}
            E' = E \frac{A^2 + 2 A \mu_c + 1}{\left( A + 1 \right)^2}
        \end{equation}
in which
        \begin{equation}
            \mu_c = \cos\phi_c
        \end{equation}
        where $\phi_c$ is the exit angle of the post-collision neutron in the center-of-mass system.
        \begin{figure}[H]
            \centering
            \includegraphics[width=0.75\linewidth]{Capture Yield/Figures/multiple_scattering.png}
            \caption{Distribution of neutrons after each collision using the old SESH post-collision energy sampling method and the new angle-dependent sampling in a 12mm \textsuperscript{181}Ta sample at 3keV initial incident energy.}
            \label{fig:multiple-scattering}
        \end{figure}

    \subsection{Incident Neutron Location Error}

    SESH had an observed problem in which as the radius of the neutron beam size increased past some value relative to the total sample surface, it would fail at some point in the calculation. This ended up being attributable to an implementation error in algorithm for calculating the initial location for which the neutron is incident on the sample. Specifically, memory errors would be observed when $r_{sample}/r_{beam} \leq \sqrt{2}$.

    After some investigation, it was determined that the incident neutron location algorithm was implemented as
    \begin{align*}
        x &= R\sqrt{\zeta}, \qquad y=R\sqrt{\zeta} \\
    \end{align*}
    Consequently, rather than the intended circular beam with uniform probability the distribution was a non-uniform, square beam. This was replaced with an improved sampling algorithm which correctly sampled across the face of the cylinder. A comparison between the original and improved algorithms is shown in \autoref{fig:loc-sampling-2d-comparison}.
\begin{figure}[h]
    \centering
    \begin{subfigure}[c]{0.48\textwidth}
        \centering
        \includegraphics[width=\textwidth]{Capture Yield/Figures/wrong_sampling.pdf}
        \subcaption{Original Sampling Algorithm}
        \label{fig:loc-sampling-2d-original}
    \end{subfigure}
    \hfill
    \begin{subfigure}[c]{0.48\textwidth}
        \centering
        \includegraphics[width=\textwidth]{Capture Yield/Figures/right_sampling.pdf}
        \subcaption{Improved Sampling Algorithm}
        \label{fig:loc-sampling-2d-improved}
    \end{subfigure}
    \caption{Comparing the observed distribution of sampled neutron incident locations across the face of a cylindrical target from original SESH algorithm versus the new implementation}
    \label{fig:loc-sampling-2d-comparison}
\end{figure}
    
    Not only was this a headache to deal with, but the non-uniform distribution had the potential to under-estimate the correction factor. Neutrons sampled as interacting closer to the edges are less likely to experience significant multiple scattering events.

    \begin{figure}[h]
        \centering
        \includegraphics[width=0.75\linewidth]{Capture Yield/Figures/sampling_1d.pdf}
        \caption{Comparison between improved and original algorithm regards to sampled radius}
        \label{fig:sampling-1d-comparison}
    \end{figure}

\begin{figure}[h!]
    \centering
    \includegraphics[width=0.75\textwidth]{Capture Yield/Figures/cc_improvements.png}
    \caption{Improvements to the capture correction factor calculation in SESH compared to MCNP.}
        \label{fig:capture-improvements}
\end{figure}



\section{Validation and Characterization of Multiple Scattering and Capture Yield}
\label{sec:capture-yield-verification}

With the theoretical framework for the multiple scattering correction established, it is essential to verify that the physics implemented in the SESH code is accurate. The primary objective of this section is to validate the core physics of the multiple scattering and capture yield algorithm. This is achieved by comparing the results from SESH against a high-fidelity benchmark, ensuring that the model can be trusted before it is integrated into the full fitting procedure. To avoid confounding issues discovered in previous chapters, a clean, single-isotope benchmark case is used.

\subsection{Experimental Validation}
\label{ssec:experimental-validation}

To provide an initial validation of the capture-yield correction implementation, results are compared against the filtered-beam capture-yield measurements of McDermott \cite{McDermott2017Ta181}. That experiment used two \textsuperscript{181}Ta samples (99.95\% pure elemental tantalum) in the form of rectangular plates with lateral dimensions of approximately $10\ \mathrm{cm}\times 10\ \mathrm{cm}$ and thicknesses of $\sim 2$~mm and $\sim 6$~mm, enabling capture-yield measurements at discrete quasi-monoenergetic energies. \cite{McDermott2017Ta181}


\begin{figure}[h!]
    \centering
    \includegraphics[width=0.95\textwidth]{Capture Yield/Figures/mcdermott_yield_comparison.png}
    \caption{Comparison of simulated \textsuperscript{181}Ta capture yield to the filtered-beam measurements of McDermott \cite{McDermott2017Ta181}. Results are shown for the two plate thicknesses reported in the experiment (2~mm and 6~mm) at energies within the ENDF-8.1 \textsuperscript{181}Ta unresolved resonance region.}
    \label{fig:mcdermott-yield-compare}
\end{figure}

The resultant simulated values are in strong agreement with the measurement, and nearly identical values to MCNP. This is an unsurprising result, since this measurement was used in the ENDF-8.1 \textsuperscript{181}Ta evaluation\cite{Brown2024}. The $6$~mm case shows that both MCNP and SAMMY slightly underpredict the measured yield. A slight disagreement is acceptable, since this was one of several experiments used in the evaluation\cite{Brown2019}, so a marginal disagreement is acceptable. The critical comparison is between the MCNP and SAMMY, for which SAMMY agrees very strongly.

While the McDermott data provide a useful validation, it does not by itself isolate the performance of the multiple-scattering correction. The measurement is taken at a small number of discrete energies and in a fixed experimental geometry, so good agreement can occur even if compensating effects are present in the model. In particular, the experiment does not systematically vary the geometric drivers of the correction (beam size, sample radius, and thickness), which are exactly the quantities that control neutron path length and leakage. For that reason, a dedicated computational benchmark is used to characterize the correction factor across a controlled range of geometries and to identify where the SESH implementation begins to deviate.

\subsection{Validation Strategy}
\label{ssec:capture-yield-validation-strategy}

Validation is carried out by comparing capture yield corrections calculated by SESH against results from a high-fidelity MCNP6 benchmark \cite{mcnp}. The MCNP geometry follows the configuration described in \autoref{ssec:mcnp-benchmark-model}, consisting of a mono-directional neutron beam incident on a cylindrical \textsuperscript{181}Ta sample. Capture tallies are used to compute the history-averaged capture probability as a function of energy, from which the benchmark correction factor $C_C$ is obtained.

The benchmark spans a broad range of experimentally relevant configurations, as well as thicker stress-test cases that probe regimes with stronger attenuation and multiple scattering. The full set of parameters considered in the validation study is summarized in \autoref{tab:capture-validation-params}.

\begin{table}[h!]
\centering
\caption{Benchmark parameter ranges used for validation of the capture yield correction model.}
\label{tab:capture-validation-params}
\begin{tabular}{l l}
\hline
\textbf{Parameter} & \textbf{Values Considered} \\
\hline
Sample thickness ($n$) & 2\,mm, 4\,mm, 8\,mm, 12\,mm \\
Sample radius ($R_{\text{sample}}$) & 20\,mm, 40\,mm, 60\,mm, 80\,mm \\
Beam radius ($R_{\text{beam}}$) & 0.0, 0.25, 0.5, 0.75, 1.0 $\times R_{\text{sample}}$ \\
\hline
\end{tabular}
\end{table}

This parameter space is sufficient to verify the capture yield correction under conditions relevant to experiment, while also identifying regimes where discrepancies may emerge as multiple scattering and geometric effects become increasingly important.

\subsection{Comparison and Results}
\label{ssec:comparison-and-results}

The capture correction factor, $C_C$, calculated by the modernized SESH code was compared directly against the MCNP benchmark results for \textsuperscript{181}Ta. The comparison was performed across the full matrix of sample thicknesses, sample radii, and beam radii previously described.

\begin{figure}[h!]
    \centering

    \begin{subfigure}{\textwidth}
        \centering
        \includegraphics[width=\textwidth]{Capture Yield/Figures/cc_1x2_t1_r80.pdf}
        \caption{
        Thin, wide sample: $t=1~\mathrm{mm}$, $R=80~\mathrm{mm}$.
        }
        \label{fig:cc_best_case}
    \end{subfigure}

    \vspace{1em}

    \begin{subfigure}{\textwidth}
        \centering
        \includegraphics[width=\textwidth]{Capture Yield/Figures/cc_1x2_t8_r20.pdf}
        \caption{
        Thick, narrow sample: $t=8~\mathrm{mm}$, $R=20~\mathrm{mm}$.
        }
        \label{fig:cc_worst_case}
    \end{subfigure}

    \caption{
    Validation of the capture correction factor $C_C(E)$ under representative geometric extremes.
    In each subfigure, the left panel compares MCNP-derived correction factors (markers) with the SAMMY/SESH implementation (solid lines) for multiple beam-to-sample radius ratios $r_b/R$, while the right panel shows the corresponding relative deviation $(C_S-C_M)/C_M$.}
    \label{fig:cc-extreme-cases}
\end{figure}


\autoref{fig:cc-extreme-cases} presents two representative limits drawn from the full set of simulated geometries. The upper panel corresponds to the most favorable configuration tested, consisting of a thin sample with a large radius, while the lower panel represents a deliberately pessimistic case with large thickness and small radius. All other simulated geometries fall between these two limits in both magnitude and structure of the correction factor and relative error.

It should be stated that the most favorable case, in which the error does not exceed 0.5\%, is the most representative of typical capture yield experiments.

A consistent feature visible in the relative error panels of \autoref{fig:cc-extreme-cases} is a localized deviation near incident neutron energies of approximately $6$--$10~\mathrm{keV}$. This structure coincides with the onset of the first inelastic scattering channel in \textsuperscript{181}Ta and is observed systematically across all tested geometries.


\begin{figure}[h!]
    \centering
    \includegraphics[width=0.95\textwidth]{Capture Yield/Figures/rms_vs_radius_eta0.pdf}
    \caption{RMS relative error between SAMMY/SESH and MCNP for the capture correction factor, aggregated over incident energy and shown as a function of sample radius $R$ for several sample thicknesses $t$. Results correspond to the pencil-beam limit ($r_b/R=0$).}
    \label{fig:rms-vs-radius}
\end{figure}

\begin{figure}[h!]
    \centering
    \includegraphics[width=0.95\textwidth]{Capture Yield/Figures/rms_vs_thickness_eta0.pdf}
    \caption{RMS relative error between SAMMY/SESH and MCNP for the capture correction factor, aggregated over incident energy and shown as a function of sample thickness $t$ for several sample radii $R$. Results correspond to the pencil-beam limit ($r_b/R=0$).}
    \label{fig:rms-vs-thickness}
\end{figure}


\autoref{fig:rms-vs-radius} and \autoref{fig:rms-vs-thickness} help clarify which physical effects dominate the residual disagreement between SAMMY/SESH and MCNP. If the mismatch were primarily driven by multiple scattering within the sample volume, one would expect a strong dependence on sample radius, since increasing $R$ increases both the path length available for secondary interactions and the probability that scattered neutrons remain within the sample. Instead, the RMS error remains comparatively flat with respect to $R$ over the range tested (\autoref{fig:rms-vs-radius}), while showing a clearer increase with sample thickness (\autoref{fig:rms-vs-thickness}). This trend is more consistent with an error mechanism tied to boundary treatments rather than multiple-scattering. Increasing thickness increases the extent of the cylindrical sidewall, and the consequence of this is that any systematic differences in how the two codes treat the cylinder boundary are amplified. Consequently, the remaining discrepancies are attributed primarily to geometric edge handling at the cylinder wall, rather than deficiencies in the underlying multiple-scattering physics model as-implemented.

\begin{figure}[h!]
    \centering

    \begin{subfigure}{0.48\textwidth}
        \centering
        \includegraphics[width=\textwidth]{Capture Yield/Figures/delta_vs_eta_including_eta1_t2_r80.pdf}
        \caption{$t=2~\mathrm{mm}$, $R=80~\mathrm{mm}$}
        \label{fig:rms_eta_t2_r80}
    \end{subfigure}
    \hfill
    \begin{subfigure}{0.48\textwidth}
        \centering
        \includegraphics[width=\textwidth]{Capture Yield/Figures/delta_vs_eta_including_eta1_t6_r80.pdf}
        \caption{$t=6~\mathrm{mm}$, $R=80~\mathrm{mm}$}
        \label{fig:rms_eta_t6_r80}
    \end{subfigure}

    \vspace{0.8em}

    \begin{subfigure}{0.48\textwidth}
        \centering
        \includegraphics[width=\textwidth]{Capture Yield/Figures/delta_vs_eta_including_eta1_t8_r80.pdf}
        \caption{$t=8~\mathrm{mm}$, $R=80~\mathrm{mm}$}
        \label{fig:rms_eta_t8_r80}
    \end{subfigure}
    \hfill
    \begin{subfigure}{0.48\textwidth}
        \centering
        \includegraphics[width=\textwidth]{Capture Yield/Figures/delta_vs_eta_including_eta1_t8_r20.pdf}
        \caption{$t=8~\mathrm{mm}$, $R=20~\mathrm{mm}$}
        \label{fig:rms_eta_t8_r20}
    \end{subfigure}

    \caption{
    RMS relative change in the SAMMY capture correction factor as a function of beam-to-sample radius ratio $r_b/R$ for representative sample geometries.
    All results are shown relative to the pencil-beam limit ($r_b/R=0$).
    }
    \label{fig:rms-vs-eta}
\end{figure}


\autoref{fig:rms-vs-eta} summarizes the sensitivity of the capture correction factor to the beam-to-sample radius ratio across representative sample geometries. For samples with sufficiently large radius, the RMS change in $C_C$ remains small and relatively flat over a wide range of $r_b/R$. This behavior is consistent across thin and moderately thick samples, supporting the conclusion that edge effects are weak when the cylindrical sidewall represents a small fraction of the total interaction surface.

In contrast, the small-radius, thick-sample configuration shown in the lower-right panel exhibits a rapid growth in RMS deviation as $r_b/R$ approaches 1. This geometry is deliberately pathological. the combination of large thickness and small radius maximizes the relative contribution of the cylindrical sidewall, amplifying any algorithmic differences in how SAMMY/SESH and MCNP treat boundary crossings and near-surface transport. The emergence of large deviations only in this limit demonstrates that the underlying correction model remains robust for experimentally relevant sample dimensions, and that failure occurs only when edge-dominated geometries are pushed well beyond typical capture measurement conditions.

\subsection{Conclusion on Physics Model Validity}
\label{ssec:conclusion-physics-model}

The strong agreement between the SESH calculations and the MCNP benchmark for the \textsuperscript{181}Ta case validates the core physics of the multiple scattering and capture yield algorithm. The observed small discrepancies for very thick samples are consistent with trends seen in the transmission validation (\autoref{chap:transmission-correction}) and define the practical limits of the model's applicability. This successful verification provides the necessary confidence to proceed with integrating this capability into the full SAMMY fitting workflow.

\section{Integration with the Parameter Fitting Workflow}
\label{sec:capture-yield-fitting-workflow}

The end goal of implementing the capture yield correction is to enable direct fitting of capture-yield measurements with the correction applied dynamically during the SAMMY optimization. In this workflow, the corrected theoretical yield becomes a function of the fitted resonance parameters, and therefore its Jacobian must be well defined for use in the iterative update.

For the thin-sample model corrected by multiple scattering,
\begin{equation}
    \label{eq:yield-model-corrected}
    \left\langle Y \right\rangle (p) = n\,C_C(p)\,\left\langle \sigma_\gamma \right\rangle (p),
\end{equation}
where $p$ denotes a generic fitted parameter, and $n$ is the areal density (sample thickness) defined in \autoref{eq:correction-thin-sample-approximation}.

\subsection{Derivative of the Corrected Yield Model}
\label{ssec:capture-yield-yield-derivative}

Applying the product rule to \autoref{eq:yield-model-corrected} gives
\begin{equation}
    \label{eq:yield-model-derivative-full}
    \frac{\partial \left\langle Y \right\rangle}{\partial p}
    =
    n\left[
        C_C(p)\,\frac{\partial \left\langle \sigma_\gamma \right\rangle}{\partial p}
        +
        \left\langle \sigma_\gamma \right\rangle(p)\,\frac{\partial C_C}{\partial p}
    \right].
\end{equation}

The first term is the familiar SAMMY Jacobian contribution: $\partial\!\left\langle \sigma_\gamma \right\rangle/\partial p$ is computed analytically by the R-matrix model within SAMMY. The second term, $\partial C_C/\partial p$, is more challenging because $C_C$ is produced by a Monte Carlo transport calculation rather than a closed-form expression.

In the transmission correction workflow (\autoref{chap:transmission-correction}), this same issue appears through the derivative $\partial C_T/\partial p$, and the impact of explicitly evaluating that derivative was found to be small for the tested cases (see the fitting comparison in \autoref{fig:fitting-comparison} and the associated error table \autoref{tab:fitting-error}). Motivated by that result, the capture yield fitting presented here adopts the analogous approximation
\begin{equation}
    \label{eq:cc-derivative-assumption}
    \frac{\partial C_C}{\partial p} \approx 0,
\end{equation}
so that the corrected-yield derivative reduces to
\begin{equation}
    \label{eq:yield-model-derivative-approx}
    \frac{\partial \left\langle Y \right\rangle}{\partial p}
    \approx
    n\,C_C\,\frac{\partial \left\langle \sigma_\gamma \right\rangle}{\partial p}.
\end{equation}

This approximation is equivalent to treating the capture correction as a slowly varying multiplicative factor during parameter updates, while still fully recomputing $C_C$ at each iteration using the current parameter values. The validity of this approximation is assessed by a dedicated fitting verification test in \autoref{ssec:capture-yield-fitting-validation}.

During initial capture-yield fitting tests, the optimization failed to respond as expected when varying the average radiative width associated with higher-$\ell$ capture channels. This was traced to a legacy constraint in the original SAMMY URR implementation: although the input format permits $\ell$-dependent average radiative widths, only the lowest-$\ell$ value is consumed by the URR reader and it is reused when a higher-$\ell$ radiative width is required\cite{sammy}.

In practice, this means that the effective model used in the URR fitting workflow enforces
\begin{equation}
    \label{eq:urr-gamma-width-coupling}
    \left\langle \Gamma_\gamma \right\rangle_{\ell=0} \equiv \left\langle \Gamma_\gamma \right\rangle_{\ell=2},
\end{equation}
and similarly for other channels separated by $\Delta \ell = 2$.

To ensure that the Jacobian is consistent with the implemented model, any update nominally applied to $\left\langle \Gamma_\gamma \right\rangle_{\ell=2}$ must be routed through the active parameter $\left\langle \Gamma_\gamma \right\rangle_{\ell=0}$. Equivalently, for $p = \left\langle \Gamma_\gamma \right\rangle_{\ell=2}$ the chain rule reduces to
\begin{equation}
    \label{eq:urr-gamma-width-chain-rule}
    \frac{\partial \left\langle Y \right\rangle}{\partial p}
    =
    \frac{\partial \left\langle Y \right\rangle}{\partial \left\langle \Gamma_\gamma \right\rangle_{\ell=0}}
    \frac{\partial \left\langle \Gamma_\gamma \right\rangle_{\ell=0}}{\partial \left\langle \Gamma_\gamma \right\rangle_{\ell=2}}
    \approx
    \frac{\partial \left\langle Y \right\rangle}{\partial \left\langle \Gamma_\gamma \right\rangle_{\ell=0}},
\end{equation}
since the coupling in \autoref{eq:urr-gamma-width-coupling} implies $\partial \langle \Gamma_\gamma \rangle_{\ell=0} / \partial \langle \Gamma_\gamma \rangle_{\ell=2} = 1$.


This identification is widely treated as a valid practical constraint and is consistent with previous evaluations of average radiative widths\cite{CarlsonEscherHussein2014,atlas,Mughabghab2011,Brown2024}.

The validity of this approximation is assessed by a dedicated fitting verification test in \autoref{ssec:capture-yield-fitting-validation}.


\subsection{Verification of the Integrated Fitting Procedure}
\label{ssec:capture-yield-fitting-validation}

After verifying the standalone physics of the capture yield correction against MCNP in \autoref{sec:capture-yield-verification}, the final step is to verify that the correction can be integrated into the SAMMY fitting workflow without biasing the recovered parameters. The goal of this verification is not to reproduce an evaluated library, but to demonstrate that the combined SAMMY+SESH model can recover known parameters from controlled synthetic data.

The verification test is structured analogously to the transmission-fitting verification in \autoref{chap:transmission-correction}: generate synthetic capture-yield data, choose an incorrect starting point, and then fit with the fully integrated self-shielded capture yield model.

The benchmark comparisons in \autoref{sec:capture-yield-verification} identified several sensitivities in which the SAMMY capture-yield simulation diverged from the corresponding MCNP model. To avoid contaminating the fitting validation with these known modeling differences, the fitting study was restricted to conditions for which SAMMY and MCNP were already shown to be in close agreement: a pencil-beam geometry, incident energies below 6.1~keV (due to the inelastic state at 6.237~keV), and the largest sample radius considered (80~mm). While this avoids previously observed modeling discrepancies between SAMMY and MCNP, it also reduces the discernability between the two parameters.

Two sample thicknesses (2~mm and 6~mm) are fitted simultaneously using a shared set of URR parameters. Only the average radiative widths are adjusted in this study. Neutron strength functions and distant-level parameters are held fixed because their implementations were validated previously.


\begin{table}[H]
    \centering
    \caption{Validation of the capture-yield fitting workflow for $^{181}$Ta average radiative widths.}
    \begin{tabular}{|c|ccc|}
        \hline
        Parameter & ENDF/B-VIII.1 & Prior (Adjusted) & Final Fit \\
        \hline
        $\langle \Gamma_\gamma \rangle_{\ell=0}$ (eV) & $6.1500\times10^{-2}\ \pm\ 2.5\times10^{-4}$ & $8.0000\times10^{-2}\ $ & $6.48\times10^{-2}$ \\
        $\langle \Gamma_\gamma \rangle_{\ell=1}$ (eV) & $4.4300\times10^{-2}\ \pm\ 2.3\times10^{-4}$ & $6.0000\times10^{-2} $ & $4.40\times10^{-2}$ \\
        $\langle \Gamma_\gamma \rangle_{\ell=2}$ (eV) & $6.1500\times10^{-2}\ \pm\ 2.5\times10^{-4}$ & \multicolumn{2}{c|}{\makebox[0pt][c]{(coupled to $\ell=0$)}} \\
        \hline
        $\chi^2$ & -- & $2.592\times10^{3}$ & $4.729$ \\
        \hline
    \end{tabular}
    \label{tab:capture-yield-fitting-results}
\end{table}

The simultaneous fit successfully reduced the objective function from $\chi^2 = 2.592\times10^{3}$ to $\chi^2 = 4.729$, demonstrating that the capture-yield correction can be exercised within the SAMMY fitting loop under the restricted validation conditions.

The fitted average radiative widths are close to the ENDF/B-VIII.1 values despite starting from deliberately perturbed inputs. In particular, the fitted $\langle \Gamma_\gamma \rangle_{\ell=1}$ moves from $6.00\times10^{-2}$ to $4.40\times10^{-2}$, which is very near the evaluated value of $4.43\times10^{-2}$. The fitted $\langle \Gamma_\gamma \rangle_{\ell=0}$ similarly shifts toward the evaluation, from $8.00\times10^{-2}$ to $6.48\times10^{-2}$, compared to the evaluated value of $6.15\times10^{-2}$. 

The residual difference between the fitted and evaluated $\ell=0$ width is not unexpected in this validation, as previously mentioned due to the highly restricted fit regime. The fit was intentionally restricted to $E<6.1$~keV with a pencil beam and the largest sample radius in order to avoid known model discrepancies identified in the characterization. While this restriction isolates the fitting workflow from those model differences, it also limits the amount of information available to distinguish the effects of the $\ell=0$ and $\ell=1$ widths using capture-yield data alone. Within that limited domain, the fit results indicate that the workflow is functioning as intended and produces parameter values that are broadly consistent with the evaluation, even when started from significantly perturbed inputs.


\subsection{Conclusion}
\label{ssec:capture-yield-conclusion}

This chapter presented a modernized capture-yield correction capability and validated it against both experiment and high-fidelity Monte Carlo. Comparisons to the $^{181}$Ta capture-yield measurements of McDermott show strong agreement in the unresolved resonance region, with results that closely track MCNP. A broader MCNP benchmark study was then used to characterize the correction factor across thickness and geometric variations, confirming that the implementation reproduces the expected behavior and remains accurate over the range of configurations representative of capture-yield experiments.

Small, localized discrepancies were observed near the onset of the first inelastic channel in $^{181}$Ta and in edge-dominated geometries. These effects define practical limits of applicability rather than indicating a failure of the underlying correction model, and they motivate the restricted validation domain used for the fitting tests. Within that domain, the capture-yield correction was successfully integrated into the SAMMY fitting workflow, including simultaneous fitting of multiple sample thicknesses and the legacy coupled-$\ell$ radiative-width treatment. Overall, the results demonstrate that the method is suitable for use in evaluations which utilize capture-yield measurements of self-shielded, thick samples with the correction applied dynamically during optimization.
