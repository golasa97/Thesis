
\section{Introduction}
\label{sec:capture-yield-introduction}

The self-shielding correction framework developed and validated for transmission measurements in \autoref{chap:transmission-correction} and \autoref{chap:multiiso-transmission-correction} is now extended to another critical experimental observable: capture yield. In the idealized thin-sample limit (vanishing areal density), one often uses the heuristic identification $Y_\gamma(E)\approx n\,\sigma_\gamma(E)$, where $n$ is the areal density (atoms/barn). However, the directly observable capture yield divided by sample thickness cannot, in general, be equated to the capture cross section except in the limit of vanishing sample thickness; for practical thicknesses the yield is modified by self-shielding and by multiple scattering within the sample \cite{u238-differential-urr}.

Accurately modeling capture yield in the unresolved resonance region (URR) therefore requires accounting for two distinct physical phenomena. The first is resonance self-shielding, driven by statistical fluctuations in the microscopic cross section, which has been the primary focus of the preceding chapters. The second is the contribution of neutron \emph{multiple scattering within the measurement sample}, which can partially cancel self-shielding in some energy regions and dominate in others \cite{u238-differential-urr,Macklin1991_U238CaptureYield_AnnNuclEnergy}. Multiple scattering represents the contribution of neutrons that scatter 1 or more times \emph{within the sample} before being captured or escaping; each additional scattering requires the neutron to survive another collision without being absorbed, so stronger absorption reduces the probability of long multi-scatter paths \cite{Vineyard1954_MultipleScattering_PhysRev}. While other multiple-scattering-related effects can arise in complete capture-yield analyses, this work focuses specifically on the correction associated with neutron histories that undergo one or more scattering events \emph{inside the sample} prior to capture or escape \cite{Block2021_Cs133_TransmissionCapture_NSE}.

This chapter will not repeat the foundational discussion of Monte Carlo cross-section generation or the statistical basis of resonance self-shielding, as these topics have been thoroughly described in \autoref{chap:implementation}, \autoref{chap:transmission-correction}, and \autoref{chap:multiiso-transmission-correction}. Instead, it documents the modernization of the existing capture-yield correction implementation, its verification against both experimental data and a comprehensive MCNP benchmark, and the integration of the corrected model into the SAMMY fitting workflow for use in parameter optimization.


\section{Multiple Scattering Correction}
\label{sec:multiple-scattering}

Multiple scattering is an effect in which a neutron is scattered one or more times before being captured in a sample. As the thickness of the sample increases, the probability of a scattered neutron experiencing a subsequent reaction (capture or additional scattering) also increases. Neutrons which undergo scattering events can therefore be miscounted as captures, artificially increasing the measured yield.
\begin{figure}[H]
    \centering
    \includegraphics[width=0.75\linewidth]{Capture Yield/Figures/multiple_scattering.pdf}
    \caption{Illustration of a neutron undergoing multiple scattering events, and getting miscounted as a capture event.}
    \label{fig:multiple-scattering-example}
\end{figure}
This is illustrated in \autoref{fig:multiple-scattering-example}, in which a neutron scatters twice before being captured.

To quantify this effect, the measured yield is expressed relative to the idealized thin-sample result. The thin-sample approximation gives the capture yield as
\begin{equation}
    \label{eq:thin-sample-approximation}
    \left\langle Y \right\rangle \approx n \left\langle \sigma_\gamma \right\rangle 
\end{equation}
where $\left\langle Y \right\rangle$ is the average yield, $n$ is the sample thickness, and $\left\langle \sigma_\gamma \right\rangle$ is the average capture cross section \cite{sesh}. The capture correction $C_C$ modifies the thin-sample approximation such that
\begin{equation}
    \label{eq:correction-thin-sample-approximation}
    \left\langle Y \right\rangle = n \left\langle \sigma_\gamma \right\rangle C_C
\end{equation}
where $C_C$ accounts for both self-shielding and multiple scattering within the sample \cite{sesh,u238-differential-urr}.


\subsection{Overview of the Monte Carlo Correction Procedure}
\label{ssec:mc-correction-overview}

The capture correction factor is obtained by a Monte Carlo simulation that tracks individual neutron histories through a cylindrical disc target. The full mathematical derivation, including all sampling formulas for position, energy, direction, and weighting, is provided in \autoref{app:multiple-scattering-derivation}. The essential features of the algorithm are summarized here.

The simulation models a cylindrical sample of radius $R$ and thickness $n$, with a mono-directional neutron beam incident on the front face. For each history, a neutron enters the sample at a uniformly sampled location on the front face at the energy of interest $E^0$. Cross sections are sampled from the average parameters using the procedure described in \autoref{sec:mc-sampling-from-average-parameters}, and the abundance-weighted average total, capture, and scattering cross sections are computed for the material.

At each step, the simulation determines the shortest distance the neutron could travel before exiting through one of the three bounding surfaces (front face, rear face, or cylindrical wall), then samples a collision distance from the total cross section, conditioned on the constraint that the interaction must occur within the sample. If a collision occurs, the target isotope is selected probabilistically according to relative scattering cross sections and abundances, and the post-collision energy is determined from two-body kinematics assuming isotropic scattering in the center-of-mass frame. The neutron direction is updated via a standard rotation matrix transformation.

Rather than terminating neutron histories by explicit capture or escape, the simulation uses an implicit capture (survival biasing) scheme \cite{mcnp}. At each collision, the neutron is forced to scatter, and its statistical weight is reduced by the ratio of the scattering-to-total interaction probability. The weight lost at each collision is tallied as the capture contribution, and the history continues until the neutron weight falls below a specified cutoff. The total capture probability for a single history is the cumulative sum of these weighted capture tallies across all collisions.

After simulating $N$ histories, the capture correction factor is assembled as
\begin{equation}
    \label{eq:capture-correction-factor-calc}
    C_{C} = \frac{\langle p \rangle }{\langle \overline{ \sigma_{\gamma} } \rangle n}
\end{equation}
where $\langle p \rangle$ is the history-averaged capture probability and $\langle \overline{\sigma_{\gamma}} \rangle$ is the history-averaged capture cross section from the Monte Carlo samples. It should be noted that $\langle \overline{\sigma_{\gamma}} \rangle$ computed by the Monte Carlo and $\langle \sigma_{\gamma} \rangle$ from \autoref{eq:correction-thin-sample-approximation} are not necessarily identical; the former is a stochastic estimate while the latter is calculated analytically.


\section{SESH Improvements}
\label{sec:sesh-improvements}

Several issues in the original SESH implementation were identified during the course of this work that affected the accuracy of the capture correction factor. The most significant are documented here, along with their impact on the computed correction.

    \subsection{Energy Deposition Improvement}
        The original SESH code used a simplified approximation for post-collision energy sampling: every neutron was assumed to lose the average energy per collision, independent of the scattering angle \cite{sesh},
        \begin{equation}
            E' = E \left[ 1 - \frac{2A}{\left( 1 + A \right)^2} \right]
        \end{equation}
        This enabled efficient pre-calculation of energy-dependent quantities that were not sampled by the Monte Carlo method. However, for thick samples where many collisions are expected, this approximation was judged to be insufficient.
        
        The approximation was replaced with an angle-dependent energy sampling procedure. The scattering angle of each post-collision neutron is sampled assuming an isotropic distribution, and the post-collision energy is calculated as
        \begin{equation}
            E' = E \frac{A^2 + 2 A \mu_c + 1}{\left( A + 1 \right)^2}
        \end{equation}
        where
        \begin{equation}
            \mu_c = \cos\phi_c
        \end{equation}
        and $\phi_c$ is the exit angle of the post-collision neutron in the center-of-mass system.
        \begin{figure}[H]
            \centering
            \includegraphics[width=0.75\linewidth]{Capture Yield/Figures/multiple_scattering.png}
            \caption{Distribution of neutrons after each collision using the original SESH post-collision energy sampling method and the new angle-dependent sampling in a 12~mm \textsuperscript{181}Ta sample at 3~keV initial incident energy.}
            \label{fig:multiple-scattering}
        \end{figure}

    \subsection{Incident Neutron Location Error}

    An error was identified in which SESH would fail when the beam radius exceeded a certain fraction of the sample radius. This was traced to an implementation error in the algorithm for calculating the initial neutron incident location. Specifically, memory errors were observed when $r_{sample}/r_{beam} \leq \sqrt{2}$.

    Investigation revealed that the incident neutron location algorithm had been implemented as
    \begin{align*}
        x &= R\sqrt{\zeta}, \qquad y=R\sqrt{\zeta} \\
    \end{align*}
    Rather than producing the intended circular beam with uniform probability, this produced a non-uniform, square beam distribution. The algorithm was replaced with a corrected sampling procedure that uniformly samples across the face of the cylinder. A comparison between the original and improved algorithms is shown in \autoref{fig:loc-sampling-2d-comparison}.
\begin{figure}[h]
    \centering
    \begin{subfigure}[c]{0.48\textwidth}
        \centering
        \includegraphics[width=\textwidth]{Capture Yield/Figures/wrong_sampling.pdf}
        \subcaption{Original Sampling Algorithm}
        \label{fig:loc-sampling-2d-original}
    \end{subfigure}
    \hfill
    \begin{subfigure}[c]{0.48\textwidth}
        \centering
        \includegraphics[width=\textwidth]{Capture Yield/Figures/right_sampling.pdf}
        \subcaption{Improved Sampling Algorithm}
        \label{fig:loc-sampling-2d-improved}
    \end{subfigure}
    \caption{Comparing the observed distribution of sampled neutron incident locations across the face of a cylindrical target from original SESH algorithm versus the new implementation.}
    \label{fig:loc-sampling-2d-comparison}
\end{figure}
    
    Beyond causing runtime failures, the non-uniform distribution had the potential to underestimate the correction factor, since neutrons sampled as interacting closer to the edges are less likely to experience significant multiple scattering events.

    \begin{figure}[h]
        \centering
        \includegraphics[width=0.75\linewidth]{Capture Yield/Figures/sampling_1d.pdf}
        \caption{Comparison between improved and original algorithm with respect to sampled radius.}
        \label{fig:sampling-1d-comparison}
    \end{figure}

\begin{figure}[h!]
    \centering
    \includegraphics[width=0.75\textwidth]{Capture Yield/Figures/cc_improvements.png}
    \caption{Progressive improvement of the SESH capture correction factor for a 12~mm \textsuperscript{181}Ta cylindrical sample. Each curve adds a successive code improvement to the original SESH implementation, compared against the MCNP benchmark. The original SESH result diverges from MCNP above approximately 10~keV, with each correction reducing the discrepancy.}
        \label{fig:capture-improvements}
\end{figure}



\section{Validation and Characterization of Multiple Scattering and Capture Yield}
\label{sec:capture-yield-verification}

With the theoretical framework for the multiple scattering correction established, it is essential to verify that the physics implemented in the SESH code is accurate. The primary objective of this section is to validate the core physics of the multiple scattering and capture yield algorithm. This is achieved by comparing the results from SESH against a high-fidelity benchmark, ensuring that the model can be trusted before it is integrated into the full fitting procedure. To avoid confounding issues discovered in \autoref{chap:multiiso-transmission-correction}, a clean, single-isotope benchmark case is used.

\subsection{Experimental Validation}
\label{ssec:experimental-validation}

To provide an initial validation of the capture-yield correction implementation, results are compared against the filtered-beam capture-yield measurements of McDermott \cite{McDermott2017Ta181}. That experiment used two \textsuperscript{181}Ta samples (99.95\% pure elemental tantalum) in the form of rectangular plates with lateral dimensions of approximately $10\ \mathrm{cm}\times 10\ \mathrm{cm}$ and thicknesses of $\sim 2$~mm and $\sim 6$~mm, enabling capture-yield measurements at discrete quasi-monoenergetic energies.


\begin{figure}[h!]
    \centering
    \includegraphics[width=0.95\textwidth]{Capture Yield/Figures/mcdermott_yield_comparison.png}
    \caption{Comparison of simulated \textsuperscript{181}Ta capture yield to the filtered-beam measurements of McDermott \cite{McDermott2017Ta181}. Results are shown for the two plate thicknesses reported in the experiment (2~mm and 6~mm) at energies within the ENDF-8.1 \textsuperscript{181}Ta unresolved resonance region.}
    \label{fig:mcdermott-yield-compare}
\end{figure}

The simulated values are in strong agreement with the measurement, and nearly identical to MCNP. This is not unexpected, since this measurement was used in the ENDF-8.1 \textsuperscript{181}Ta evaluation\cite{Brown2024}. The 6~mm case shows that both MCNP and SAMMY slightly underpredict the measured yield, though a slight disagreement is acceptable since this was one of several experiments used in the evaluation\cite{Brown2019}. The critical comparison is between MCNP and SAMMY, for which SAMMY agrees very well.

While the McDermott data provide a useful validation, they do not by themselves isolate the performance of the multiple-scattering correction. The measurement is taken at a small number of discrete energies and in a fixed experimental geometry, so good agreement can occur even if compensating effects are present in the model. In particular, the experiment does not systematically vary the geometric drivers of the correction (beam size, sample radius, and thickness), which are exactly the quantities that control neutron path length and leakage. For that reason, a dedicated computational benchmark is used to characterize the correction factor across a controlled range of geometries and to identify where the SESH implementation begins to deviate.

\subsection{Validation Strategy}
\label{ssec:capture-yield-validation-strategy}

Validation is carried out by comparing capture yield corrections calculated by SESH against results from a high-fidelity MCNP6 benchmark \cite{mcnp}. The MCNP geometry follows the configuration described in \autoref{ssec:mcnp-benchmark-model}, consisting of a mono-directional neutron beam incident on a cylindrical \textsuperscript{181}Ta sample. Capture tallies are used to compute the history-averaged capture probability as a function of energy, from which the benchmark correction factor $C_C$ is obtained.

The benchmark spans a broad range of experimentally relevant configurations, as well as thicker stress-test cases that probe regimes with stronger attenuation and multiple scattering. The full set of parameters considered in the validation study is summarized in \autoref{tab:capture-validation-params}.

\begin{table}[h!]
\centering
\caption{Benchmark parameter ranges used for validation of the capture yield correction model.}
\label{tab:capture-validation-params}
\begin{tabular}{l l}
\hline
\textbf{Parameter} & \textbf{Values Considered} \\
\hline
Sample thickness ($n$) & 2\,mm, 4\,mm, 8\,mm, 12\,mm \\
Sample radius ($R_{\text{sample}}$) & 20\,mm, 40\,mm, 60\,mm, 80\,mm \\
Beam radius ($R_{\text{beam}}$) & 0.0, 0.25, 0.5, 0.75, 1.0 $\times R_{\text{sample}}$ \\
\hline
\end{tabular}
\end{table}

This parameter space is sufficient to verify the capture yield correction under conditions relevant to experiment, while also identifying regimes where discrepancies may emerge as multiple scattering and geometric effects become increasingly important.

\subsection{Comparison and Results}
\label{ssec:comparison-and-results}

The capture correction factor, $C_C$, calculated by the modernized SESH code was compared directly against the MCNP benchmark results for \textsuperscript{181}Ta. The comparison was performed across the full matrix of sample thicknesses, sample radii, and beam radii previously described.

\begin{figure}[h!]
    \centering

    \begin{subfigure}{\textwidth}
        \centering
        \includegraphics[width=\textwidth]{Capture Yield/Figures/cc_1x2_t1_r80.pdf}
        \caption{
        Thin, wide sample: $t=1~\mathrm{mm}$, $R=80~\mathrm{mm}$.
        }
        \label{fig:cc_best_case}
    \end{subfigure}

    \vspace{1em}

    \begin{subfigure}{\textwidth}
        \centering
        \includegraphics[width=\textwidth]{Capture Yield/Figures/cc_1x2_t8_r20.pdf}
        \caption{
        Thick, narrow sample: $t=8~\mathrm{mm}$, $R=20~\mathrm{mm}$.
        }
        \label{fig:cc_worst_case}
    \end{subfigure}

    \caption{
    Validation of the capture correction factor $C_C(E)$ under representative geometric extremes.
    In each subfigure, the left panel compares MCNP-derived correction factors (markers) with the SAMMY/SESH implementation (solid lines) for multiple beam-to-sample radius ratios $r_b/R$, while the right panel shows the corresponding relative deviation $(C_S-C_M)/C_M$.}
    \label{fig:cc-extreme-cases}
\end{figure}


\autoref{fig:cc-extreme-cases} presents two representative limits drawn from the full set of simulated geometries. The upper panel corresponds to the most favorable configuration tested, consisting of a thin sample with a large radius, while the lower panel represents a deliberately pessimistic case with large thickness and small radius. All other simulated geometries fall between these two limits in both magnitude and structure of the correction factor and relative error.

It should be stated that the most favorable case, in which the error does not exceed 0.5\%, is the most representative of typical capture yield experiments.

A consistent feature visible in the relative error panels of \autoref{fig:cc-extreme-cases} is a localized deviation near incident neutron energies of approximately $6$--$10~\mathrm{keV}$. This structure coincides with the onset of the 6.237~keV inelastic scattering state in \textsuperscript{181}Ta and is observed systematically across all tested geometries.


\begin{figure}[h!]
    \centering
    \includegraphics[width=0.95\textwidth]{Capture Yield/Figures/rms_vs_radius_eta0.pdf}
    \caption{RMS relative error between SAMMY/SESH and MCNP for the capture correction factor, aggregated over incident energy and shown as a function of sample radius $R$ for several sample thicknesses $t$. Results correspond to the pencil-beam limit ($r_b/R=0$).}
    \label{fig:rms-vs-radius}
\end{figure}

\begin{figure}[h!]
    \centering
    \includegraphics[width=0.95\textwidth]{Capture Yield/Figures/rms_vs_thickness_eta0.pdf}
    \caption{RMS relative error between SAMMY/SESH and MCNP for the capture correction factor, aggregated over incident energy and shown as a function of sample thickness $t$ for several sample radii $R$. Results correspond to the pencil-beam limit ($r_b/R=0$).}
    \label{fig:rms-vs-thickness}
\end{figure}


\autoref{fig:rms-vs-radius} and \autoref{fig:rms-vs-thickness} help clarify which physical effects dominate the residual disagreement between SAMMY/SESH and MCNP. If the mismatch were primarily driven by multiple scattering within the sample volume, one would expect a strong dependence on sample radius, since increasing $R$ increases both the path length available for secondary interactions and the probability that scattered neutrons remain within the sample. Instead, the RMS error remains comparatively flat with respect to $R$ over the range tested (\autoref{fig:rms-vs-radius}), while showing a clearer increase with sample thickness (\autoref{fig:rms-vs-thickness}). This trend is more consistent with an error mechanism tied to boundary treatments rather than multiple scattering. Increasing thickness increases the extent of the cylindrical sidewall, and the consequence of this is that any systematic differences in how the two codes treat the cylinder boundary are amplified. Consequently, the remaining discrepancies are attributed primarily to geometric edge handling at the cylinder wall, rather than deficiencies in the underlying multiple-scattering physics model as implemented.

\begin{figure}[h!]
    \centering

    \begin{subfigure}{0.48\textwidth}
        \centering
        \includegraphics[width=\textwidth]{Capture Yield/Figures/delta_vs_eta_including_eta1_t2_r80.pdf}
        \caption{$t=2~\mathrm{mm}$, $R=80~\mathrm{mm}$}
        \label{fig:rms_eta_t2_r80}
    \end{subfigure}
    \hfill
    \begin{subfigure}{0.48\textwidth}
        \centering
        \includegraphics[width=\textwidth]{Capture Yield/Figures/delta_vs_eta_including_eta1_t6_r80.pdf}
        \caption{$t=6~\mathrm{mm}$, $R=80~\mathrm{mm}$}
        \label{fig:rms_eta_t6_r80}
    \end{subfigure}

    \vspace{0.8em}

    \begin{subfigure}{0.48\textwidth}
        \centering
        \includegraphics[width=\textwidth]{Capture Yield/Figures/delta_vs_eta_including_eta1_t8_r80.pdf}
        \caption{$t=8~\mathrm{mm}$, $R=80~\mathrm{mm}$}
        \label{fig:rms_eta_t8_r80}
    \end{subfigure}
    \hfill
    \begin{subfigure}{0.48\textwidth}
        \centering
        \includegraphics[width=\textwidth]{Capture Yield/Figures/delta_vs_eta_including_eta1_t8_r20.pdf}
        \caption{$t=8~\mathrm{mm}$, $R=20~\mathrm{mm}$}
        \label{fig:rms_eta_t8_r20}
    \end{subfigure}

    \caption{
    RMS relative change in the SAMMY capture correction factor as a function of beam-to-sample radius ratio $r_b/R$ for representative sample geometries.
    All results are shown relative to the pencil-beam limit ($r_b/R=0$).
    }
    \label{fig:rms-vs-eta}
\end{figure}


\autoref{fig:rms-vs-eta} summarizes the sensitivity of the capture correction factor to the beam-to-sample radius ratio across representative sample geometries. For samples with sufficiently large radius, the RMS change in $C_C$ remains small and relatively flat over a wide range of $r_b/R$. This behavior is consistent across thin and moderately thick samples, supporting the conclusion that edge effects are weak when the cylindrical sidewall represents a small fraction of the total interaction surface.

In contrast, the small-radius, thick-sample configuration shown in the lower-right panel exhibits a rapid growth in RMS deviation as $r_b/R$ approaches 1. This geometry is deliberately pathological: the combination of large thickness and small radius maximizes the relative contribution of the cylindrical sidewall, amplifying any algorithmic differences in how SAMMY/SESH and MCNP treat boundary crossings and near-surface transport. The emergence of large deviations only in this limit demonstrates that the underlying correction model remains robust for experimentally relevant sample dimensions, and that failure occurs only when edge-dominated geometries are pushed well beyond typical capture measurement conditions.

\subsection{Conclusion on Physics Model Validity}
\label{ssec:conclusion-physics-model}

The strong agreement between the SESH calculations and the MCNP benchmark for the \textsuperscript{181}Ta case validates the core physics of the multiple scattering and capture yield algorithm. The observed small discrepancies for very thick samples are consistent with trends seen in the transmission validation (\autoref{chap:transmission-correction}) and define the practical limits of the model's applicability. This successful verification provides the necessary confidence to proceed with integrating this capability into the full SAMMY fitting workflow.

\section{Integration with the Parameter Fitting Workflow}
\label{sec:capture-yield-fitting-workflow}

The end goal of implementing the capture yield correction is to enable direct fitting of capture-yield measurements with the correction applied dynamically during the SAMMY optimization. In this workflow, the corrected theoretical yield becomes a function of the fitted resonance parameters, and therefore its Jacobian must be well defined for use in the iterative update.

For the thin-sample model corrected by multiple scattering,
\begin{equation}
    \label{eq:yield-model-corrected}
    \left\langle Y \right\rangle (p) = n\,C_C(p)\,\left\langle \sigma_\gamma \right\rangle (p),
\end{equation}
where $p$ denotes a generic fitted parameter, and $n$ is the areal density (sample thickness) defined in \autoref{eq:correction-thin-sample-approximation}.

\subsection{Derivative of the Corrected Yield Model}
\label{ssec:capture-yield-yield-derivative}

Applying the product rule to \autoref{eq:yield-model-corrected} gives
\begin{equation}
    \label{eq:yield-model-derivative-full}
    \frac{\partial \left\langle Y \right\rangle}{\partial p}
    =
    n\left[
        C_C(p)\,\frac{\partial \left\langle \sigma_\gamma \right\rangle}{\partial p}
        +
        \left\langle \sigma_\gamma \right\rangle(p)\,\frac{\partial C_C}{\partial p}
    \right].
\end{equation}

The first term is the familiar SAMMY Jacobian contribution: $\partial\!\left\langle \sigma_\gamma \right\rangle/\partial p$ is computed analytically by the R-matrix model within SAMMY. The second term, $\partial C_C/\partial p$, is more challenging because $C_C$ is produced by a Monte Carlo transport calculation rather than a closed-form expression.

In the transmission correction workflow (\autoref{chap:transmission-correction}), this same issue appears through the derivative $\partial C_T/\partial p$, and the impact of explicitly evaluating that derivative was found to be small for the tested cases (see the fitting comparison in \autoref{fig:fitting-comparison} and the associated error table \autoref{tab:fitting-error}). Motivated by that result, the capture yield fitting presented here adopts the analogous approximation
\begin{equation}
    \label{eq:cc-derivative-assumption}
    \frac{\partial C_C}{\partial p} \approx 0,
\end{equation}
so that the corrected-yield derivative reduces to
\begin{equation}
    \label{eq:yield-model-derivative-approx}
    \frac{\partial \left\langle Y \right\rangle}{\partial p}
    \approx
    n\,C_C\,\frac{\partial \left\langle \sigma_\gamma \right\rangle}{\partial p}.
\end{equation}

This approximation is equivalent to treating the capture correction as a slowly varying multiplicative factor during parameter updates, while still fully recomputing $C_C$ at each iteration using the current parameter values. The validity of this approximation is assessed by the fitting verification test in \autoref{ssec:capture-yield-fitting-validation}.

\subsection{Coupling of Radiative Widths}
\label{ssec:legacy-gamma-coupling}

During initial capture-yield fitting tests, the optimization failed to respond as expected when varying the average radiative width associated with higher-$\ell$ capture channels. This was traced to a legacy constraint in the original SAMMY URR implementation: although the input format permits $\ell$-dependent average radiative widths, only the lowest-$\ell$ value is consumed by the URR reader and it is reused when a higher-$\ell$ radiative width is required\cite{sammy}.

In practice, this means that the effective model used in the URR fitting workflow enforces
\begin{equation}
    \label{eq:urr-gamma-width-coupling}
    \left\langle \Gamma_\gamma \right\rangle_{\ell=0} \equiv \left\langle \Gamma_\gamma \right\rangle_{\ell=2},
\end{equation}
and similarly for other channels separated by $\Delta \ell = 2$.

To ensure that the Jacobian is consistent with the implemented model, any update nominally applied to $\left\langle \Gamma_\gamma \right\rangle_{\ell=2}$ must be routed through the active parameter $\left\langle \Gamma_\gamma \right\rangle_{\ell=0}$. Equivalently, for $p = \left\langle \Gamma_\gamma \right\rangle_{\ell=2}$ the chain rule reduces to
\begin{equation}
    \label{eq:urr-gamma-width-chain-rule}
    \frac{\partial \left\langle Y \right\rangle}{\partial p}
    =
    \frac{\partial \left\langle Y \right\rangle}{\partial \left\langle \Gamma_\gamma \right\rangle_{\ell=0}}
    \frac{\partial \left\langle \Gamma_\gamma \right\rangle_{\ell=0}}{\partial \left\langle \Gamma_\gamma \right\rangle_{\ell=2}}
    \approx
    \frac{\partial \left\langle Y \right\rangle}{\partial \left\langle \Gamma_\gamma \right\rangle_{\ell=0}},
\end{equation}
since the coupling in \autoref{eq:urr-gamma-width-coupling} implies $\partial \langle \Gamma_\gamma \rangle_{\ell=0} / \partial \langle \Gamma_\gamma \rangle_{\ell=2} = 1$.

This constraint is derived from the assumption that radiative widths only depend on parity, and consequently, only two independent $\Gamma_{\gamma}$ values can exist \cite{SIRAKOV20081223}. This assumption is widely treated as a valid practical constraint and is consistent with previous evaluations of average radiative widths\cite{CarlsonEscherHussein2014,atlas,Mughabghab2011,Brown2024}.

\subsection{Verification of the Integrated Fitting Procedure}
\label{ssec:capture-yield-fitting-validation}

After verifying the standalone physics of the capture yield correction against MCNP in \autoref{sec:capture-yield-verification}, the final step is to verify that the correction can be integrated into the SAMMY fitting workflow without biasing the recovered parameters. The goal of this verification is not to reproduce an evaluated library, but to demonstrate that the combined SAMMY+SESH model can recover known parameters from controlled synthetic data.

The verification test is structured analogously to the transmission-fitting verification in \autoref{chap:transmission-correction}: generate synthetic capture-yield data, choose an incorrect starting point, and then fit with the fully integrated self-shielded capture yield model.

The benchmark comparisons in \autoref{sec:capture-yield-verification} identified several sensitivities in which the SAMMY capture-yield simulation diverged from the corresponding MCNP model. To avoid contaminating the fitting validation with these known modeling differences, the fitting study was restricted to conditions for which SAMMY and MCNP were already shown to be in close agreement: a pencil-beam geometry, incident energies below 6.1~keV (due to the inelastic state at 6.237~keV), and the largest sample radius considered (80~mm). While this avoids previously observed modeling discrepancies between SAMMY and MCNP, it also reduces the discernability between the two parameters.

Two sample thicknesses (2~mm and 6~mm) are fitted simultaneously using a shared set of URR parameters. Only the average radiative widths are adjusted in this study. Neutron strength functions and distant-level parameters are held fixed because their implementations were validated previously.


\begin{table}[H]
    \centering
    \caption{Validation of the capture-yield fitting workflow for $^{181}$Ta average radiative widths.}
    \begin{tabular}{|c|ccc|}
        \hline
        Parameter & ENDF/B-VIII.1 & Prior (Adjusted) & Final Fit \\
        \hline
        $\langle \Gamma_\gamma \rangle_{\ell=0}$ (eV) & $6.1500\times10^{-2}\ \pm\ 2.5\times10^{-4}$ & $8.0000\times10^{-2}\ $ & $6.48\times10^{-2}$ \\
        $\langle \Gamma_\gamma \rangle_{\ell=1}$ (eV) & $4.4300\times10^{-2}\ \pm\ 2.3\times10^{-4}$ & $6.0000\times10^{-2} $ & $4.40\times10^{-2}$ \\
        $\langle \Gamma_\gamma \rangle_{\ell=2}$ (eV) & $6.1500\times10^{-2}\ \pm\ 2.5\times10^{-4}$ & \multicolumn{2}{c|}{\makebox[0pt][c]{(coupled to $\ell=0$)}} \\
        \hline
        $\chi^2$ & -- & $2.592\times10^{3}$ & $4.729$ \\
        \hline
    \end{tabular}
    \label{tab:capture-yield-fitting-results}
\end{table}

The simultaneous fit successfully reduced the objective function from $\chi^2 = 2.592\times10^{3}$ to $\chi^2 = 4.729$, demonstrating that the capture-yield correction can be exercised within the SAMMY fitting loop under the restricted validation conditions.

The fitted average radiative widths are close to the ENDF/B-VIII.1 values despite starting from deliberately perturbed inputs. In particular, the fitted $\langle \Gamma_\gamma \rangle_{\ell=1}$ moves from $6.00\times10^{-2}$ to $4.40\times10^{-2}$, which is very near the evaluated value of $4.43\times10^{-2}$. The fitted $\langle \Gamma_\gamma \rangle_{\ell=0}$ similarly shifts toward the evaluation, from $8.00\times10^{-2}$ to $6.48\times10^{-2}$, compared to the evaluated value of $6.15\times10^{-2}$. 

The residual difference between the fitted and evaluated $\ell=0$ width is not unexpected in this validation, as previously mentioned due to the highly restricted fit regime. The fit was intentionally restricted to $E<6.1$~keV with a pencil beam and the largest sample radius in order to avoid known model discrepancies identified in the characterization. While this restriction isolates the fitting workflow from those model differences, it also limits the amount of information available to distinguish the effects of the $\ell=0$ and $\ell=1$ widths using capture-yield data alone. Within that limited domain, the fit results indicate that the workflow is functioning as intended and produces parameter values that are broadly consistent with the evaluation, even when started from significantly perturbed inputs.


\section{Conclusion}
\label{sec:capture-yield-conclusion}

This chapter presented a modernized capture-yield correction capability and validated it against both experiment and high-fidelity Monte Carlo. Comparisons to the $^{181}$Ta capture-yield measurements of McDermott show strong agreement in the unresolved resonance region, with results that closely track MCNP. A broader MCNP benchmark study was then used to characterize the correction factor across thickness and geometric variations, confirming that the implementation reproduces the expected behavior and remains accurate over the range of configurations representative of capture-yield experiments.

Small, localized discrepancies were observed near the onset of the first inelastic 6.237~keV channel in $^{181}$Ta\cite{ta-181-inelastic} and in edge-dominated geometries. These effects define practical limits of applicability rather than indicating a failure of the underlying correction model, and they motivate the restricted validation domain used for the fitting tests. Within that domain, the capture-yield correction was successfully integrated into the SAMMY fitting workflow, including simultaneous fitting of multiple sample thicknesses and the legacy coupled-$\ell$ radiative-width treatment. Overall, the results demonstrate that the method is suitable for use in evaluations which utilize capture-yield measurements of self-shielded, thick samples with the correction applied dynamically during optimization.
