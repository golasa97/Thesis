\section{Introduction}
\label{sec:capture-yield-introduction}

The self-shielding correction framework developed and validated for transmission measurements in \autoref{chap:transmission-correction} and \autoref{chap:multiiso-transmission-correction} is now extended to another critical experimental observable: capture yield. Accurately modeling capture yield in the unresolved resonance region (URR) requires accounting for two distinct physical phenomena. The first is the resonance self-shielding effect, driven by the statistical fluctuations in the microscopic cross section, which has been the primary focus of the preceding chapters. The second, which is particularly pronounced in capture experiments, is the effect of neutron multiple scattering within the measurement sample.

This chapter will not repeat the foundational discussion on the underlying physics of Monte Carlo cross-section generation and the statistical nature of resonance self-shielding, as these topics have been thoroughly described in \autoref{chap:implementation}, \autoref{chap:transmission-correction}, and \autoref{chap:multiiso-transmission-correction}. Instead, it will focus on the theoretical and implementation aspects unique to the capture yield correction. Specifically, this chapter details the physics of the multiple scattering correction, presents a verification of the physics model against a high-fidelity MCNP benchmark, and validates the integrated fitting procedure using synthetic data.


\section{Multiple Scattering in a Cylindrical Sample}
\label{sec:multiple-scattering}

Multiple scattering is an effect which occurs in which a neutron is scattered one or more times before being captured in a sample. As the thickness of the sample increases, the probability of a scattered neutron experiencing a subsequent reaction (that being a capture or additional scattering events) also increases. Therefore, neutrons which undergo scattering events can end up miscounted as captures, artificially increasing the measured yield.
\begin{figure}[H]
    \centering
    \includegraphics[width=0.75\linewidth]{Capture Yield/Figures/multiplescattering.png}
    \caption{Illustration of a neutron undergoing multiple scattering events, and getting miscounted as a capture event.}
    \label{fig:multiple-scattering-example}
\end{figure}
This is illustrated in \autoref{fig:multiple-scattering-example}, in which a neutron scatters twice before being captured.

This correction is made with reference to the thin sample approximation,
\begin{equation}
    \label{eq:thin-sample-approximation}
    \left\langle Y \right\rangle \approx n \left\langle \sigma_\gamma \right\rangle 
\end{equation}
where $\left\langle Y \right\rangle$ is the average yield, $n$ is the sample thickness, and $\left\langle \sigma_\gamma \right\rangle$ is the average capture cross section. The capture correction $C_C$ corrects the thin sample approximation such that
\begin{equation}
    \label{eq:correction-thin-sample-approximation}
    \left\langle Y \right\rangle = n \left\langle \sigma_\gamma \right\rangle C_C
\end{equation}

In order to calculate the correction, a series of Monte Carlo simulations are used to simulate a neutron undergoing a series of capture or scattering events in a cylindrical disc target. The target has a radius $R$, is parallel to the $z$ axis, and with front and rear faces located at $z=0$ and $z=n$, respectively, which are centered at $(x,y)=(0,0)$. A neutron is born at the energy of interest, $E^0$, and is incident on the front face of the target, with the location
\begin{equation}
    \label{eq:incident-neutron-location}
    \overrightarrow{x}^0 = \begin{bmatrix}
        x^0 \\
        y^0 \\
        z^0
    \end{bmatrix} =
    \begin{bmatrix} 
        r_0\cos{\left( \theta_0 \right)} \\
        r_0\sin{\left( \theta_0 \right)} \\
        0
    \end{bmatrix}
\end{equation}
where
\begin{align*}
    r_0 &= R\sqrt{\zeta}, \\
    \theta_0 &= 2\pi\zeta,
\end{align*}
and $\zeta$ is a uniformly distributed random value between 0 and 1. The neutron is also traveling along the $z$-axis, and therefore is born with the direction vector
\begin{equation}
    \label{eq:incident-neutron-direction}
    \overrightarrow{\Omega}^0 = \begin{bmatrix}
        \Omega^0_u \\
        \Omega^0_v \\
        \Omega^0_w
    \end{bmatrix} =
    \begin{bmatrix}
        0 \\
        0 \\
        1
    \end{bmatrix}
\end{equation}

Cross sections are sampled at the energy $E^0$ according to the procedure described in \autoref{sec:mc-sampling-from-average-parameters}. These values are then used to produce the average cross-sections,
\begin{align}
    \label{eq:average-total-cross-section}
    \overline{ \sigma_{t} }^{0} &= \sum_{j} \sigma_{t,j}^{0} \delta_{j} \\
    \label{eq:average-capture-cross-section}
    \overline{ \sigma_{\gamma} }^{0} &= \sum_{j} \sigma_{\gamma,j}^{0} \delta_{j} \\
    \label{eq:average-scattering-cross-section}
    \overline{ \sigma_{sc} }^{0} &= \sum_{j} \sigma_{sc,j}^{0} \delta_{j}
\end{align}
where $\overline{ \sigma_{t} }$, $\overline{ \sigma_{\gamma} }$, and $\overline{ \sigma_{sc} }$  are the weighted average total, capture, and scattering cross-sections respectively. The $0$ superscript indicates that the cross-sections are taken at the energy $E^{0}$, i.e., before any scattering events have occurred. The terms $\sigma_{j}^0$ and $\delta_{j}$ are the microscopic cross-section and relative abundance of the $j^{th}$ isotope, respectively.

\section{Scattering Events}

\subsection{Sampling Location}
\label{ssec:sampling-location-ms}
The distance to leave the sample is then calculated according to the shortest path to intersect with one of the surfaces in the direction in which the neutron is traveling. For the initial neutron, that is, before any scattering events have occurred, this is just the thickness of the sample $n$. However, for a scattering event this must be solved generally for a neutron that undergone $k$ scattering events. In this case, the neutron would have position vector $\overrightarrow{x}^k$ and direction $\overrightarrow{\Omega}^k$. The quantity that must be determined is the shortest path for it to leave the sample. In the case of a cylindrical target, there are three separate surfaces in which the neutron could escape:
\begin{enumerate}
    \item The front surface at $z=0$,
    \item The rear surface at $z=n$, and
    \item The cylindrical surface at $x^2 + y^2 = R^2$.
\end{enumerate}
Therefore, there are three distances that must be calculated:
\begin{align*}
    d_{front}   &\equiv \text{Distance neutron must travel to intersect with front face} \\
    d_{back}    &\equiv \text{Distance neutron must travel to intersect with back face} \\
    d_{cyl}     &\equiv \text{Distance neutron must travel to intersect with cylindrical surface}
\end{align*}
\begin{figure}[h]
    \centering
    \includegraphics[width=0.75\linewidth]{Capture Yield/Figures/ms_face_intersection.png}
    \caption{Example of the face intersection scheme, resulting in $d_{front}$ being selected as the shortest face until the particle exits.}
    \label{fig:ms-face-intersection}
\end{figure}
The distances to the planar front and back faces ($d_{front}$ and $d_{back}$) are determined. A finite, positive distance is calculated only if the neutron's path is directed towards the face in question; otherwise, the distance is treated as infinite. Using $\Omega_w^k$ to denote the z-component of the direction vector $\overrightarrow{\Omega}^k$, the distances are:
\begin{align}
    d_{front} &= \begin{cases} -z^k / \Omega_w^k, & \text{if } \Omega_w^k < 0 \\ \infty, & \text{otherwise} \end{cases} \\
    d_{back} &= \begin{cases} (n - z^k) / \Omega_w^k, & \text{if } \Omega_w^k > 0 \\ \infty, & \text{otherwise} \end{cases}
\end{align}
while $d_{cyl}$ is is determined as
\begin{equation}
    \label{eq:cyl-intersection-distance}
    d_{cyl} = \frac{\sqrt{b^2 + ac} - b}{a}
\end{equation}
where
\begin{align}
    a &= \left( \Omega_{u}^{k} \right)^2 + \left( \Omega_{v}^{k} \right)^2\\
    b &= x^k \Omega_{u}^{k} + y^k\Omega_{v}^{k} \\
    c &= R^2 - \left( x^k \right)^2 - \left( y^k \right)^2
\end{align}

The path length until the neutron escapes the sample, $d^k$, is the shortest of these three potential path lengths:
\begin{equation}
    d^k = \min{\left(d_{front}, d_{back}, d_{cyl}\right)}
\end{equation}

Next, a new set of total, capture, and elastic cross sections are sampled at energy $E^k$, which are then used to determine the sampled distance until the next collision,
\begin{equation}
    \label{eq:free-path-sampling}
    s^k = -\frac{1}{\overline{\sigma_{t} }^{k} }  \ln{\left\{
                1 - \zeta \left[ 1 - \exp{\left( \overline{\sigma_{t}}^{k} d^k \right)} \right] 
    \right\}}
\end{equation}
where $\zeta$ is a randomly selected value between 0 and 1. This distance term $s^{k}$ is then used to calculate the location of the next scattering event,
\begin{equation}
    \label{eq:sampling-new-location}
    \overrightarrow{x}^{k+1} = \overrightarrow{x}^{k} + \Omega^{k}s^{k}
\end{equation}

\subsection{Sampling Energy}
\label{ssec:sampling-energy-ms}
The energy in which the neutron leaves the scattering event, i.e., $E^{k+1}$, must be determined. This introduces its own complication, as $E^{k+1}$ is dependent on the mass of the isotope it interacts with, along with its scattering angle. 

First, addressing the mass dependency. This quantity must be sampled according to each isotopes calculated scattering cross section and their given abundances, as
\begin{equation}
    w_{j} = \frac{\delta_j \sigma_{j,sc}^{k} }{\overline{\sigma_{sc}}^{k}}
\end{equation}
This quantity is then used to calculate a cumulative weight, such that
\begin{equation}
    W_j = \sum_{j} w_j
\end{equation}
This defines the window of probability for the scattering event occurring on isotope $j$ as being between $(W_{j-1}, W_{j}]$.
A random number $\zeta$ is sampled between 0 and 1, is used such that
\begin{equation}
    W_{j-1} < \zeta \leq W_{j}
\end{equation}
determines that the neutron will be sampled as scattering off a nucleus with the mass $A_j$.

Next, addressing the angle dependency. The scattering angle is assumed to be isotropic, therefore can scatter with the angle $\phi$, determined as
\begin{equation}
    \phi^k = 2\pi\zeta
\end{equation}
Therefore, the energy $E^{k+1}$ is determined as
\begin{equation}
    E^{k+1} = E^{k} \frac{A_j^{2} + 2A_{j}\cos{\left(\phi^{k}\right) + 1}}{\left( A_{j} + 1 \right)^2}
\end{equation}

\subsection{Sampling Direction}
\label{ssec:sampling-direction}
The final component of determining the new parameters of a scattering event include calculating the direction the neutron is traveling, $\overrightarrow{\Omega}^{k+1}$. Similar to \autoref{ssec:sampling-energy-ms}, an angle $\theta^k$ must be sampled uniformly in the range $[0,2\pi]$. However $\theta$ must be converted from center of mass scale to lab scale, where
\begin{align}
    \cos{\theta'} &= \frac{1 + A_j \cos{\left( \theta^k \right)}}{\sqrt{1 + \left( A_j\right)^2 + 2A_j\cos{\left(\theta^k \right)}}} \\
    \sin{\theta'} &= \sqrt{1 - \left( \cos{\theta'} \right)^2}
\end{align}
while $\phi=\phi'$.
Determining $\overrightarrow{\Omega}^{k+1}$ from $\overrightarrow{\Omega}^{k}$, $\phi'$, and $\theta'$ proceed according to
\begin{equation}
    \label{eq:direction-calculation-ms}
    \overrightarrow{\Omega}^{k+1} = \begin{bmatrix}
        \Omega^{k+1}_u \\[8pt]
        \Omega^{k+1}_v \\[8pt]
        \Omega^{k+1}_w
    \end{bmatrix}
    = \begin{bmatrix}
        \frac{\Omega_{u}^{k} \Omega_{v}^{k}} { \sqrt{1 - \left(\Omega_{w}^{k}\right)^2 }} &
        \frac{-\Omega_v^k} { \sqrt{1 - \left(\Omega_{w}^{k}\right)^2 }} &
        \Omega_u \\[10pt]
        \frac{\Omega_{v}^{k} \Omega_{w}^{k}} { \sqrt{1 - \left(\Omega_{w}^{k}\right)^2 }} &
        \frac{-\Omega_u^k} { \sqrt{1 - \left(\Omega_{w}^{k}\right)^2 }} &
        \Omega_v \\[10pt]
        \frac{-1} { \sqrt{1 - \left(\Omega_{w}^{k}\right)^2 }} &
        0 &
        \Omega_w
    \end{bmatrix} \times
    \begin{bmatrix}
        \sin{\theta'}\cos{\phi'} \\[8pt]
        \sin{\theta'}\sin{\phi'} \\[8pt]
        \cos{\theta'}
    \end{bmatrix}
\end{equation}
However, in the case where $\left(\Omega_w^k\right)^2 = 1$, \autoref{eq:direction-calculation-ms} cannot be used as it would divide by 0. Instead, the operation
\begin{equation}
        \overrightarrow{\Omega}^{k+1} = \begin{bmatrix}
        \Omega^{k+1}_u \\[8pt]
        \Omega^{k+1}_v \\[8pt]
        \Omega^{k+1}_w
    \end{bmatrix} = \Omega_w^k     \begin{bmatrix}
        \sin{\theta'}\cos{\phi'} \\[8pt]
        \sin{\theta'}\sin{\phi'} \\[8pt]
        \cos{\theta'}
    \end{bmatrix}
\end{equation}
is used instead. Finally, all the components required to sample the subsequent scattering event are obtained: $\overrightarrow{x}^{k+1}$, $\overrightarrow{\Omega}^{k+1}$, and $E^{k+1}$.

\subsection{Weighting Neutrons}
\label{ssec:killing-neutrons-ms}
A calculation must be made to ensure that sufficient scattering events are being accounted for, but not so many scattering events that it is too computationally expensive, or unrealistic collision events bias the yield in some way.

A weighted importance is used to determine the contribution of a particle to capture yield and when to consider the neutron sufficiently unimportant. This is the same calculation performed by default with MCNP\cite{mcnp}. 

Given the sampled distance $s^{k}$,
\begin{align}
    \label{eq:tot-frac}
    \gamma_{tot}^{k} &=  1 - \exp{ \left(-s^{k} \overline{\sigma_{t}}^{k} \right)} \\
    \label{eq:el-frac}
    \gamma_{sc}^{k} &= \gamma_{tot}^{k} \frac{\overline{\sigma_{sc}}^{k}} {\overline{\sigma_{t}}^{k}} \\
    \label{eq:cap-frac}
    \gamma_{cap}^{k} &= \gamma_{tot}^{k} \frac{\overline{\sigma_{\gamma}}^{k}}{\overline{\sigma_{t}}^{k}}
\end{align}

The simulation employs an implicit capture (or survival biasing) scheme. At each collision, the neutron is forced to scatter, and its statistical weight is reduced by the scattering probability. The "lost" weight is tallied as the capture contribution. Starting with an initial weight $w^{0}=1$, the weight after the $k^{th}$ collision is:
\begin{equation}
    \label{eq:neutron-weight}
    w^{k} = w^{k-1}\gamma_{sc}^{k}
\end{equation}
where $\gamma_{sc}^{k}$ is the scattering interaction fraction from \autoref{eq:el-frac}.

This quantity is used in two ways; first, the factor is used to determine the total number of collisions $K$ that a neutron will experience before it is no longer statistically significant. If the weight ends up less than some cutoff factor, $\varepsilon$, the total number of collisions is reached,
\begin{equation}
    \label{eq:collision-counter}
    K = \min{ \left\{ k : w^{k} \leq \varepsilon \right\}}
\end{equation}

The second purpose of the weighting factor is to determine the total probability of capture after $K$ collisions for a single neutron history. The total probability of a neutron being captured is the sum of all weighted capture events until the collision $K$. The probability that a neutron will be captured at the $k^{th}$ collision is the product of the neutron's surviving weight from the previous collision times the capture interacting fraction $\gamma_{cap}^{k}$. This is given explicitly as
\begin{equation}
    \label{eq:capture-total-prob}
     p = \sum_{k=1}^{K} w^{k-1}\gamma_{cap}^{k}
\end{equation}

\section{Calculating The Capture Correction Factor}
\label{sec:capture-correction-factor}

The two additional quantities required to calculate the capture correction factor are the energy averaged capture cross-section, $\langle \overline{\sigma_{\gamma}} \rangle$, and energy averaged capture probability, $\langle p \rangle$. These are obtained using the $p$ term from \autoref{eq:capture-total-prob}, and $\overline{\sigma_{\gamma}}^{0}$ from \autoref{eq:average-capture-cross-section}, as
\begin{equation}
    \label{eq:avg-capture-prob}
    \langle p \rangle = \frac{1}{N} \sum_{i=1}^{N} p_{i}
\end{equation}
and
\begin{equation}
    \label{eq:avg-capture-xs}
    \langle \overline{\sigma_{\gamma}} \rangle = \frac{1}{N} \sum_{i=1}^{N} \overline{\sigma_{\gamma}}^{0}_{i}
\end{equation}
where the sum is over $N$ total neutron histories. $p$ and $\overline{\sigma_{\gamma}}$ recieve the $i$ subscript to denote that their quantities refer to the $i^{th}$ neutron history.

These quantities are then used to calculate the capture correction factor,
\begin{equation}
    \label{eq:capture-correction-factor-calc}
    C_{C} = \frac{\langle p \rangle }{\langle \overline{ \sigma_{\gamma} } \rangle n}
\end{equation}
which can then be used in \autoref{eq:correction-thin-sample-approximation} to estimate the energy averaged capture yield $\langle Y \rangle$. 

It should be noted that the average capture cross section $\langle \overline{\sigma_{\gamma}} \rangle$ calculated from \autoref{eq:avg-capture-xs} and $\langle \sigma_{\gamma} \rangle$ are not necessarily the same quantity. The $\langle \overline{\sigma_{\gamma}} \rangle$ is computed via Monte Carlo simulation, while $\langle \sigma_{\gamma} \rangle$ from \autoref{eq:correction-thin-sample-approximation} is calculated analytically.

\section{SESH Improvements}
    \subsection{Energy Deposition Improvement}
        Originally in SESH, there was a very simple approximation for sampling what energy an interaction occurred at. It was assumed that every neutron lost the average scattering energy per collision, independent of scattering angle, i.e.,
        \begin{equation}
            E' = E \left[ 1 - \frac{2A}{\left( 1 + A \right)^2} \right]
        \end{equation}
        for all post-collision neutrons. This enabled all of the energy dependent parameters that did not get sampled using the Monte Carlo method to be pre-calculated efficiently. However, it was assumed this would be an insufficient approximation for thick samples in which many collisions were expected.
        
        This was substituted with an angle-dependent energy sampling procedure. The scattering angle of each post collision neutron would be sampled assuming an isotropic scattering distribution, and the post collision energy would be calculated as a function of the scattering angle,
        \begin{equation}
            E' = E \frac{A^2 + 2 A \mu_c + 1}{\left( A + 1 \right)^2}
        \end{equation}
in which
        \begin{equation}
            \mu_c = \cos\phi_c
        \end{equation}
        where $\phi_c$ is the exit angle of the post-collision neutron in the center-of-mass system.
        \begin{figure}[H]
            \centering
            \includegraphics[width=0.75\linewidth]{Capture Yield/Figures/multiple_scattering.png}
            \caption{Distribution of neutrons after each collision using the old SESH post-collision energy sampling method and the new angle-dependent sampling in a 12mm \textsuperscript{181}Ta sample at 3keV initial incident energy.}
            \label{fig:multiple-scattering}
        \end{figure}

    \subsection{Incident Neutron Location Error}

    SESH had an observed problem in which as the radius of the neutron beam size increased past some value relative to the total sample surface, it would fail at some point in the calculation. This ended up being attributable to an implementation error in algorithm for calculating the initial location for which the neutron is incident on the sample. Specifically, memory errors would be observed when $r_{sample}/r_{beam} \leq \sqrt{2}$.

    After some investigation, it was determined that the incident neutron location algorithm was implemented as
    \begin{align*}
        x &= R\sqrt{\zeta}, \qquad y=R\sqrt{\zeta} \\
    \end{align*}
    Consequently, rather than the intended circular beam with uniform probability the distribution was a non-uniform, square beam. This was replaced with an improved sampling algorithm which correctly sampled across the face of the cylinder. A comparison between the original and improved algorithms is shown in \autoref{fig:loc-sampling-2d-comparison}.
    \begin{figure}[h]
        \centering
        \begin{adjustbox}{width=1.2\textwidth, center}
            \includegraphics{Capture Yield/Figures/sampling_2d_comparison.png}
        \end{adjustbox}
        \caption{Comparing the observed distribution of sampled neutron incident locations across the face of a cylindrical target from original SESH algorithm versus the new implementation}
        \label{fig:loc-sampling-2d-comparison}
    \end{figure}
    Not only was this a headache to deal with, but the non-uniform distribution had the potential to under-estimate the correction factor. Neutrons sampled as interacting closer to the edges are less likely to experience significant multiple scattering events.

    \begin{figure}[h]
        \centering
        \includegraphics[width=0.75\linewidth]{Capture Yield/Figures/sampling_1d_comparison.png}
        \caption{Comparison between improved and original algorithm regards to sampled radius}
        \label{fig:sampling-1d-comparison}
    \end{figure}


\section{Verification of the Capture Yield Physics Model}
\label{sec:capture-yield-verification}

With the theoretical framework for the multiple scattering correction established, it is essential to verify that the physics implemented in the SESH code is accurate. The primary objective of this section is to validate the core physics of the multiple scattering and capture yield algorithm. This is achieved by comparing the results from SESH against a high-fidelity benchmark, ensuring that the model can be trusted before it is integrated into the full fitting procedure. To avoid confounding issues discovered in previous chapters, a clean, single-isotope benchmark case is used.


\subsection{Validation Strategy}
\label{ssec:capture-yield-validation-strategy}

Validating the capture yield correction requires a benchmark that isolates the effects of multiple scattering without introducing additional sources of systematic error. As demonstrated in the multi-isotope transmission validation (\autoref{chap:multiiso-transmission-correction}), the probability table (P-Table) method used by MCNP and NJOY exhibits a variance-conservation issue in multi-isotope systems, leading to an underprediction of self-shielding effects. This limitation makes multi-isotope comparisons unsuitable for direct validation of the capture yield correction.

To avoid this ambiguity, validation is performed using a single-isotope benchmark. The isotope \textsuperscript{181}Ta is selected because it eliminates multi-isotope variance effects, possesses a well-characterized resonance structure in the unresolved resonance region (URR), and was previously used for transmission correction validation (\autoref{chap:transmission-correction}), providing a consistent reference point.

Unlike transmission corrections, capture yield corrections are sensitive not only to sample thickness and temperature, but also to geometric effects arising from neutron transport within the sample. In particular, the correction depends on both the sample radius, which governs lateral neutron leakage, and the radius of the incident neutron beam, which influences the average neutron path length prior to escape. The expected physical behavior is that the capture yield correction increases as the beam radius decreases, since neutrons traverse longer paths within the sample, and decreases as the sample radius decreases, due to enhanced radial leakage. These trends provide clear, physically motivated expectations against which the model can be evaluated.

Validation is carried out by comparing capture yield corrections calculated by SESH against results from a high-fidelity MCNP6 benchmark \cite{mcnp}. The MCNP geometry follows the configuration described in \autoref{ssec:mcnp-benchmark-model}, consisting of a mono-directional neutron beam incident on a cylindrical \textsuperscript{181}Ta sample. Capture tallies are used to compute the history-averaged capture probability as a function of energy, from which the benchmark correction factor $C_C$ is obtained.

The benchmark spans a broad range of experimentally relevant configurations, as well as thicker stress-test cases that probe regimes with stronger attenuation and multiple scattering. The full set of parameters considered in the validation study is summarized in \autoref{tab:capture-validation-params}.

\begin{table}[h!]
\centering
\caption{Benchmark parameter ranges used for validation of the capture yield correction model.}
\label{tab:capture-validation-params}
\begin{tabular}{l l}
\hline
\textbf{Parameter} & \textbf{Values Considered} \\
\hline
Sample thickness ($n$) & 2\,mm, 4\,mm, 8\,mm, 12\,mm \\
Sample radius ($R_{\text{sample}}$) & 20\,mm, 40\,mm, 60\,mm, 80\,mm \\
Beam radius ($R_{\text{beam}}$) & 0.0, 0.25, 0.5, 0.75, 1.0 $\times R_{\text{sample}}$ \\
\hline
\end{tabular}
\end{table}

This parameter space is sufficient to verify the capture yield correction under conditions relevant to experiment, while also identifying regimes where discrepancies may emerge as multiple scattering and geometric effects become increasingly important.

\subsection{Comparison and Results}
\label{ssec:comparison-and-results}

The capture correction factor, $C_C$, calculated by the modernized SESH code was compared directly against the MCNP benchmark results for \textsuperscript{181}Ta. The comparison was performed across the full matrix of sample thicknesses, sample radii, and beam radii previously described.

The results for varying sample thickness, shown in \autoref{fig:cy-verification-ta181} and \autoref{fig:cy-verification-err-ta181}, demonstrate excellent agreement between SESH and MCNP for the thinner samples. As sample thickness increases, SESH begins to show a small, systematic under-prediction of the correction factor, particularly at lower energies where self-shielding is strongest. However, even for the thickest (12mm) sample, the relative error generally remains below 2\%.

The validation study also confirmed the expected physical trends with respect to sample and beam geometry, as shown in \autoref{fig:cy-verification-sample-radius}, \autoref{fig:cy-verification-err-sample-radius}, \autoref{fig:cy-verification-beam-radius}, and \autoref{fig:cy-verification-err-beam-radius}. The capture yield correction was observed to increase as the beam radius decreases and to decrease as the sample radius decreases. 


\begin{figure}[h!]
    \centering
    %\includegraphics[width=\linewidth]{Capture Yield/Figures/cy_verification_ta181.png}
    \caption{Comparison of the capture correction factor ($C_C$) for \textsuperscript{181}Ta calculated by SESH and MCNP for various sample thicknesses.}
    \label{fig:cy-verification-ta181}
\end{figure}

\begin{figure}[h!]
    \centering
    %\includegraphics[width=\linewidth]{Capture Yield/Figures/cy_verification_err_ta181.png}
    \caption{Relative error between the SESH and MCNP calculation of $C_C$ for \textsuperscript{181}Ta for various sample thicknesses.}
    \label{fig:cy-verification-err-ta181}
\end{figure}

\begin{figure}[h!]
    \centering
    %\includegraphics[width=\linewidth]{Capture Yield/Figures/cy_verification_sample_radius.png}
    \caption{Comparison of the capture correction factor ($C_C$) for \textsuperscript{181}Ta calculated by SESH and MCNP for various sample radii.}
    \label{fig:cy-verification-sample-radius}
\end{figure}

\begin{figure}[h!]
    \centering
    %\includegraphics[width=\linewidth]{Capture Yield/Figures/cy_verification_err_sample_radius.png}
    \caption{Relative error between the SESH and MCNP calculation of $C_C$ for \textsuperscript{181}Ta for various sample radii.}
    \label{fig:cy-verification-err-sample-radius}
\end{figure}

\begin{figure}[h!]
    \centering
    %\includegraphics[width=\linewidth]{Capture Yield/Figures/cy_verification_beam_radius.png}
    \caption{Comparison of the capture correction factor ($C_C$) for \textsuperscript{181}Ta calculated by SESH and MCNP for various beam radii.}
    \label{fig:cy-verification-beam-radius}
\end{figure}

\begin{figure}[h!]
    \centering
    %\includegraphics[width=\linewidth]{Capture Yield/Figures/cy_verification_err_beam_radius.png}
    \caption{Relative error between the SESH and MCNP calculation of $C_C$ for \textsuperscript{181}Ta for various beam radii.}
    \label{fig:cy-verification-err-beam-radius}
\end{figure}

\subsection{Conclusion on Physics Model Validity}
\label{ssec:conclusion-physics-model}

The strong agreement between the SESH calculations and the MCNP benchmark for the \textsuperscript{181}Ta case validates the core physics of the multiple scattering and capture yield algorithm. The observed small discrepancies for very thick samples are consistent with trends seen in the transmission validation (\autoref{chap:transmission-correction}) and define the practical limits of the model's applicability. This successful verification provides the necessary confidence to proceed with integrating this capability into the full SAMMY fitting workflow.


