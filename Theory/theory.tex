\section{Unresolved Resonance Region}
\label{sec:unresolved-resonance-region}

The Unresolved Resonance Region (URR) presents similar nuclear reaction phenomena as the Resolved Resonance Region (RRR); however, the crucial distinction lies in the experimental observability of individual resonances. In the URR, the resonances are too closely spaced in energy to be experimentally resolved, meaning that individual resonance parameters (e.g., resonance energies and widths) cannot be directly determined. This stands in contrast to the RRR, where the Single-Level Breit-Wigner (SLBW) approximation (or more complex multi-level formulations) can precisely describe the energy-dependent cross section based on discrete resonance parameters.

Because a complete set of individual resonances cannot be resolved, it becomes impractical and often impossible to calculate the "true" point-wise cross section in the URR. Therefore, instead of determining explicit resonance shapes, evaluators resort to calculating \textit{average} cross sections. These average cross sections are derived from statistical properties of the resonances, such as average level spacings and average resonance widths, rather than their individual values. This approach assumes that these parameters are sufficiently statistical and therefore accurately models the average cross section in the URR.

This formulation provides a convenient analytical form for the average total cross-section. However, it is missing a crucial element of the unresolved resonance region. The underlying statistical nature of the \textit{true} cross section in the unresolved resonance region exhibits significant rapid fluctuations around this analytical average cross section. These rapid fluctuations are responsible for resonant self-shielding. Resonant self-shielding significantly impacts transport equations, and must be accounted for to accurately model neutron transport. The following section characterizes the impact of cross section variance due to resonance fluctuations.

\section{Resonance Self Shielding}
\label{sec:resonance-self-shielding}
Resonance self-shielding is a phenomena which occurs when the resolution of a measurement is less than the level spacing between resonances. It can be summarized as the non-linear relationship between an observable quantity, i.e., transmission, and the total cross-section. Self-shielding is a consequence of the variance of the cross-section over some energy region. As the variance in the cross-section increases, the energy-averaged transmission increases relative to the transmission of the average cross-section,
\begin{equation}
    \label{eq:self-shielding-definition}
    \langle T \rangle > e^{-n \langle \sigma \rangle}
\end{equation}
in which the operator $\langle \cdot \rangle$ denotes an energy-averaged quantity,
\begin{equation}
    \label{eq:energy-average-operator}
    \langle \cdot \rangle = \frac{1}{E_{1} - E_{0}} \int_{E_{0}}^{E_{1}} \cdot dE
\end{equation}
The energy-averaged transmission, $\langle T \rangle$, is given by:
\begin{equation}
    \label{eq:energy-average-Transmission}
    \langle T \rangle = \langle e^{-n \sigma(E)} \rangle
\end{equation}
in which $\sigma(E)$ is the total cross-section as a function, and $n$ is the thickness of some material.

To demonstrate how the variance in the cross-section contributes to self-shielding, consider the Taylor-Series expansion of $\langle T \rangle$. The energy-differential transmission can be redefined in terms of the average cross-section over that energy bin, and the deviation of the energy-differential cross-section from the average:
\begin{equation}
    \left\langle e^{-n \sigma(E)} \right\rangle = \left\langle e^{-n \langle \sigma \rangle} e^{-n \left(\sigma(E) - \langle \sigma \rangle \right)} \right\rangle
\end{equation}

Because $\langle \sigma \rangle$ has to be constant over that energy region, that can be rearranged to:
\begin{equation}
    \label{eq:taylor-series-step}
    \left\langle e^{-n \sigma(E)} \right\rangle = e^{-n \langle \sigma \rangle} \left\langle e^{-n \left( \sigma(E) - \langle \sigma \rangle \right)} \right\rangle
\end{equation}
The second term in \autoref{eq:taylor-series-step} can be approximated using a Taylor expansion:
\begin{equation}
    e^{-n \left( \sigma(E) - \langle \sigma \rangle \right)} \approx 1 + n (\sigma(E) - \langle \sigma \rangle) + \frac{n^{2}}{2} (\sigma(E) - \langle \sigma \rangle)^2
\end{equation}
Taking the average of this approximation, and noting that $\langle \sigma(E) - \langle \sigma \rangle \rangle = 0$, we get:
\begin{equation}
    \left\langle e^{-n \left( \sigma(E) - \langle \sigma \rangle \right)} \right\rangle \approx 1 + \frac{n^{2}}{2} \langle (\sigma(E) - \langle \sigma \rangle)^2 \rangle = 1 + \frac{n^{2}}{2} \text{Var}(\sigma)
\end{equation}
This leads to the final approximation for $\langle T \rangle$:
\begin{equation}
    \label{eq:taylor-series-expansion}
    \langle T \rangle \approx e^{-n \langle \sigma \rangle} \left(1 + \frac{n^{2}}{2} \text{Var}(\sigma) \right) = e^{-n \langle \sigma \rangle} + \frac{n^{2}}{2} e^{-n \langle \sigma \rangle} \text{Var}(\sigma)
\end{equation}

Clearly, according to \autoref{eq:taylor-series-expansion}, as the variance in the cross-section increases, the self-shielding will increase. Consequently, by accounting for variance in the cross-section, the self-shielding will be accounted for.

\section{The Self-Shielding Correction Factor}
\label{sec:correction-factor}

The self-shielding correction factor is used to solve the inequality given in \autoref{eq:self-shielding-definition}. It is defined as:
\begin{equation}
    C_{T} = \frac{\langle e^{-n \sigma (E)} \rangle}{ e ^{- n \langle \sigma \rangle}}
\end{equation}

The experimental transmission obtained in the unresolved resonance region, $\langle T \rangle$, is given by:
\begin{equation}
    \label{eq:transmission}
    \left\langle T \right\rangle = \frac{1}{E_2 - E_1} \int_{E_1}^{E_2} e^{-n \sigma(E)}dE
\end{equation}
in which $\sigma(E)$ is the true cross section of the material. It is assumed that the energy bin is wide enough such that a statistically meaningful sample of resonances are captured.

In order to express the average transmission as a function of the average cross section:
\begin{equation}
    \label{eq:transmission-of-average}
    \left\langle T \right\rangle = \frac{1}{E_2 - E_1} \int_{E_1}^{E_2} e^{-n \left( \sigma(E) + \langle \sigma \rangle - \langle \sigma \rangle \right)}dE
\end{equation}
in which 
\begin{equation}
    \label{eq:energy-average-cross-section}
    \left\langle \sigma \right\rangle = \frac{1}{E_2 - E_1} \int_{E_1}^{E_2} \sigma(E) dE
\end{equation}
Therefore, the term defining the ``transmission of the average cross section'', $e^{-n \langle \sigma \rangle}$, can be removed, such that 
\begin{equation}
    \label{eq:corrected-average-transmission}
    \langle T \rangle = e^{-n \langle \sigma \rangle} C_T
\end{equation}
in which the term $C_T$ is known as the transmission correction factor, and is defined as:
\begin{equation}
    \label{eq:transmission-correction-factor}
    C_T = \frac{1}{E_2 - E_1} \int_{E_1}^{E_2} e^{-n \left( \sigma(E) - \langle \sigma \rangle \right)}dE
\end{equation}

In other words, the correction factor is the average deviation of the true cross section from the mean cross section over a given energy bin.
\begin{equation}
    \label{eq:ct-elegant-form}
    C_T = \left\langle e^{-n (\sigma - \langle \sigma \rangle)} \right\rangle
\end{equation}
or in a form that will be more useful in a later section (e.g. \autoref{sec:sesh}),
\begin{equation}
    \label{eq:ct-sesh-form}
    C_T = \frac{\langle e^{-n \sigma} \rangle}{ e^{-n \langle \sigma \rangle}}
\end{equation}


It should be noted that this is specifically the \textit{transmission} self-shielding correction factor. Other forms will be used later, but the end goal is the same: determine the average measurement as a function of the average cross section.

\section{Sampling Cross Sections from Average Parameters}
\label{sec:mc-sampling-from-average-parameters}

As previously stated in \autoref{sec:resonance-self-shielding}, in the Unresolved Resonance Region, individual resonances cannot be directly observed due to their close spacing. As such, only average resonance parameters can be obtained from measurements in the URR. While analytical equations, (e.g., \autoref{eq:hauser-feshbach-cross-section}), provide energy-averaged cross sections based on these average parameters, these analytical solutions intrinsically yield only the average behavior. They do not, however, directly provide the variance of the true, fluctuating energy-dependent cross section within the URR. The phenomenon of resonance self-shielding (discussed in \autoref{sec:resonance-self-shielding}) is directly dependent on this variance. However, this only discusses the determination of self-shielding from a known energy differential cross-section, which is not able to be observed.

Since the true energy-differential cross section in the URR is unknown, and its variance cannot be directly calculated from the average parameters alone, Monte Carlo methods provide the necessary means to address this challenge. By statistically sampling individual resonance parameters from their known probability distributions, Monte Carlo sampling allow for the generation of distribution of cross section "realizations" which agree with the average analytical cross section and account for the variance in cross section around said average. This section will detail how these fluctuating cross sections are determined from average parameters by leveraging the Single-Level Breit-Wigner (SLBW) equation, and sampling parameters from their respective statistical distributions.

\subsection{The Single-Level Breit-Wigner (SLBW) Equation for Cross Section Reconstruction}
\label{sec:slbw-for-sampling}

The Single-Level Breit-Wigner (SLBW) approximation\cite{t2}, typically used in the Resolved Resonance Region (RRR) for well-separated resonances, serves as the fundamental basis for reconstructing cross-section distributions from sampled resonance parameters in the URR. While more complex formalisms such as the R-Matrix theory are required to accurately model resonances in the RRR, the statistical nature of the URR allows for the use of the computationally simpler SLBW formula to generate cross-section realizations that are statistically consistent with the average parameters. For a given channel $c$, the total SLBW cross section\footnote{This isn't the \textit{total} total cross section as is generally referred to. Instead, it is the cross-section associated with forming a compound nucleus in a particular $J,\ell,\pi$ state, irrespective of the outgoing particle after the formation of that compound nucleus (neutron in the case of elastic scattering, $\gamma$-ray in the case of capture, etc). The incoming channel $c$ is used as shorthand for that $J,\ell,\pi$ state. The conventional \textit{total} total cross section is equivalent to summing over all possible channels.} is determined as a sum over resonances $r$ via:
\begin{equation}
    \label{eq:slbw}
    \sigma_c = \frac{ 4 \pi g_c}{k_c^2} 
        \left\{ \sin^2{\phi_c} + \sum_r \frac{\Gamma_{n,r,c}}{\Gamma_{tot,r,c}}
            \left( \psi_r \cos{2 \phi_{c}} + \chi_r \sin{2\phi_c}
            \right)
        \right\}
\end{equation}
The statistical spin factor, $g_c$ is defined as
\begin{equation}
    g_c = \frac{2J + 1}{2(2I + 1)}
\end{equation}
in which $I$ is the spin of the target nucleus, and $J$ is the total angular momentum (spin) of the resonance. The channel spin $s$ is formed by the coupling of the target spin $I$ and the neutron spin, while $J$ is formed by the coupling of $s$ and the orbital angular momentum $\ell$. $\phi_c$ is the hard-sphere scattering phase shift of a particular channel.

$\psi$ and $\chi$ are the Doppler broadened resonance profiles. Those are determined by using the Faddeeva function\cite{algo-916}, such that
\begin{equation}
    \label{eq:faddeeva}
    \psi + i\chi = \frac{\sqrt{\pi}}{2}\theta w \left(\frac{\theta x}{2}, \frac{\theta}{2} \right)
\end{equation}
in which $\theta$ is the Doppler width,
\begin{equation}
    \theta = \frac{\Gamma_{tot}}{\sqrt{\frac{4 \kappa T E}{A}}}
    \label{eq:doppler-width}
\end{equation}
$x$ is the resonance shape,
\begin{equation}
    x = \frac{2 (E - E_r)}{\Gamma_{tot}}
\end{equation}
and $w(\alpha,\beta)$ is defined as
\begin{equation}
    w(\alpha,\beta) = \frac{i}{\pi} \int_{-\infty}^{\infty} \frac{e^{-t^2}}{\alpha + i\beta - t}dt
\end{equation}
where the arguments to the Faddeeva function are $\alpha = \frac{\theta x}{2}$ and $\beta = \frac{\theta}{2}$.

The total microscopic cross section for a given nuclide at a specific energy is then determined by summing the contributions from all possible incoming channels $c$. Each channel corresponds to a unique set of quantum numbers (spin, orbital angular momentum, and parity) that contribute to the reaction. The other relevant reaction cross sections for a given channel can also be calculated from these parameters. The capture cross section:
\begin{equation}
    \sigma_{c,\gamma} =  \frac{ 4 \pi g_c}{k_c^2}  \sum_r \frac{\Gamma_{n,r,c} \Gamma_{\gamma,r,c}}{\Gamma_{r,tot}^2} \psi_r
\end{equation}
And the fission cross section is nearly identical:
\begin{equation}
    \sigma_{c,\gamma} =  \frac{ 4 \pi g_c}{k_c^2}  \sum_r \frac{\Gamma_{n,r,c} \Gamma_{f,r,c}}{\Gamma_{r,tot}^2} \psi_r
\end{equation}

The determination of all possible channel states will be elaborated in a later section. It is crucial to note that this SLBW formulation, as presented, presumes that the individual resonance parameters ($E_r$, $\Gamma_{n,r}$, and $\Gamma_{tot,r}$, for each resonance $r$) are precisely known. However, in the URR, these individual parameters are not experimentally resolved. Instead, only their \textit{average} values and statistical distributions are known. This discrepancy necessitates the use of statistical sampling techniques to generate instances of these resonance parameters from their average values and distributions, which then allows for the reconstruction of fluctuating cross sections using the SLBW equation.

\subsection{Sampling Cross-Sections from Average Parameters}
\label{sec:sampling-parameters}


The procedure for generating a statistical realization of the cross section over a given energy range involves several steps, iterated across different channels and multiple times to build a comprehensive distribution. For each given channel $c$, defined by a unique $(\ell, J, \pi)$ combination, a ladder of resonance energies is determined. This is done by sampling from the Wigner Distribution, in which
\begin{equation}
    \label{eq:wigner-distribution}
    P(s) = \frac{s\pi}{2 \langle D \rangle_c^2} e^{-\frac{\pi s^2}{4 \langle D \rangle_c^2}}
\end{equation}
where $\langle D \rangle_c$ is the average level spacing associated with that channel $c$, and $s$ is the sampled level spacing. This is used to produce a resonance ladder
\begin{equation}
    E_{c,r+1} = E_{c,r} + s
\end{equation}

The following step is to sample neutron widths for each channel and resonance energy.

\begin{equation}
    \label{eq:porter-thomas}
    \Gamma_{n,r,c} = P\left( \frac{\Gamma_{n,c}}{\langle \Gamma_{n,c}  \rangle}\right)\langle \Gamma_{n,c}  \rangle
\end{equation}
in which $\langle \Gamma_{n,c}  \rangle$ is the average neutron width for that channel, and $P\left( \Gamma_{n,c} / \langle \Gamma_{n,c}  \rangle \right)$ is the Porter-Thomas distribution, where:
\begin{equation}
    P(x) = \frac{e^{-x/2}}{\sqrt{2\pi x}}
\end{equation}

Once a neutron width for that channel is sampled, the total resonance width is determined. This is done as a sum over the neutron width plus the other possible channels, i.e., radiation width, fission width, and inelastic width:
\begin{equation}
    \Gamma_{tot,r,c} = \Gamma_{n,r,c} + \Gamma_{n',r,c} + \Gamma_{\gamma,r,c} + \Gamma_{f,r,c}
\end{equation}

These sampled parameters are processed via the steps described in \autoref{sec:slbw-for-sampling} to be input into \autoref{eq:slbw} and generate a cross section for that given channel $c$. This process is repeated for all possible channels to provides a single realization of the total cross section for a given set of average parameters:
\begin{equation}
    \sigma_{tot}^i = \sum_c \sigma_c
\end{equation}
Finally, this is repeated some $N$ number of realizations to produce a population of sampled cross sections, $\{ \sigma_{tot}^1, \sigma_{tot}^2, \ldots, \sigma_{tot}^{N} \}$. This set of sampled cross sections, if defined using accurate average parameters over some statistical number of resonances, will accurately reproduce the true variance in the energy-differential cross section over some energy bin.

This is the general procedure of how nuclear data processing codes (NJOY\cite{njoy}, FRENDY\cite{frendy}, AMPX\cite{scale2024manual}) simulate cross-section variance required to account for self-shielding in the URR. Self-shielding due to cross section variance from resonant fluctuations can therefore be accurately modeled strictly from average parameters.

Note, this only addresses total cross section. It does not mention partial cross sections, interpreting between parameters to be used in \autoref{eq:hauser-feshbach-cross-section} and \autoref{eq:slbw}, energy dependencies, or any approximations. Those details are implementation-specific, and will be discussed in subsequent chapters.
