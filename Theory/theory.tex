\section{Unresolved Resonance Region}
\label{sec:unresolved-resonance-region}

The Unresolved Resonance Region (URR) presents similar nuclear reaction phenomena as the Resolved Resonance Region (RRR); however, the crucial distinction lies in the experimental observability of individual resonances. In the URR, the resonances are too closely spaced in energy to be experimentally resolved, meaning that individual resonance parameters (e.g., resonance energies and widths) cannot be directly determined. This stands in contrast to the RRR, where the Single-Level Breit-Wigner (SLBW) approximation (or more complex multi-level formulations) can precisely describe the energy-dependent cross section based on discrete resonance parameters.

Because a complete set of individual resonances cannot be resolved, it becomes impractical and often impossible to calculate the ``true" point-wise cross section in the URR. Therefore, instead of determining explicit resonance shapes, evaluators resort to calculating \textit{average} cross sections. These average cross sections are derived from statistical properties of the resonances, such as average level spacings and average resonance widths, rather than their individual values. This approach assumes that these parameters are sufficiently statistical and therefore accurately models the average cross section in the URR.

This formulation provides a convenient analytical form for the average total cross-section. However, it is missing a crucial element of the unresolved resonance region. The underlying statistical nature of the \textit{true} cross section in the unresolved resonance region exhibits significant rapid fluctuations around this analytical average cross section. These rapid fluctuations are responsible for resonance self-shielding. Resonance self-shielding significantly impacts transport equations, and must be accounted for to accurately model neutron transport. The following section characterizes the impact of cross section variance due to resonance fluctuations.

\section{Resonance Self-Shielding and the Correction Factor}
\label{sec:resonance-self-shielding}

Before developing the mathematical formalism of self-shielding, it is important to understand the experimental context in which it arises. The primary experimental observables used to determine cross sections in the URR are neutron transmission and capture yield.

In a transmission experiment, a pulsed neutron beam is directed at a sample, and the ratio of neutrons detected with and without the sample is recorded as a function of neutron energy (determined by time-of-flight). The transmission is related to the total cross section by $T(E) = e^{-n\sigma_t(E)}$, where $n$ is the atomic thickness of the sample in atoms/barn. When the energy resolution of the measurement is insufficient to resolve individual resonances, the measured transmission represents an energy average, $\langle T \rangle$, over many resonance fluctuations within each energy bin.

In a capture experiment, the $\gamma$-rays emitted following neutron capture are detected, and the capture yield $Y(E)$ is measured. For a thin sample, the capture yield is approximately proportional to the capture cross section, $Y(E) \approx n\sigma_\gamma(E)$. For thicker samples, self-shielding in the total cross section also affects the capture yield, since neutrons must first survive transmission through the sample to be captured. This leads to a more complex self-shielding correction for capture measurements.

Both of these measurements are subject to self-shielding when the energy resolution is lower than the resonance spacing, and accurate extraction of average resonance parameters from these measurements requires that the self-shielding be properly accounted for. This is the experimental motivation for the self-shielding formalism developed in this chapter.

It should be noted that self-shielding is also encountered in reactor physics calculations through the Bondarenko method\cite{bondarenko1964}, where self-shielding factors are used to generate effective multi-group cross sections. The present work, however, focuses on self-shielding as it pertains to the analysis of differential cross-section measurements.

\subsection{Definitions}

Consider a neutron transmission measurement over an energy bin $[E_1, E_2]$ that is wide enough to contain a statistically meaningful number of resonances. The energy-averaging operator over this bin is defined as:
\begin{equation}
    \label{eq:energy-average-operator}
    \langle \cdot \rangle = \frac{1}{E_{2} - E_{1}} \int_{E_{1}}^{E_{2}} \cdot\, dE
\end{equation}
The energy-averaged transmission is then:
\begin{equation}
    \label{eq:energy-average-Transmission}
    \langle T \rangle = \left\langle e^{-n \sigma(E)} \right\rangle = \frac{1}{E_{2} - E_{1}} \int_{E_{1}}^{E_{2}} e^{-n \sigma(E)}\, dE
\end{equation}
in which $\sigma(E)$ is the energy-dependent total cross section, and $n$ is the atomic thickness of the sample in atoms/barn. The true average cross section over the same energy bin is:
\begin{equation}
    \label{eq:energy-average-cross-section}
    \left\langle \sigma \right\rangle = \frac{1}{E_2 - E_1} \int_{E_1}^{E_2} \sigma(E)\, dE
\end{equation}

The central question is: what is the relationship between the measurable quantity $\langle T \rangle$ and the desired quantity $\langle\sigma\rangle$? If the cross section were constant over the energy bin, then $\langle T \rangle = e^{-n\langle\sigma\rangle}$ and the average cross section could be trivially extracted. However, in the URR the cross section fluctuates due to unresolved resonances, and the relationship is not so simple.

\subsection{The Self-Shielding Inequality}

Resonance self-shielding is the phenomenon that arises from the non-linear relationship between transmission and cross section. When the cross section fluctuates within an energy bin, the energy-averaged transmission is always greater than the transmission calculated from the average cross section\cite{jeff18}:
\begin{equation}
    \label{eq:self-shielding-definition}
    \langle T \rangle > e^{-n \langle \sigma \rangle}
\end{equation}
This result follows from the mathematical theory of convex functions (Jensen's inequality\cite{jensen1906}), but its physical origin can be understood directly from the structure of the cross section in the URR.

At energies near a resonance peak, the cross section is large and the transmission $T(E) = e^{-n\sigma(E)}$ is exponentially suppressed. At energies between resonances, the cross section is close to the potential scattering value and the transmission is correspondingly high. Because the exponential function suppresses the contribution of high cross-section regions far more aggressively than it enhances the contribution of low cross-section regions, the average transmission is dominated by the inter-resonance valleys where neutrons pass through the sample relatively unimpeded. This preferential transmission through the low cross-section regions is the origin of the term \textit{self-shielding}: the strong resonances effectively shield themselves by depleting the neutron flux at their peak energies\cite{Kidman01061972}\cite{gopalakrishnan}.

The magnitude of this effect depends on two factors: the degree of cross-section fluctuation within the energy bin (i.e., the variance of the cross section), and the sample thickness $n$. For a smooth cross section with no fluctuations, or for an infinitely thin sample, $\langle T \rangle = e^{-n\langle\sigma\rangle}$ exactly, and no self-shielding correction is needed. As either the variance or the sample thickness increases, the inequality in \autoref{eq:self-shielding-definition} becomes more pronounced\cite{gopalakrishnan}.

\subsection{The Transmission Correction Factor}

The self-shielding inequality can be converted into an equality by introducing a correction factor. The average transmission can be rewritten by adding and subtracting $\langle\sigma\rangle$ in the exponent\cite{jeff18}\cite{fitacs}:
\begin{equation}
    \langle T \rangle = \left\langle e^{-n\sigma(E)} \right\rangle = \left\langle e^{-n\langle\sigma\rangle}\, e^{-n(\sigma(E) - \langle\sigma\rangle)} \right\rangle
\end{equation}
Since $\langle\sigma\rangle$ is a constant (it is the average of $\sigma(E)$ over the energy bin, and thus a single number rather than a function of energy), it can be factored out of the averaging operator exactly:
\begin{equation}
    \label{eq:exact-factoring}
    \langle T \rangle = e^{-n\langle\sigma\rangle} \underbrace{\left\langle e^{-n(\sigma(E) - \langle\sigma\rangle)} \right\rangle}_{C_T}
\end{equation}
This factoring is exact and requires no approximation. It naturally separates the average transmission into two components: the transmission of the average cross section, $e^{-n\langle\sigma\rangle}$, and a correction factor $C_T$ that accounts for the effect of cross-section fluctuations. The correction factor is thus defined as:
\begin{equation}
    \label{eq:corrected-average-transmission}
    \langle T \rangle = e^{-n \langle \sigma \rangle} C_T
\end{equation}
where $C_T$ is the transmission self-shielding correction factor:
\begin{equation}
    \label{eq:ct-elegant-form}
    C_T = \left\langle e^{-n (\sigma - \langle \sigma \rangle)} \right\rangle
\end{equation}
or equivalently:
\begin{equation}
    \label{eq:ct-sesh-form}
    C_T = \frac{\langle e^{-n \sigma} \rangle}{ e^{-n \langle \sigma \rangle}}
\end{equation}

The self-shielding inequality established in \autoref{eq:self-shielding-definition} guarantees that $C_T \geq 1$ whenever the cross section fluctuates within the energy bin, with equality only when the cross section is constant (zero variance). But this raises a natural question: what property of the cross section determines the \textit{magnitude} of $C_T$?

\subsection{Connection to Cross-Section Variance}

To understand what drives the magnitude of the self-shielding correction, one can expand the correction factor in a Taylor series\cite{sesh}. Defining the deviation from the mean as $\delta\sigma(E) = \sigma(E) - \langle\sigma\rangle$, the correction factor becomes:
\begin{equation}
    C_T = \left\langle e^{-n\,\delta\sigma(E)} \right\rangle
\end{equation}
Expanding the exponential to second order:
\begin{equation}
    e^{-n\,\delta\sigma(E)} \approx 1 - n\,\delta\sigma(E) + \frac{n^{2}}{2}\,\delta\sigma(E)^2
\end{equation}
Taking the energy average and noting that $\langle \delta\sigma \rangle = 0$ by definition (the mean deviation from the mean is zero):
\begin{equation}
    \label{eq:ct-variance-approx}
    C_T \approx 1 + \frac{n^{2}}{2} \operatorname{Var}(\sigma)
\end{equation}
where $\operatorname{Var}(\sigma) = \langle(\sigma - \langle\sigma\rangle)^2\rangle$ is the variance of the cross section over the energy bin. Substituting back into \autoref{eq:corrected-average-transmission}:
\begin{equation}
    \label{eq:taylor-series-expansion}
    \langle T \rangle \approx e^{-n \langle \sigma \rangle} \left(1 + \frac{n^{2}}{2} \operatorname{Var}(\sigma) \right)
\end{equation}

This result makes explicit that the self-shielding correction is fundamentally driven by the \textit{variance} of the cross section within the energy bin. The correction factor $C_T$ departs from unity in proportion to both the cross-section variance and the square of the sample thickness $n$. When either is zero, there is no self-shielding. When both are large, the correction can be substantial.

This also clarifies why an ``uncorrected'' cross section $\overline{\sigma} = -\frac{1}{n}\ln\langle T\rangle$ extracted na\"{i}vely from a transmission measurement is not the true average cross section: it is systematically lower than $\langle\sigma\rangle$ by an amount that depends on the sample thickness and the cross-section variance. Different sample thicknesses will yield different values of $\overline{\sigma}$, even though the true average cross section is a property of the nucleus alone.

It should be noted that \autoref{eq:corrected-average-transmission} is specifically the \textit{transmission} self-shielding correction. Other experimental observables (e.g., capture yield) have analogous but distinct correction factors, which will be developed in later chapters. In all cases, the underlying goal is the same: to express the measurable average quantity as a function of the true average cross section.

\section{Sampling Cross Sections from Average Parameters}
\label{sec:mc-sampling-from-average-parameters}

The goal of the evaluation process is to determine the \textit{true average cross section}, which is the cross section that would be obtained by averaging the actual energy-dependent cross section $\sigma(E)$ over an energy interval containing a statistically significant number of resonances:
\begin{equation}
    \label{eq:true-average-xs}
    \langle \sigma \rangle_{\text{true}} = \frac{1}{E_2 - E_1} \int_{E_1}^{E_2} \sigma(E)\, dE
\end{equation}
This quantity is distinct from the cross section that would be inferred na\"{i}vely from an average transmission measurement without correcting for self-shielding. The true average cross section is the quantity that, together with its associated variance, fully characterizes the statistical properties of the cross section needed for both self-shielding corrections and reactor calculations.

As previously stated in \autoref{sec:resonance-self-shielding}, in the Unresolved Resonance Region, individual resonances cannot be directly observed due to their close spacing. As such, only average resonance parameters can be obtained from measurements in the URR. While analytical equations, (e.g., \autoref{eq:hauser-feshbach-cross-section}), provide energy-averaged cross sections based on these average parameters, these analytical solutions intrinsically yield only the average behavior. They do not, however, directly provide the variance of the true, fluctuating energy-dependent cross section within the URR. The phenomenon of resonance self-shielding (discussed in \autoref{sec:resonance-self-shielding}) is directly dependent on this variance. However, this only discusses the determination of self-shielding from a known energy differential cross-section, which is not able to be observed.

Since the true energy-differential cross section in the URR is unknown, and its variance cannot be directly calculated from the average parameters alone, Monte Carlo methods provide the necessary means to address this challenge. By statistically sampling individual resonance parameters from their known probability distributions, Monte Carlo sampling allows for the generation of cross-section ``realizations'' which agree with the average analytical cross section and account for the variance in cross section around said average. This section describes the theoretical basis for this approach: the choice of cross-section formalism and the statistical properties of the resonance parameters that govern cross-section fluctuations. The specific implementation details are deferred to the following chapter.

\subsection{The Single-Level Breit-Wigner (SLBW) Equation for Cross Section Reconstruction}
\label{sec:slbw-for-sampling}

The Single-Level Breit-Wigner (SLBW) approximation\cite{t2}\cite{Thomas1955} serves as the fundamental basis for reconstructing cross-section distributions from sampled resonance parameters in the URR. It is well known that the SLBW formalism is insufficient for precise cross-section reconstruction in the resolved resonance region, where inter-level interference effects are significant and multi-level formalisms (such as the Reich-Moore approximation to R-Matrix theory\cite{reich-moore}) are required\cite{Saussure01121976}. However, the situation in the URR is fundamentally different: one does not need to reconstruct the \textit{exact} cross section, but rather to generate cross-section realizations whose \textit{statistical properties} (mean and variance) are consistent with the average resonance parameters. De~Saussure and Perez\cite{Saussure01111973} demonstrated that a set of multi-level resonance parameters can be transformed into an equivalent set of single-level pseudoparameters that reproduce the same average cross section, providing theoretical justification for the SLBW approach in statistical contexts. Fr\"{o}hner\cite{Fröhner01081992} subsequently validated this approach for $^{238}$U in the URR, showing that Monte Carlo sampling of SLBW resonances from average parameters and their statistical distributions accurately reproduced measured thick-sample transmission data and capture self-indication ratios. The SLBW formalism has since been adopted as the standard approach for URR cross-section sampling in major nuclear data processing codes\cite{Dunn01092004}\cite{njoy}.

For a given channel $c$, defined by the quantum numbers $J$, $\ell$, and parity $\pi$, the total SLBW cross section\footnote{This is the cross section associated with forming a compound nucleus in a particular $J,\ell,\pi$ state, irrespective of the exit channel. The incoming channel $c$ is used as shorthand for that state. The conventional total cross section is obtained by summing over all channels.} is determined as a sum over resonances $r$ via:
\begin{equation}
    \label{eq:slbw}
    \sigma_c = \frac{ 4 \pi g_c}{k_c^2} 
        \left\{ \sin^2{\phi_c} + \sum_r \frac{\Gamma_{n,r,c}}{\Gamma_{tot,r,c}}
            \left( \psi_r \cos{2 \phi_{c}} + \chi_r \sin{2\phi_c}
            \right)
        \right\}
\end{equation}
Here $k_c$ is the neutron wave number, $g_c = (2J+1)/[2(2I+1)]$ is the statistical spin factor for a target nucleus with ground-state spin $I$, and $\phi_c$ is the hard-sphere scattering phase shift, which depends on the channel radius and orbital angular momentum $\ell$. Each resonance $r$ is characterized by its energy $E_r$, neutron width $\Gamma_{n,r,c}$, and total width
\begin{equation}
    \label{eq:total-width}
    \Gamma_{tot} = \Gamma_n + \Gamma_\gamma + \Gamma_{n'}
\end{equation}
for non-fissile isotopes, where $\Gamma_\gamma$ is the radiation width and $\Gamma_{n'}$ is the inelastic width. The first term in \autoref{eq:slbw}, $\sin^2\phi_c$, represents potential scattering, which varies smoothly with energy. The terms in the sum contain the resonance contributions and their interference with potential scattering.

The line shape functions $\psi_r$ and $\chi_r$ are the symmetric and asymmetric Doppler-broadened resonance profiles, determined using the Faddeeva function\cite{algo-916}:
\begin{equation}
    \label{eq:faddeeva}
    \psi + i\chi = \frac{\sqrt{\pi}}{2}\theta\, w \left(\frac{\theta x}{2}, \frac{\theta}{2} \right)
\end{equation}
in which $\theta$ is the reduced Doppler width parameter, defined as the ratio of the total resonance width $\Gamma_{tot}$ to the Doppler width $\Delta$:
\begin{equation}
    \theta = \frac{\Gamma_{tot}}{\Delta}, \quad \text{where} \quad \Delta = \sqrt{\frac{4 \kappa T E_r}{A}}
    \label{eq:doppler-width}
\end{equation}
Here $\kappa$ is the Boltzmann constant, $T$ is the thermodynamic temperature of the sample, $E_r$ is the resonance energy, and $A$ is the atomic mass ratio of the target nucleus. The variable $x = 2(E - E_r)/\Gamma_{tot}$ parameterizes the distance from resonance center in units of the total width, and $w(\alpha,\beta)$ is the complex error function:
\begin{equation}
    w(\alpha,\beta) = \frac{i}{\pi} \int_{-\infty}^{\infty} \frac{e^{-t^2}}{\alpha + i\beta - t}\,dt
\end{equation}
Doppler broadening redistributes resonance strength in energy, making resonances shorter and wider while preserving their integrated area. This has a direct impact on self-shielding: at higher temperatures, the reduced peak-to-valley contrast decreases the cross-section variance within an energy bin, which in turn reduces the magnitude of the correction factor $C_T$.

The partial reaction cross sections follow directly. The capture cross section for a given channel is:
\begin{equation}
    \label{eq:slbw-capture}
    \sigma_{c,\gamma} =  \frac{ 4 \pi g_c}{k_c^2}  \sum_r \frac{\Gamma_{n,r,c} \Gamma_{\gamma,r,c}}{\Gamma_{r,tot}^2} \psi_r
\end{equation}
Analogous expressions exist for elastic scattering and fission, but since this work focuses on non-fissile isotopes, the fission cross section is omitted. The total microscopic cross section for a nuclide at a given energy is obtained by summing over all channels $c$.

This SLBW formulation presumes that the individual resonance parameters ($E_r$, $\Gamma_{n,r}$, $\Gamma_{\gamma,r}$) are known for each resonance. In the URR, they are not: only their average values and statistical distributions are available from the resolved resonance region and from nuclear theory. The following section describes how these average parameters are used to reconstruct the cross-section variance that enters the correction factor.

\subsection{Monte Carlo Sampling Average Parameters to Obtain the Correction Factor}
\label{sec:sampling-parameters}

Although the individual resonance parameters in the URR are not experimentally observable, their statistical properties (average level spacings $\langle D \rangle$, average neutron widths $\langle \Gamma_n^0 \rangle$, average radiation widths $\langle \Gamma_\gamma \rangle$, and the probability distributions governing fluctuations about these averages) are known from the resolved resonance region and from nuclear theory. These statistical properties are sufficient to reconstruct the cross-section variance that drives self-shielding, through Monte Carlo sampling.

For each spin-group, a sequence of resonance energies is generated by sampling spacings from the Wigner distribution\cite{Wigner1951}. For each resonance in the sequence, partial widths are sampled from $\chi^2$ distributions\cite{Porter1956}, where the number of degrees of freedom $\nu$ depends on the reaction channel and reflects the number of independent decay amplitudes. The neutron width follows a $\chi^2$ distribution with $\nu = 1$ (the Porter-Thomas distribution\cite{Porter1956}), which is peaked near zero with a long tail. Or in other words, some resonances are very strong while most are very small. This is the most extreme statistical fluctuation among the resonance parameters, and since $\Gamma_n$ appears in the numerator of every SLBW reaction cross section (\autoref{eq:slbw}, \autoref{eq:slbw-capture}), it is the dominant source of cross-section variance. By contrast, the radiation width has a large number of degrees of freedom ($\nu_\gamma \rightarrow \infty$) owing to the high density of $\gamma$-ray final states, and is therefore effectively constant across resonances within a given spin group.

The sampled resonance parameters are substituted into the SLBW formula (\autoref{eq:slbw}) to compute a single realization of the energy-dependent cross section, $\sigma_i$. By repeating this process many times, one builds a statistical ensemble of cross-section realizations $\{\sigma_1, \sigma_2, \ldots, \sigma_N\}$ from which the mean $\langle \sigma \rangle$ and variance $\text{Var}(\sigma)$ can be extracted. Crucially, each sampled cross section also yields a corresponding transmission $T_i = e^{-n\sigma_i}$, and the ensemble average $\langle T \rangle = \frac{1}{N}\sum_i T_i$ provides the theoretical average transmission for a sample of thickness $n$. The correction factor from \autoref{eq:ct-sesh-form} then follows directly:
\begin{equation}
    C_T = \frac{\langle T \rangle}{e^{-n\langle\sigma\rangle}} = \frac{\frac{1}{N}\sum_i e^{-n\sigma_i}}{e^{-n\langle\sigma\rangle}}
\end{equation}
This is the Monte Carlo realization of the correction factor defined in \autoref{eq:ct-sesh-form}: the same quantity, now computed from sampled average parameters rather than from a known energy-differential cross section.

This is the general procedure by which nuclear data processing codes (NJOY\cite{njoy}, FRENDY\cite{frendy}, AMPX\cite{Dunn01092004}\cite{scale2024manual}) simulate the cross-section variance required to account for self-shielding in the URR. Self-shielding due to cross-section variance from resonance fluctuations can therefore be accurately modeled strictly from average parameters. For evaluation purposes, the process is embedded within a fitting loop: a set of trial average resonance parameters produces a theoretical $C_T$, which is compared to experimental data. The parameters are then adjusted and the correction factor recalculated until convergence. The implementation of this procedure within SAMMY (including the conversion from average parameters to channel-specific quantities, the sampling algorithms, and the validation of the resulting cross-section distributions) is described in the following chapter.
