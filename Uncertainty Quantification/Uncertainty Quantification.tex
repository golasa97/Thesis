
\section{Uncertainty Quantification in the URR}

A significant challenge in the evaluation of URR parameters is the determination of realistic parameter uncertainties. The requirement to fit average resonance parameters to energy-averaged measurements, which can combine hundreds or thousands of data points, produces exceptionally small statistical uncertainties. If unmodified, these exceptionally small uncertainties will produce unrealistically small parametric uncertainties in an evaluation, and evaluators must artifically increase parametric uncertainties in order to properly reflect accurate theoretical uncertainties in integral and differential benchmarks\cite{rh103}. This practice, while pragmatic, is obviously not ideal and indicates that a significant source of uncertainty is being systematically missed in URR evaluations.

One of the most significant sources of this missing uncertainty is the model uncertainty associated with self-shielding corrections. Currently, there is no established methodology to tackle the uncertainty quantification associated with self-shielding corrections\cite{templates}. The uncertainty associated with the self-shielding correction is attributable to finite resonance sampling effects. This is a consequence of the fact that any energy bin over which data is averaged contains only a finite number of resonances. This limited sample is not necessarily representative of the true, continuous distributions of the Wigner and Porter-Thomas distributions that dictate neutron resonance statistics. The deviation between the true, finite sample of resonance in an energy bin and the ideal distributions introduces a quantifiable uncertainty that has been unaccounted for in previous evaluations.

The ultimate goal is to propose a method for incorporating this new self-shielding modeling uncertainty component into the evaluation procedure, in order to determine more realistic and physically meaningful uncertainty estimates on the final evaluated parameters. In order to accomplish this, a representative case for $^{181}$Ta will be generated using robust R-Matrix theory, which will then be used to validate that the computationally efficient SLBW equations can accurately reproduce self-shielding uncertainty. These generated uncertainties will be combined with given experimental uncertainties in order to produce parameters with realistic and meaningful uncertainties.


\section{Self-Shielding Correction Limitations}
    
There are two assumptions regarding correcting for self-shielding of an average cross section over some energy range:
\begin{enumerate}
    \item There are a statistical number of resonances in an energy bin.
    \item The energy bin is narrow enough such that any energy dependence can be ignored\cite{wpec-15}.
\end{enumerate}
These assumptions can be valid for sufficiently statistical nuclei (i.e., the self-shielding model uncertainty is negligible), but for some cases, both assumptions cannot be met simultaneously.

The finite resonance effect is a consequence of imperfect statistical sampling. Any real-world energy-bin contains a finite resonance ladder which represents a single statistical realization from the Porter-Thomas and Wigner distributions with a limited number of resonances. As the energy bin increases in size and the number of sampled resonances increases, the sampled distribution will more closely agree with the theoretical distributions. However, the bin size can only increase so much before violating the energy-independence requirement.

\begin{figure}[h]
    \centering
    \includegraphics[width=0.85\linewidth]{Uncertainty Quantification/Figures/ta181-gn0-convergence.png}
    \caption{Convergence of the observed average s-wave reduced neutron width $(\Gamma_n^0)$ as sampled bin-size increases for $^{181}$Ta. The sampled average (blue line) approaches the true average (dashed line) as more resonances are included.}
    \label{fig:ta181-gn0-convergence}
\end{figure}

\autoref{fig:ta181-gn0-convergence} illustrates the statistical convergence of sampled resonances as the energy bin increases over the most recent ENDF-8.1 $^{181}$Ta evaluation\cite{endf-8.1}. One important feature is due to the nature of the Porter-Thomas distribution, which heavily skews towards sampling many small resonances, and few large ones. A small, unrepresentative energy bin might only contain small resonances, and significantly underestimate the average reduced width. Conversely, the inclusion of one large resonance in a small energy bin will significantly over estimate the average reduced width.

The average level spacing determines how wide an energy bin has to be in order to ensure convergence. While the relatively large S-wave $\langle \Gamma_n^0\rangle$ in $^{181}$Ta causes large fluctuations in its observed average, this is counteracted by a small average level spacing (4.1 eV). Consequently, a relatively narrow bin size can include hundreds of resonances, ensuring it satisfies both the statistical and energy-independence assumptions.

\begin{figure}[h]
    \centering
    \includegraphics[width=1.0\linewidth]{Uncertainty Quantification/Figures/zr-xs-binsize-comparisons.png}
    \caption{Total Cross-Section of $^{90}$Zr, grouped over increasingly coarse energy bins.}
    \label{fig:zr90-xs-binsize-comparisons}
\end{figure}

In contrast to the well-behaved nature of $^{181}$Ta, nuclides with large average level spacing, such as $^{90}$Zr, present a significant challenge where the two primary assumptions for self-shielding corrections are in direct conflict.  With an average s-wave level spacing of approximately 8.6 keV, achieving the same level of statistical convergence seen in \autoref{fig:ta181-gn0-convergence} would require an energy bin of roughly 1.7 MeV.

This conflict is illustrated in \autoref{fig:zr90-xs-binsize-comparisons} for which the bin size is varied over the cross-section obtained by Musgrove, et al\cite{Musgrove1977}. The top panel, with a relatively fine 10 keV binning, displays the underlying resonant structure. As the bin width is increased to 100 keV and 500 keV, this structure is progressively averaged out. Not only does the bin size remain insufficiently statistical, the coarsest grid size obscures important energy-dependent features which are essential to the evaluation that will be discussed in \autoref{chap:evaluation}. This means that for proper uncertainty quantification and parameter fitting for $^{90}$Zr, the self-shielding correction uncertainty must be accounted for.

\section{A Monte Carlo Method for Quantifying the Finite Resonance Effect}

To properly quantify the model uncertainty arising from the finite resonance effect, a computational method based on stochastic sampling was developed. The fundamental goal is to simulate the inherent statistical nature of the unresolved resonance region (URR) to determine the variance of the self-shielding correction factor, $C_T$, for a given energy bin and sample thickness. This is achieved by generating thousands of statistically equivalent, physically plausible resonance ladders and observing the resulting distribution of the calculated transmission.

The simulation process follows a sequence of discrete steps for each realization:

\begin{enumerate}
    \item \textbf{Generate Stochastic Resonance Ladders:} The process begins by generating a unique resonance ladder based on a set of energy-dependent average resonance parameters ($\langle D \rangle$, $\langle \Gamma_n^0 \rangle$, etc.). For each realization, individual resonance energies ($E_r$) are sampled from a Wigner distribution, and their corresponding reduced neutron widths ($\Gamma_n^0$) are sampled from a Porter-Thomas distribution. This creates a unique, stochastic, but physically plausible set of resonances that is consistent with the governing nuclear statistics.

    \item \textbf{Calculate Pointwise Cross-Sections:} Each generated resonance ladder is then used to calculate a high-fidelity, pointwise cross-section. The resonance parameters are passed to the R-Matrix calculation engine within SAMMY to produce a theoretical 0K cross-section. This cross-section is subsequently Doppler broadened to the temperature of the corresponding experimental measurement (e.g., 294K), resulting in a realistic, temperature-dependent pointwise cross-section file for each unique realization.

    \item \textbf{Simulate Pointwise Transmission:} From each broadened cross-section realization, the corresponding pointwise transmission, $T(E)$, is calculated using the Beer-Lambert law, $T(E) = e^{-n\sigma(E)}$. This calculation is performed for each sample thickness relevant to the analysis (e.g., 3mm, 6mm, and 12mm for the $^{181}$Ta case).

    \item \textbf{Build Statistical Distributions:} This entire process is repeated to generate thousands of unique realizations. The resulting pointwise transmission files for each realization are then numerically integrated over the predefined energy bins of interest to calculate the energy-averaged transmission, $\langle T \rangle$, and average cross section, $\langle \sigma \rangle$. This creates a large population of $\langle T \rangle$ and $\langle \sigma \rangle$ values for each energy bin, from which a distribution of the self-shielding correction factor, $C_T = \frac{\langle T \rangle}{e^{-n\langle\sigma\rangle}}$, can be determined. The standard deviation of this $C_T$ distribution is the quantified self-shielding model uncertainty for that specific energy bin and sample thickness.
\end{enumerate}

This procedure effectively translates the uncertainty from the finite sampling of resonance parameters into a quantifiable uncertainty on the correction factor that will be applied to an experimental measurement.

\section{Case Study: Analysis of Self-Shielding Uncertainty for $^{181}$Ta}
The methodology was first applied to $^{181}$Ta to investigate the relationship between the number of resonances in an energy bin and the resulting self-shielding model uncertainty. This analysis serves to test the validity of common heuristics used in evaluations and to demonstrate the significance of this uncertainty component. Over 4,100 stochastic realizations of the $^{181}$Ta URR were generated, and the resulting transmissions were calculated for three different sample thicknesses.

\subsection{Impact of Energy Bin Width}
A key question in URR analysis is how wide an energy bin must be to be considered "statistical." A common rule of thumb suggested in the ENDF-6 formats manual is to include approximately 10 resonances per bin. The results of this study, however, indicate that this heuristic may be insufficient, particularly for heavily self-shielded measurements.

\begin{figure}[h]
    \centering
    \includegraphics[width=0.85\linewidth]{Uncertainty Quantification/Figures/ct-error-vs-binwidth.png}
    \caption{The relative uncertainty of the self-shielding correction factor $(\Delta C_T / C_T)$ as a function of the number of resonances included in the energy bin for three different sample thicknesses of $^{181}$Ta. The uncertainty is shown for an energy bin centered at 3.5 keV.}
    \label{fig:ct-error-vs-binwidth}
\end{figure}

\autoref{fig:ct-error-vs-binwidth} shows the relative uncertainty of the correction factor, $\Delta C_T / C_T$, as a function of the number of resonances included in an energy bin centered at 3.5 keV. The uncertainty decreases predictably as the number of resonances increases, as expected from statistical sampling. However, the magnitude of the uncertainty is strongly dependent on the sample thickness. For the thinnest sample (3mm), the uncertainty is relatively small, dropping below 2% with only a few dozen resonances. In contrast, for the heavily self-shielded 12mm sample, the uncertainty remains above 5% even when 50 resonances are included in the bin. At the heuristic of 10 resonances per bin, the model uncertainty is nearly 10% for the 12mm sample, a significant value that cannot be ignored. This demonstrates that a single heuristic for bin size is not universally applicable; the required number of resonances depends heavily on the degree of self-shielding.

\subsection{Comparison with Experimental Uncertainty}
To contextualize the magnitude of this model uncertainty, it was compared to the statistical uncertainty from a $^{181}$Ta transmission measurement performed on a 12mm sample. The model uncertainty, expressed as a relative uncertainty on the transmission ($\Delta T / T$), was calculated from the standard deviation of the simulated $\langle T \rangle$ distributions.

\begin{figure}[h]
    \centering
    \includegraphics[width=0.85\linewidth]{Uncertainty Quantification/Figures/ta181-uq-vs-experimental.png}
    \caption{Comparison of the self-shielding model uncertainty (orange) with the experimental statistical uncertainty (blue) for a 12mm $^{181}$Ta transmission measurement. The model uncertainty is a significant, and in some energy regions dominant, component of the total uncertainty budget.}
    \label{fig:ta181-uq-vs-experimental}
\end{figure}

As shown in \autoref{fig:ta181-uq-vs-experimental}, the calculated self-shielding model uncertainty is a significant, and in some energy regions, dominant component of the total uncertainty budget. At lower energies in the URR (3-5 keV), the model uncertainty is comparable in magnitude to the experimental uncertainty. This finding is critical: it confirms that the finite resonance effect is not a minor correction but a primary source of uncertainty that has been systematically neglected in previous evaluations. Ignoring this component leads directly to the underestimation of parameter uncertainties. Furthermore, this analysis provides a quantitative tool for experimental design, allowing for a trade-off analysis between energy bin width, sample thickness, and counting statistics to optimize future measurements.

\end{document} 
