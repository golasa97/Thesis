\section{Introduction}
\label{sec:eval_introduction}

Zirconium and its isotopes are significant materials in nuclear engineering, frequently used as cladding for nuclear fuel rods and in structural components within reactor cores due to their low thermal neutron absorption cross-section and high resistance to corrosion. Accurate nuclear data for Zirconium isotopes is therefore critical for the safe and efficient design and operation of nuclear reactors. However, existing nuclear data evaluations for Zirconium, particularly in the unresolved resonance region (URR), exhibit several known deficiencies.

A prominent issue is the failure of the current ENDF/B-VIII.1 evaluations to properly model isotopic data. As shown in \autoref{fig:eval_deficiency_structure}, the evaluations are using infinitely dilute cross-sections (File 3 data) derived from natural Zirconium measurements, which are then applied incorrectly to isotopic evaluations\cite{SETH1965306, osti_10132931}. This leads to non-physical artifacts, such as the propagation of resonant structures unique to $^{90}$Zr into the evaluation for $^{91}$Zr, where no such structure is experimentally observed.

\begin{figure}[h!]
    \centering
    \includegraphics[width=0.95\textwidth]{Evaluation/Figures/motivation-discreps-pars.pdf}
    \caption{Comparison of enriched transmission measurements (Musgrove, Green) against calculations based on the ENDF/B-VIII.1 evaluation, highlighting mismatches in the URR.}
    \label{fig:eval_deficiency_structure}
\end{figure}

\begin{figure}[h!]
    \centering
    \includegraphics[width=0.95\textwidth]{Evaluation/Figures/motivation-discreps-experiment.pdf}
    \caption{Comparison of enriched transmission measurements (Musgrove, Green) against calculations based on the ENDF/B-VIII.1 evaluation, highlighting mismatches in the URR that drive the present re-evaluation.}
    \label{fig:eval_deficiency_file2}
\end{figure}

A further deficiency observed in \autoref{fig:eval_deficiency_file2} is the significant deviation between theoretical cross-sections calculated from $^{91}$Zr URR parameters and experiment. Paradoxically, the natural zirconium data appears to be better represented by the current evaluation than the isotopic data. This is a symptom of compensating errors between the individual isotopic evaluations. This situation is unphysical and indicates that the current File 2 parameters do not accurately represent the isotopic behavior when considered individually.  When self-shielded transmission data are converted to infinitely dilute cross sections without accounting for self-shielding, the resulting cross sections are systematically underestimated.

Additionally, a key deficiency is the un-represented intermediate structure in the p-wave neutron cross-section of $^{90}$Zr. This non-statistical effect, often attributed to a doorway state mechanism, is a known feature of the cross-section that is not captured by the statistical treatment of the URR in current evaluations. This omission, combined with the other issues, leads to inaccuracies in reactor physics calculations that are sensitive to the detailed resonance structure.

This work aims to address these deficiencies by performing a new evaluation of the URR for $^{90}$Zr and $^{91}$Zr. The primary objectives of this evaluation are threefold. First, it serves as a practical application of the advanced self-shielding correction and fitting methodologies developed in the preceding chapters. This includes the direct fitting of experimental transmission data without pre-processing, made possible by the integration of the SESH code into SAMMY. Second, the evaluation seeks to produce a more physically accurate and defensible set of average resonance parameters for both $^{90}$Zr and $^{91}$Zr by leveraging these new tools. Finally, a key goal is the explicit characterization of the intermediate structure in $^{90}$Zr and the derivation of more realistic parameter uncertainties by propagating the model uncertainty associated with the self-shielding correction.

The evaluation procedure proceeds in several stages. First, the experimental datasets are prepared: pointwise transmission and cross-section measurements are binned onto an energy grid that preserves the intermediate structure in $^{90}$Zr while maintaining sufficient resonance statistics per bin, and the finite-resonance-effect self-shielding uncertainty is computed for each bin using the Monte Carlo methodology from Chapter~\ref{chap:uncertainty}. This model uncertainty is combined in quadrature with the reported experimental uncertainties to define the total uncertainty used in fitting.

Next, the initial URR parameters are determined. The concurrent RRR evaluation by Siemers et al.\ \cite{GregEvaluationInProgress} provides the statistical properties of the resolved resonances (level spacings, strength functions, radiation widths, and distant-level parameters) that serve as starting values for the URR fit. For $^{90}$Zr, the intermediate structure is characterized via autocorrelation analysis and a doorway state model is fitted to the observed structure, with the distant-level parameters adjusted to accommodate the smooth doorway contributions.

With the data prepared and initial parameters established, the fitAPI code performs a simultaneous $\chi^2$ minimization across all datasets, including enriched transmission, natural transmission, total cross-section, and capture cross-section measurements, fitting the compound nucleus URR parameters while holding the doorway state contributions fixed. The fitted parameters and their covariances are then propagated to evaluated total and capture cross sections on the evaluation energy grid.

Finally, the evaluation is validated by comparing the resulting cross sections directly against ENDF/B-VIII.1 and by computing self-shielded transmission on a fine energy grid for each experimental configuration, which is compared against equivalent MCNP simulations using the ENDF/B-VIII.1 probability tables. This provides a direct, observable-level assessment of whether the new evaluation improves agreement with the experimental data. The following sections detail each of these stages and present the resulting evaluation.

\section{Methodology}
\label{sec:eval_methodology}

\subsection{Data Selection and Fitting Strategy}
\label{ssec:eval_data_selection}

A comprehensive URR evaluation requires both high-quality experimental data and a fitting strategy that leverages the distinct sensitivities of different measurement types. The primary experimental constraints come from transmission and total cross-section measurements, which provide the strongest sensitivity to the URR parameters governing elastic scattering. Both enriched isotopic samples and natural zirconium samples are used in the fit, each contributing complementary information. The key datasets are summarized in \autoref{tab:datasets}.

\begin{table}[h!]
    \centering
    \caption{Experimental datasets selected for the evaluation.}
    \label{tab:datasets}
    \begin{tabular}{l l l l}
        \hline
        \textbf{Author} & \textbf{Isotope} & \textbf{Reaction} & \textbf{Details} \\
        \hline
        Musgrove (1977) \cite{musgrove1977neutron90} & $^{90}$Zr & Transmission & 97.7\% enriched, 0.0827 at/b \\
        Green (1973) \cite{green1973total}           & $^{90}$Zr & Transmission & 97.7\% enriched, 0.0799 at/b \\
        Ohgama (2005) \cite{ohgama2006measurement}   & $^{90}$Zr & Capture XS   & Pointwise data at 0.550 MeV \\
        Macklin (1963) \cite{macklin1963}            & $^{90}$Zr & Capture XS   & Pointwise data at 0.030 MeV \\
        Musgrove (1977) \cite{musgrove1977neutron91} & $^{91}$Zr & Transmission & 89.2\% enriched \\
        Ohgama (2005) \cite{ohgama2005measurement}   & $^{91}$Zr & Capture XS   & [0.02, 0.550] MeV range \\
        Rapp (2019) \cite{rapp2019}                  & $^{\mathrm{nat}}$Zr & Transmission & Natural, 6~cm sample \\
        Rapp (2019) \cite{rapp2019}                  & $^{\mathrm{nat}}$Zr & Transmission & Natural, 10~cm sample \\
        \hline
    \end{tabular}
\end{table}

A significant challenge in this evaluation is the limited availability of experimental capture cross-section data for zirconium isotopes in the URR. While capture data would provide direct sensitivity to the average radiation width $\langle\Gamma_\gamma\rangle$, the scarcity of such measurements, particularly for $^{90}$Zr above 800~keV, means that the radiation width is poorly constrained and must be extrapolated from resolved resonance region (RRR) values.

A key methodological choice in this work is the simultaneous fitting of multiple measurement types within a single optimization. By fitting enriched transmission, natural transmission, total cross-section, and capture cross-section data together, the evaluation exploits the distinct parameter sensitivities of each observable to achieve a more robust parameter determination than would be possible from any single dataset. The physical basis for these complementary sensitivities is discussed in \autoref{ssec:eval_discussion_advances}.

The fitting strategy also employs a hybrid approach depending on the energy region. Below the URR lower boundary (i.e., below 800~keV for $^{90}$Zr and below 220~keV for $^{91}$Zr), the resolved resonance evaluation provides a well-characterized cross section. To ensure continuity at the RRR/URR interface, the URR parameters are constrained to reproduce the pointwise cross section from the RRR evaluation, averaged into 50~keV energy bins. Above the URR lower boundary, where individual resonances can no longer be resolved, the fit is constrained directly by experimental transmission and cross-section measurements. This hybrid strategy ensures that the URR parameters simultaneously satisfy two critical requirements: consistency with the well-characterized resolved region at lower energies, and accurate reproduction of experimental observables throughout the URR.

A full statistical analysis of the resolved resonance region was not performed as part of this work, as a concurrent evaluation has established the statistical quality of the existing resonance parameters \cite{GregEvaluationInProgress}. That analysis confirmed that the resolved resonances for $^{90}$Zr meet the relevant statistical checks up to 800~keV, and for $^{91}$Zr up to 220~keV. Based on these findings, the energy bounds for this URR evaluation were updated from previous evaluations, as summarized in \autoref{tab:urr_bounds}.

\begin{table}[h!]
    \centering
    \caption{URR energy bounds for the zirconium isotopes evaluated in this work.}
    \label{tab:urr_bounds}
    \begin{tabular}{l c c l}
        \hline
        \textbf{Isotope} & \textbf{Lower Bound (keV)} & \textbf{Upper Bound (MeV)} & \textbf{Upper Bound Basis} \\
        \hline
        $^{90}$Zr & 800 & 1.76 & First inelastic level\cite{zr-90-excited} \\
        $^{91}$Zr & 220 & 1.205 & First inelastic level\cite{zr-91-excited} \\
        \hline
    \end{tabular}
\end{table}

In both cases, the lower bound was raised relative to previous evaluations to exclude the well-behaved resolved region, and the upper bound is determined by the energy of the first inelastic scattering level for each isotope. This redefinition allows the URR analysis to focus on the energy ranges where statistical descriptions are most necessary and valid. Importantly, these higher energy regions exhibit more statistical behavior, which improves the validity of the average resonance parameter treatment.


\subsection{Prior Parameter Determination}
\label{ssec:eval_priors}

The URR evaluation requires prior estimates of the average resonance parameters to initialize the fitting procedure. The key parameters of interest include the neutron strength function $S_\ell$, the distant-level parameter $R_\ell^\infty$, the average radiation width $\langle\Gamma_\gamma\rangle$, and the average level spacing $D$ for each partial wave $\ell$. These priors were calculated from the resolved resonance sequences using established statistical methods.

The resulting prior parameters are summarized in \autoref{tab:prior_parameters}. Uncertainties on the prior parameters were propagated from the uncertainties on the individual resolved resonance parameters and the statistical uncertainty inherent in extracting average quantities from a finite resonance sample.

\begin{table}[h!]
    \centering
    \caption{Prior average resonance parameters calculated from resolved resonance region data.}
    \label{tab:prior_parameters}
    \begin{tabular}{l c c c c c}
        \hline
        \textbf{Isotope} & $\ell$ & $S_\ell$ ($\times 10^{-4}$) & $R_\ell^\infty$ & $\langle\Gamma_\gamma\rangle$ (eV) & $D$ (eV) \\
        \hline
        \multirow{3}{*}{$^{90}$Zr} 
            & 0 & $0.617 \pm 0.064$ & $-0.166 \pm 0.063$ & $0.224 \pm 0.016$ & 8337.57 \\
            & 1 & $5.387 \pm 0.274$ & $-0.195 \pm 0.086$ & $0.662 \pm 0.027$ & --- \\
            & 2 & $2.099 \pm 0.208$ & $-0.231 \pm 0.117$ & $0.224 \pm 0.016$ & --- \\
        \hline
        \multirow{3}{*}{$^{91}$Zr}
            & 0 & $0.399 \pm 0.021$ & $-0.189 \pm 0.044$ & $0.165 \pm 0.005$ & 540.96 \\
            & 1 & $5.006 \pm 0.277$ & $-0.226 \pm 0.063$ & $0.237 \pm 0.007$ & --- \\
            & 2 & $0.325 \pm 0.092$ & $-0.274 \pm 0.089$ & $0.165 \pm 0.005$ & --- \\
        \hline
    \end{tabular}
\end{table}

Notably, the p-wave strength function for both isotopes is significantly larger than the s-wave, which is characteristic of the zirconium mass region\cite{atlas}. 

\subsection{Data Binning Strategy}
\label{ssec:eval_binning}

The selection of an appropriate energy binning scheme is a critical step in URR analysis, as it directly impacts the validity of the self-shielding correction factors and the resulting model uncertainty. As discussed in Chapter~\ref{chap:uncertainty}, the ideal energy bin must satisfy two competing requirements: it must be wide enough to contain a statistically meaningful number of resonances, yet narrow enough that the average resonance parameters can be assumed constant across its energy range.

For an isotope like $^{90}$Zr, with an average s-wave level spacing of approximately 8.3~keV, satisfying both conditions is challenging. To achieve a truly statistical sample of, for example, 200 resonances per bin would require a bin width of about 1.6~MeV, far too wide to assume constant average parameters, especially given the presence of intermediate structure. Conversely, bins narrow enough to resolve the intermediate structure contain only a few resonances each.

Rather than using fixed-width bins, the energy bins were chosen to be as wide as possible while still preserving the shape of the intermediate structure in the cross-section data. In practice, this means narrower bins were used in energy regions where the cross section varies rapidly, such as near the doorway state peaks in $^{90}$Zr, and wider bins in regions where the cross section is relatively smooth. A minimum bin width of 40~keV was enforced to ensure a reasonable number of resonances per bin, with a maximum width of 100--200~keV depending on the energy region.

\subsection{Intermediate Structure in $^{90}$Zr}
\label{ssec:eval_intermediate_structure}

A distinctive feature of the $^{90}$Zr neutron cross section is the presence of intermediate structure: broad enhancements spanning several hundred keV that cannot be explained by the statistical fluctuations of individual compound nucleus resonances. This structure has long been attributed to doorway state mechanisms \cite{musgrove1977neutron90, divadeenam1972neutron}. Properly accounting for this intermediate structure is essential for accurate URR evaluation, as it represents a significant departure from the purely statistical behavior assumed by conventional average cross-section formalisms.

Importantly, this intermediate structure phenomenon is unique to $^{90}$Zr among the stable zirconium isotopes. The $^{90}$Zr nucleus has a closed neutron shell at $N = 50$, which creates favorable conditions for the observation of doorway state effects. As discussed by Feshbach et al.\ \cite{feshbach1967doorway}, the reduced density of compound nuclear levels near closed shells means that doorway states are less strongly damped and their resonant structure is more likely to be experimentally observable. In contrast, $^{91}$Zr ($N = 51$) has one neutron outside the closed shell, leading to higher level densities and stronger damping that washes out the intermediate structure. Consequently, the doorway state treatment described in this section applies only to $^{90}$Zr; the $^{91}$Zr evaluation uses conventional energy-independent URR parameters.

To confirm the presence of intermediate structure and characterize its energy scale, an autocorrelation analysis\cite{auto-correlation} was performed on the total cross section derived from the Green (1973) transmission data\cite{green1973total}. The autocorrelation function, defined as
\begin{equation}
    C(\Delta) = \frac{1}{N} \sum_{i=1}^{N} \left[ \sigma(E_i) - \frac{1}{\Delta} \int_{E_i - \Delta/2}^{E_i + \Delta/2} \sigma(E) \, dE \right]^2,
    \label{eq:autocorrelation}
\end{equation}
measures the variance of the cross section relative to a running average computed over a window of width $\Delta$. For a cross section governed purely by statistical compound nucleus fluctuations, $C(\Delta)$ decays smoothly once $\Delta$ exceeds the average resonance spacing $D$. The presence of intermediate structure on a scale $\Gamma^\downarrow$ manifests as a slower decay or plateau in $C(\Delta)$ for $\Delta < \Gamma^\downarrow$, since the running average tracks the intermediate structure rather than averaging it away.

%%%%%%%%%%%%%%%%%%%%%%%%%%%%%%%%%%%%%%%%%%%%%%%%%%%%%%%%%%%%%%%%%%%%%%%%%%%%%%%
% FIGURE PLACEHOLDER - Autocorrelation
%%%%%%%%%%%%%%%%%%%%%%%%%%%%%%%%%%%%%%%%%%%%%%%%%%%%%%%%%%%%%%%%%%%%%%%%%%%%%%%
\begin{figure}[h!]
    \centering
    \begin{subfigure}[b]{0.48\textwidth}
        \centering
        \includegraphics[width=\textwidth]{Evaluation/Figures/Green1973_Scatter.pdf}
        \caption{Green (1973) $^{90}$Zr total cross-section data}
        \label{fig:green-1973-pointwise}
    \end{subfigure}
    \begin{subfigure}[b]{0.48\textwidth}
        \centering
        \includegraphics[width=\textwidth]{Evaluation/Figures/Green1973_CDelta.pdf}
        \caption{Autocorrelation function $C(\Delta)$ (\autoref{eq:autocorrelation})}
        \label{fig:green-1973-autocorrelation}
    \end{subfigure}
    \caption{Green (1973) $^{90}$Zr total cross-section data and corresponding autocorrelation function $C(\Delta)$.}
    \label{fig:eval_autocorrelation}
\end{figure}

The autocorrelation result shown in \autoref{fig:eval_autocorrelation} demonstrates structure persisting at energy scales of several hundred keV, far larger than the $\sim$8~keV average s-wave level spacing for $^{90}$Zr. Specifically, $C(\Delta)$ rises sharply from zero and exhibits several overlapping peaks at $\Delta \approx 150$~keV, $\approx 275$~keV, and $\approx 350$~keV, indicating that multiple distinct intermediate states contribute to the observed cross-section structure. The persistence of these features at energy scales two orders of magnitude larger than the average level spacing confirms that the broad enhancements visible in the cross-section data are not statistical artifacts but represent a coherent, non-resonant modulation of the average cross section. This quantitative confirmation of significant intermediate structure motivated a targeted investigation of its physical origin, which led to the theoretical doorway state framework described below.

A critical distinction must be made between resonant structure that contributes to self-shielding fluctuations and smooth intermediate structure that does not. The intermediate structure observed in $^{90}$Zr represents a smooth, energy-dependent enhancement of the average strength function rather than additional discrete resonances. The autocorrelation analysis confirms that this structure varies on energy scales ($\sim$100--300~keV) that are larger than the averaging intervals used in URR calculations. Consequently, the doorway state contribution appears as a slowly-varying baseline that shifts the mean cross section but does not contribute additional variance to the resonance fluctuations within each energy bin. This distinction has important practical consequences: the doorway state enhancement increases the average cross section (and thus the average self-shielding), but the \emph{fluctuation} in self-shielding, which determines the model uncertainty quantified in \autoref{ssec:eval_ct_uncertainty}, remains governed by the compound nucleus resonance statistics.

A survey of the literature on intermediate structure in the zirconium region identified shell-model calculations by Divadeenam et al.\ \cite{divadeenam1972neutron} that predict multiple two-particle--one-hole (2p-1h) doorway states for the $^{91}$Zr compound nucleus, with both s-wave ($1/2^+$) and p-wave ($1/2^-$) contributions. The broader theoretical framework for such phenomena was established by Feshbach, Kerman, and Lemmer \cite{feshbach1967doorway}, who showed that doorway states, relatively simple configurations through which the incident particle must pass to form the fully equilibrated compound nucleus, can produce resonant structure in the energy-averaged cross section.

Following the formalism of Ref.~\cite{feshbach1967doorway}, the contribution of multiple doorway states to the strength function takes the form:
\begin{equation}
    S_{\ell,\mathrm{ds}}(E) = \alpha \sum_{d} \frac{\Gamma^\uparrow_d \, \Gamma^\downarrow_d}{(E - E_d)^2 + \frac{1}{4}(\Gamma^\uparrow_d + \Gamma^\downarrow_{d})^2},
    \label{eq:doorway_strength}
\end{equation}
where the sum runs over all doorway states $d$ of angular momentum $\ell$, $E_d$ is the doorway energy, $\Gamma^\uparrow_d$ is the escape width (coupling to the entrance channel), $\Gamma^\downarrow_d$ is the spreading (damping) width, and $\alpha$ is a normalization factor. The total strength function for each partial wave is the sum of the compound nucleus (statistical) contribution and the doorway state contribution:
\begin{equation}
    S_\ell(E) = S_\ell^{\mathrm{CN}} + S_{\ell,\mathrm{ds}}(E).
    \label{eq:total_strength}
\end{equation}

Because the doorway state contributions represent smooth enhancements to the cross section, analogous to the role of the distant-level parameter $R_\ell^\infty$ in contributing to the potential scattering background, the $R_\ell^\infty$ parameters were allowed to vary during the fit to properly accommodate the combined smooth contributions from both sources.

The doorway state energies $E_d$ and escape widths $\Gamma^\uparrow_d$ were taken from the shell-model calculations of Divadeenam et al.\ \cite{divadeenam1972neutron}, with small adjustments to the energies (within the theoretical uncertainties) to better match the observed peak positions in the experimental data. The spreading widths $\Gamma^\downarrow_d$ were then determined by fitting to the transmission data while holding the compound nucleus strength functions fixed at their values derived from the resolved resonance region.

Seven p-wave ($1/2^-$) and three s-wave ($1/2^+$) doorway states were included in the model, spanning the energy range from approximately 0.1 to 1.5~MeV. The fitted parameters are summarized in \autoref{tab:doorway_parameters}.

\begin{table}[h!]
    \centering
    \caption{Doorway state parameters for $^{90}$Zr. Doorway energies $E_d$ and escape widths $\Gamma^\uparrow_d$ are based on shell-model calculations for the $^{91}$Zr compound nucleus \cite{divadeenam1972neutron}, with small energy adjustments to match observed peak positions. Spreading widths $\Gamma^\downarrow_d$ were fitted to the Green (1973) transmission data. A global scale factor $\alpha = 2.09 \times 10^{-3}$ converts the doorway contributions to strength function units.}
    \label{tab:doorway_parameters}
    \begin{tabular}{c c c}
        \hline
        $E_d$ (MeV) & $\Gamma^\uparrow_d$ (keV) & $\Gamma^\downarrow_d$ (keV) \\
        \hline
        \multicolumn{3}{c}{\textit{p-wave ($1/2^-$) doorway states}} \\
        \hline
        0.529 & 17.7  & 10.0 \\
        0.700 & 87.0  & 30.1 \\
        0.860 & 32.9  & 10.0 \\
        1.127 & 5.1   & 77.9 \\
        1.225 & 0.04  & 16.7 \\
        1.315 & 2.7   & 31.2 \\
        1.430 & 9.6   & 600.0 \\
        \hline
        \multicolumn{3}{c}{\textit{s-wave ($1/2^+$) doorway states}} \\
        \hline
        0.263 & 3.6  & 241.4 \\
        0.791 & 6.0  & 43.7 \\
        1.239 & 0.2  & 600.0 \\
        \hline
    \end{tabular}
\end{table}

The dominant contributions come from the p-wave doorways at 0.529, 0.700, and 0.860~MeV, which together account for the broad cross-section enhancement observed in the experimental data. States with fitted spreading widths at the upper bound (600~keV) are either very broad or poorly constrained by the available data. The adjusted distant-level parameters from this doorway fit were $R^\infty_0 = 0.162$ and $R^\infty_1 = -0.095$; these values served as the initial parameters for the subsequent compound nucleus parameter fit described in \autoref{ssec:eval_fitting}.

\begin{figure}[h!]
    \centering
    \includegraphics[width=0.95\textwidth]{Evaluation/Figures/zr90_doorway_fit_totalxs.pdf}
    \caption{Total cross section comparison for $^{90}$Zr showing the fitted doorway state model (solid red) versus a compound-nucleus-only model (dashed blue) against the Green (1973) transmission-derived data. The doorway model captures the broad enhancements near 0.55, 0.7, and 0.85~MeV that are absent in the purely statistical treatment. Vertical dotted lines indicate s-wave doorway state energies, and solid gray lines show the p-wave doorway state energies.}
    \label{fig:eval_doorway_totalxs}
\end{figure}

\begin{figure}[h!]
    \centering
    \includegraphics[width=0.95\textwidth]{Evaluation/Figures/zr90_doorway_fit_total_strengths.pdf}
    \caption{Energy-dependent strength functions for $^{90}$Zr including doorway state contributions. The total s-wave strength function $S_0(E)$ (blue) and p-wave strength function $S_1(E)$ (orange) are shown as solid lines, while the energy-independent compound nucleus values are shown as dashed lines. The p-wave doorway contributions produce enhancements of up to a factor of three above the compound nucleus baseline, dominating the cross-section structure observed in \autoref{fig:eval_doorway_totalxs}.}
    \label{fig:eval_doorway_strengths}
\end{figure}

The impact of including the intermediate structure model is illustrated in Figures~\ref{fig:eval_doorway_totalxs} and \ref{fig:eval_doorway_strengths}. The doorway model dramatically improves agreement with the experimental data compared to the compound-nucleus-only treatment. The strength function plot reveals that the p-wave doorway contributions produce enhancements of up to a factor of three above the compound nucleus baseline near 0.55 and 0.7~MeV, which directly correspond to the cross-section peaks observed in the data. This demonstrates that the intermediate structure physics is essential for accurate representation of the $^{90}$Zr cross section.


\subsection{Self-Shielding Uncertainty from Finite-Resonance Effects}
\label{ssec:eval_ct_uncertainty}

The finite-resonance self-shielding uncertainty developed in Chapter~\ref{chap:uncertainty} is applied here to each energy bin of the Zr transmission datasets. \autoref{fig:eval_zr90_musgrove_ctunc_vs_measurement_unc} compares the magnitude of this model uncertainty to the reported experimental uncertainty for the Musgrove (1977) $^{90}$Zr dataset, showing that the model uncertainty becomes comparable to or exceeds the experimental uncertainty at higher energies where fewer resonances contribute to each bin.

\begin{figure}[h!]
    \centering
    \includegraphics[width=0.85\textwidth]{Evaluation/Figures/zr90-musgrove-ctunc-vs-measurement-unc-2.pdf}
    \caption{Comparison of the estimated self-shielding correction (model) uncertainty to the experimental (reported) uncertainty for the Musgrove (1977) 2~cm $^{90}$Zr transmission dataset. At higher energies within the URR, the model uncertainty exceeds the experimental uncertainty.}
    \label{fig:eval_zr90_musgrove_ctunc_vs_measurement_unc}
\end{figure}

The total uncertainty on each binned data point combines the experimental and model contributions in quadrature:
\begin{equation}
    \Delta_i = \sqrt{\Delta_{i,\mathrm{stats}}^2 + \Delta_{i,\mathrm{model}}^2},
    \label{eq:total_uncertainty}
\end{equation}
where the model uncertainty component is propagated from the $C_T$ variance as
\begin{equation}
    \Delta_{\mathrm{model},i} = \Delta C_{T,i} \cdot e^{-n\langle\sigma\rangle}.
    \label{eq:model_uncertainty}
\end{equation}
This combined uncertainty is used in the $\chi^2$ minimization during fitting, ensuring that the fitted parameters and their covariances properly reflect both experimental limitations and model uncertainty.


\subsection{Fitting Procedure}
\label{ssec:eval_fitting}

SAMMY's original URR fitting capability is built around the Bayesian $M+W$ update framework \cite{sammy}, which was designed to adjust an initial parameter set using a linearized relationship between fitted parameters and observables. In practice, URR observables such as transmission and self-shielded average cross sections are strongly nonlinear, and the workflow developed in this dissertation introduces additional nonlinearity through the dependence of the correction factor $C_T(\mathbf{p})$ on the fitted parameters. As a result, the linearization that underlies the $M+W$ update can become a poor local approximation when the model response changes rapidly with the parameters or when the initial guess is not already close to the solution. A second practical limitation is that the original URR fitter relies on parameter-type-dependent transforms to an internal fitting space (``$u$-space'') to improve numerical behavior and enforce positivity constraints. While effective for the original URR model, this tight coupling between the minimizer and the meaning of each parameter reduces flexibility: extending the model (e.g., adding intermediate-structure terms or new parameter groupings) requires extending the fitter with new special-case logic.

For these reasons, this work uses a direct $\chi^2$-minimization approach in physical parameter space for URR evaluations. The goal is not to change the Bayesian interpretation of parameter uncertainties, but to provide a fitter that behaves reliably for nonlinear observables and remains agnostic to the specific URR parameterization used to calculate the theoretical observables. The parameter vector is updated by minimizing the total $\chi^2$ over all selected datasets (multiple thicknesses and isotopes where applicable):
\begin{equation}
    \chi^2 = \sum_{\text{datasets}} \sum_{i} \frac{(T_i^{\text{calc}} - T_i^{\text{exp}})^2}{\Delta_i^2},
    \label{eq:chi2}
\end{equation}
where $T_i^{\text{calc}}$ is the calculated transmission (including self-shielding corrections), $T_i^{\text{exp}}$ is the experimental transmission, and $\Delta_i$ is the total uncertainty from \autoref{eq:total_uncertainty}.

At each iteration, the theoretical values and their respective derivatives are recomputed using the current parameters, including recomputation of self-shielding corrections. Convergence is assessed using the reduction in $\chi^2$ and the stability of parameter updates. The final parameter covariance is taken from the local curvature information (inverse Hessian) at the minimum. Implementation details and the full optimization algorithm are provided in \autoref{app:gradient-descent}.

To propagate the fitted parameter covariances to cross-section covariances, a Monte Carlo sampling approach is employed. Parameter vectors are sampled from the multivariate normal distribution defined by the fitted parameters and their covariance matrix; for each sampled parameter set, the corresponding cross sections are computed over the energy grid of interest; and the covariance of the resulting cross-section ensemble is computed. This approach naturally captures the nonlinear relationship between parameters and cross sections and produces energy-energy correlation matrices that reflect the dominant parameter couplings.

\section{Results}
\label{sec:eval_results}

This section presents the results of the URR evaluation in three parts: first, the fitting performance against the experimental transmission data; second, the final evaluated cross sections and covariances for $^{90}$Zr; and third, the corresponding results for $^{91}$Zr.


\subsection{Transmission Fitting Performance}
\label{ssec:eval_results_fitting}

The quality of the URR parameter fit is assessed by comparing the calculated transmission to the experimental data for each dataset. Figures~\ref{fig:eval_trans_green_zr90}--\ref{fig:eval_trans_musgrove_zr91} show the fitted transmission and relative residuals for the enriched $^{90}$Zr and $^{91}$Zr datasets that were used to constrain the evaluation.

\begin{figure}[h!]
    \centering
    \includegraphics[width=0.95\textwidth]{Evaluation/Figures/transmission_green_zr90.pdf}
    \caption{Fitted transmission for the Green (1973) enriched $^{90}$Zr dataset (0.08~at/b sample thickness). Top: Comparison of calculated (red) and experimental (blue) transmission. Bottom: Relative residuals $(C-E)/E$ with $\pm 5\%$ reference lines.}
    \label{fig:eval_trans_green_zr90}
\end{figure}

The Green (1973) dataset (\autoref{fig:eval_trans_green_zr90}) spans the full URR energy range and provides the primary constraint on the $^{90}$Zr parameters. The fitted model reproduces the experimental transmission within approximately 1--3\% throughout the URR. The residuals show a slight systematic negative bias, indicating that the calculated transmission is consistently lower than experiment by 1--2\%. This bias is most pronounced at the lowest URR energies near 0.85~MeV, where residuals reach approximately $-3\%$.

\begin{figure}[h!]
    \centering
    \includegraphics[width=0.95\textwidth]{Evaluation/Figures/transmission_musgrove_zr90.pdf}
    \caption{Fitted transmission for the Musgrove (1977) enriched $^{90}$Zr dataset (0.083~at/b sample thickness). Top: Comparison of calculated and experimental transmission. Bottom: Relative residuals.}
    \label{fig:eval_trans_musgrove_zr90}
\end{figure}

The Musgrove (1977) $^{90}$Zr dataset (\autoref{fig:eval_trans_musgrove_zr90}) covers a narrower energy range (0.85--1.45~MeV) but provides complementary information due to its slightly different sample thickness. The fit quality is excellent, with residuals consistently within $\pm 1\%$ and minimal systematic bias. The improved agreement compared to the Green dataset may reflect differences in experimental systematics or the narrower energy range that focuses on the region where the URR model is most applicable.

\begin{figure}[h!]
    \centering
    \includegraphics[width=0.95\textwidth]{Evaluation/Figures/transmission_musgrove_zr91.pdf}
    \caption{Fitted transmission for the Musgrove (1977) enriched $^{91}$Zr dataset (0.064~at/b sample thickness). Top: Comparison of calculated and experimental transmission. Bottom: Relative residuals.}
    \label{fig:eval_trans_musgrove_zr91}
\end{figure}

The $^{91}$Zr fit (\autoref{fig:eval_trans_musgrove_zr91}) demonstrates good agreement throughout the URR, with residuals typically within $\pm 2\%$. Unlike $^{90}$Zr, no intermediate structure is present in $^{91}$Zr, and the transmission increases smoothly with energy as expected for a conventional URR cross section. A slight positive bias is observed in the 0.35--0.45~MeV range, but this lies below the defined URR boundary (220~keV) and may reflect RRR contributions. The absence of systematic residual structure within the URR confirms that energy-independent average resonance parameters adequately describe the $^{91}$Zr cross section.

The natural zirconium transmission measurements from Rapp (2019) were included in the fitting procedure alongside the enriched samples. These thick natural samples experience strong self-shielding, providing valuable discrimination between smooth and resonant contributions to the cross section. Because natural zirconium contains contributions from all stable isotopes, including both $^{90}$Zr and $^{91}$Zr, the natural data simultaneously constrain the parameters of both isotopes and enforce consistency of the combined isotopic cross sections with direct measurement.

\begin{figure}[h!]
    \centering
    \includegraphics[width=0.95\textwidth]{Evaluation/Figures/transmission_rapp_6cm.pdf}
    \caption{Fitted transmission for the Rapp (2019) natural zirconium dataset (6~cm sample). Residuals of 5--10\% in the 0.8--1.0~MeV region likely reflect limitations in the other zirconium isotopes ($^{92,94,96}$Zr), which were not re-evaluated.}
    \label{fig:eval_trans_rapp_6cm}
\end{figure}

\begin{figure}[h!]
    \centering
    \includegraphics[width=0.95\textwidth]{Evaluation/Figures/transmission_rapp_10cm.pdf}
    \caption{Fitted transmission for the Rapp (2019) natural zirconium dataset (10~cm sample). The heavily self-shielded thick sample shows larger residuals, particularly in the 0.8--1.4~MeV region where $^{90}$Zr intermediate structure contributes.}
    \label{fig:eval_trans_rapp_10cm}
\end{figure}

The 6~cm sample (\autoref{fig:eval_trans_rapp_6cm}) shows generally good agreement, with residuals within $\pm 5\%$ for most energy bins. A notable outlier near 0.85~MeV shows approximately $-10\%$ deviation, occurring precisely where the $^{90}$Zr intermediate structure peak contributes most strongly to the natural zirconium cross section. This residual likely reflects limitations in the other stable zirconium isotopes ($^{92,94,96}$Zr), which retain their ENDF/B-VIII.1 parameters and were not re-evaluated in this work.

The 10~cm sample (\autoref{fig:eval_trans_rapp_10cm}) exhibits larger residuals, particularly in the 0.5--1.4~MeV region where deviations of 5--15\% are observed. The heavily self-shielded thick sample amplifies any deficiencies in the cross-section model. The pattern of residuals, predominantly negative at lower energies and improving toward higher energies, suggests that the combined natural zirconium cross section is overpredicted in the intermediate structure region. This may reflect limitations in the other stable zirconium isotopes ($^{92,94,96}$Zr), which were not re-evaluated in this work and retain their ENDF/B-VIII.1 parameters, or indicate that the $^{90}$Zr doorway state contributions are slightly overestimated when propagated to natural mixtures.

%\begin{figure}[h!]
%    \centering
%    \includegraphics[width=0.95\textwidth]{Evaluation/Figures/residual_by_energy.pdf}
%    \caption{Average absolute residual magnitude as a function of energy, aggregated across all transmission datasets. The peak near 0.8--1.0~MeV reflects the challenging intermediate structure region in $^{90}$Zr.}
%    \label{fig:eval_residual_summary}
%\end{figure}
%
%\autoref{fig:eval_residual_summary} summarizes the fitting performance across all transmission datasets by showing the average absolute residual magnitude as a function of energy. The residuals are largest ($\sim$4--5\%) in the 0.4--0.6~MeV region, which lies below the $^{90}$Zr URR boundary and includes contributions from the tail of the resolved resonance region. Within the $^{90}$Zr URR (0.8--1.78~MeV), residuals are generally 1--2\%, indicating good model performance with the doorway state treatment. The residuals decrease toward higher energies, reaching approximately 1\% above 1.5~MeV where the cross section is dominated by conventional URR behavior.

The overall fit quality demonstrates that the evaluation methodology, including the intermediate structure model for $^{90}$Zr and the self-shielding uncertainty quantification, produces physically reasonable parameters that reproduce the experimental observables within their uncertainties.


\subsection{$^{90}$Zr Evaluated Cross Sections}
\label{ssec:eval_results_zr90}

The final fitted URR parameters for $^{90}$Zr are presented in \autoref{tab:zr90_final_parameters}. These parameters, combined with the doorway state contributions described in \autoref{ssec:eval_intermediate_structure}, fully specify the evaluated cross sections in the URR.

\begin{table}[h!]
    \centering
    \caption{Initial and final compound nucleus URR parameters for $^{90}$Zr. Initial values for $S_\ell$ and $\langle\Gamma_\gamma\rangle$ are the RRR-derived priors; initial values for $R_\ell^\infty$ are the adjusted values from the doorway state fit (\autoref{ssec:eval_intermediate_structure}). Final values are the converged result. The doorway state parameters are given separately in \autoref{tab:doorway_parameters}. The $\ell = 2$ radiation width is linked to $\ell = 0$.}
    \label{tab:zr90_final_parameters}
    \begin{tabular}{c cc cc cc}
        \hline
        & \multicolumn{2}{c}{$S_\ell^{\mathrm{CN}}$ ($\times 10^{-4}$)} & \multicolumn{2}{c}{$R_\ell^\infty$} & \multicolumn{2}{c}{$\langle\Gamma_\gamma\rangle$ (eV)} \\
        $\ell$ & Initial & Final & Initial & Final & Initial & Final \\
        \hline
        0 & 0.616 & 0.657 & $0.162$ & $0.254$ & 0.224 & 0.177 \\
        1 & 5.251 & 3.714 & $-0.095$ & $-0.009$ & 0.430 & 0.302 \\
        2 & 2.099 & 0.942 & $-0.231$ & $-0.146$ & --- & --- \\
        \hline
    \end{tabular}
\end{table}

The distant-level parameters $R_\ell^\infty$ were pre-adjusted during the doorway state fit (\autoref{ssec:eval_intermediate_structure}) to accommodate the smooth baseline contribution from the doorway states, and those adjusted values served as the initial parameters for the final fit. The fitAPI further refined them, with $R_0^\infty$ shifting from $0.162$ to $0.254$ and $R_1^\infty$ from $-0.095$ to $-0.009$. The compound nucleus p-wave strength function ($S_1^{\mathrm{CN}} = 3.71 \times 10^{-4}$) is substantially lower than the RRR-derived prior ($5.39 \times 10^{-4}$) because the doorway states now explicitly account for much of the observed p-wave strength. The d-wave strength function is also reduced significantly ($S_2^{\mathrm{CN}} = 0.94 \times 10^{-4}$ from a prior of $2.10 \times 10^{-4}$), indicating that the inclusion of the natural zirconium data in the fit provides additional constraint on the higher partial waves.

\begin{figure}[h!]
    \centering
    \includegraphics[width=0.95\textwidth]{Evaluation/Figures/totxs_zr90_linlin.pdf}
    \caption{Total cross section for $^{90}$Zr comparing the RPI evaluation (red) to ENDF/B-VIII.1 (blue, with uncertainty band). The RPI evaluation captures the intermediate structure enhancement near 0.5--0.8~MeV from the p-wave doorway states. The binned RRR cross section from the concurrent resolved resonance evaluation (gray points) demonstrates continuity at the RRR/URR interface near 0.8~MeV. Above the URR boundary, the RPI evaluation is consistently $\sim$5\% higher than ENDF/B-VIII.1.}
    \label{fig:eval_zr90_totxs_comparison}
\end{figure}

The evaluated total cross section for $^{90}$Zr is compared to ENDF/B-VIII.1 in \autoref{fig:eval_zr90_totxs_comparison}. The most striking feature is the intermediate structure captured by the RPI evaluation in the pre-URR region (0.5--0.8~MeV), where the p-wave doorway states produce broad enhancements that are entirely absent in the featureless ENDF/B-VIII.1 cross section. Above the URR boundary at 0.8~MeV, the RPI evaluation produces cross sections approximately 5\% higher than ENDF/B-VIII.1 across the full energy range. This systematic offset is consistent with the hypothesis that the ENDF evaluation was derived from self-shielded natural measurements without proper correction, resulting in a systematic underestimate of the true cross section. The binned resolved resonance cross section from the concurrent RRR evaluation demonstrates smooth continuity at the RRR/URR interface.

\begin{figure}[h!]
    \centering
    \includegraphics[width=0.95\textwidth]{Evaluation/Figures/zr90_totxs_uncertainty_correlation.pdf}
    \caption{Uncertainty and correlation for the $^{90}$Zr total cross section in the URR. Left: Relative uncertainty as a function of energy (0.25--0.50\%). Right: Energy-energy correlation matrix showing cross-shaped regions of reduced correlation at specific energies corresponding to energy bin boundaries where the doorway state contributions shift the local parameter sensitivity.}
    \label{fig:eval_zr90_totxs_covariance}
\end{figure}

The uncertainty and correlation structure of the total cross section are shown in \autoref{fig:eval_zr90_totxs_covariance}. The relative uncertainty ranges from approximately 0.25\% near the lower URR boundary to 0.50\% at higher energies. These uncertainties properly include the self-shielding model uncertainty propagated through the fitting procedure. The correlation matrix reveals a distinctive cross-shaped pattern of reduced correlations at specific energies within the URR. These features occur at energies where the doorway state contributions shift the balance between the smooth and resonant components of the cross section, producing local changes in parameter sensitivity that partially decorrelate adjacent energy bins. Because the doorway state parameters were held fixed during the compound nucleus parameter fit, the uncertainty in energy regions dominated by doorway contributions is artificially suppressed. This represents a limitation of the current approach: a complete uncertainty quantification would require propagating the doorway state parameter uncertainties as well, which is identified as an area for future work.

\begin{figure}[h!]
    \centering
    \includegraphics[width=0.95\textwidth]{Evaluation/Figures/capxs_zr90_loglog.pdf}
    \caption{Capture cross section for $^{90}$Zr comparing the RPI evaluation (red) to ENDF/B-VIII.1 (blue, with uncertainty band). The binned RRR cross section (brown points) and Ohgama (2005) data (green) are also shown. No experimental capture data exist in the defined URR above 0.8~MeV.}
    \label{fig:eval_zr90_capxs_comparison}
\end{figure}

The capture cross section (\autoref{fig:eval_zr90_capxs_comparison}) presents a significant challenge due to the limited experimental data for $^{90}$Zr capture. The Ohgama (2005) measurement at 0.55~MeV provides a single constraint near the URR boundary, and the binned RRR evaluation constrains the lower energies, but no experimental capture data exist above 0.8~MeV. The RPI evaluation predicts lower capture cross sections than ENDF/B-VIII.1 throughout the URR, with the two evaluations diverging increasingly at higher energies where ENDF/B-VIII.1 shows an upturn that is not supported by extrapolation from the resolved region. The absence of experimental constraint in the URR proper means that the capture cross section carries substantial uncertainty and represents an area where future measurements would significantly improve the evaluation.

\begin{figure}[h!]
    \centering
    \includegraphics[width=0.95\textwidth]{Evaluation/Figures/zr90_capxs_uncertainty_correlation.pdf}
    \caption{Uncertainty and correlation for the $^{90}$Zr capture cross section. Left: Relative uncertainty (0.40--0.56\%). Right: Energy-energy correlation matrix showing near-unity correlations (0.995--1.000) throughout the URR.}
    \label{fig:eval_zr90_capxs_covariance}
\end{figure}

The capture cross-section covariance (\autoref{fig:eval_zr90_capxs_covariance}) shows relative uncertainties of approximately 0.40--0.56\%, increasing toward higher energies. The correlation matrix exhibits near-unity correlations (0.995--1.000) across all energy pairs, reflecting the fact that the capture cross section throughout the URR is controlled almost entirely by a single parameter, the radiation width $\langle\Gamma_\gamma\rangle$. This near-perfect correlation means that the capture cross section can shift up or down as a unit but has very little freedom to change shape. The small uncertainties should be interpreted with caution: because the doorway parameters were held fixed and no experimental capture data constrain the URR directly, the reported uncertainties reflect only the compound nucleus parameter contributions and represent a lower bound on the true uncertainty.


\subsection{$^{91}$Zr Evaluated Cross Sections}
\label{ssec:eval_results_zr91}

The final fitted URR parameters for $^{91}$Zr are presented in \autoref{tab:zr91_final_parameters}. Unlike $^{90}$Zr, no intermediate structure model is required, and conventional energy-independent URR parameters adequately describe the cross section.

\begin{table}[h!]
    \centering
    \caption{Initial and final URR parameters for $^{91}$Zr. Initial values are the starting parameters provided to the fitting code; final values are the converged result. The $\ell = 2$ radiation width is linked to $\ell = 0$.}
    \label{tab:zr91_final_parameters}
    \begin{tabular}{c cc cc cc}
        \hline
        & \multicolumn{2}{c}{$S_\ell$ ($\times 10^{-4}$)} & \multicolumn{2}{c}{$R_\ell^\infty$} & \multicolumn{2}{c}{$\langle\Gamma_\gamma\rangle$ (eV)} \\
        $\ell$ & Initial & Final & Initial & Final & Initial & Final \\
        \hline
        0 & 0.399 & 0.559 & $-0.189$ & $-0.167$ & 0.165 & 0.127 \\
        1 & 5.006 & 5.765 & $-0.226$ & $-0.212$ & 0.237 & 0.220 \\
        2 & 0.325 & 0.265 & $-0.274$ & $-0.264$ & --- & --- \\
        \hline
    \end{tabular}
\end{table}

The fitted parameters show some notable departures from the prior values. The s-wave strength function increases from $0.40$ to $0.56 \times 10^{-4}$, and the p-wave strength function increases modestly from $5.01$ to $5.77 \times 10^{-4}$. The s-wave radiation width decreases from 0.165~eV to 0.127~eV, reflecting the additional constraint provided by the natural zirconium transmission data, which are sensitive to the combined isotopic cross sections and help discriminate between elastic and capture contributions. The distant-level parameters remain close to their prior values, indicating that the smooth background cross-section component is well constrained by the resolved resonance region.

\begin{figure}[h!]
    \centering
    \includegraphics[width=0.95\textwidth]{Evaluation/Figures/totxs_zr91_linlin.pdf}
    \caption{Total cross section for $^{91}$Zr comparing the RPI evaluation (red) to ENDF/B-VIII.1 (blue, with uncertainty band). Unlike $^{90}$Zr, no intermediate structure is present, and the RPI cross section decreases smoothly with energy. The ENDF/B-VIII.1 evaluation exhibits pronounced unphysical fluctuations, particularly near 0.15--0.8~MeV, that arise from the propagation of natural zirconium structure into the isotopic evaluation. The binned RRR cross section (gray points) demonstrates continuity at the 220~keV URR boundary.}
    \label{fig:eval_zr91_totxs_comparison}
\end{figure}

The $^{91}$Zr total cross section (\autoref{fig:eval_zr91_totxs_comparison}) exhibits the smooth, monotonically decreasing behavior expected for a conventional URR cross section without doorway state contributions. The contrast with ENDF/B-VIII.1 is striking: the current library shows pronounced oscillatory structure throughout the 0.15--0.8~MeV range, with the ENDF uncertainty band encompassing a wide spread that reflects the unphysical fluctuations. These artifacts are not supported by the enriched $^{91}$Zr transmission data and likely arise from the use of natural zirconium measurements that were not properly deconvolved to extract isotopic cross sections. The RPI evaluation eliminates these artifacts entirely, producing a smooth cross section that is well constrained by the binned RRR data at the 220~keV boundary.

\begin{figure}[h!]
    \centering
    \includegraphics[width=0.95\textwidth]{Evaluation/Figures/zr91_totxs_uncertainty_correlation.pdf}
    \caption{Uncertainty and correlation for the $^{91}$Zr total cross section. Left: Relative uncertainty as a function of energy (0.25--0.60\%). Right: Energy-energy correlation matrix showing high correlations above 0.5~MeV with reduced correlations ($\sim$0.2--0.4) at the lowest URR energies.}
    \label{fig:eval_zr91_totxs_covariance}
\end{figure}

The covariance structure (\autoref{fig:eval_zr91_totxs_covariance}) shows relative uncertainties of 0.25--0.60\%, increasing toward higher energies where the experimental constraints become sparser. The correlation matrix exhibits high correlations (0.8--1.0) above approximately 0.5~MeV, reflecting the dominance of a few common parameters, primarily the p-wave strength function, in determining the cross-section magnitude in that region. However, the correlations decrease substantially (to $\sim$0.2--0.4) at the lowest URR energies near 0.3~MeV. This reduced correlation near the URR lower boundary reflects the transition from the resolved resonance region to the statistical URR treatment: the cross section in this transition region is sensitive to both the RRR tail and the URR average parameters, producing partial decorrelation from the higher-energy behavior. This correlation structure contrasts with $^{90}$Zr, where the cross-shaped decorrelation pattern arose from the doorway state contributions.

\begin{figure}[h!]
    \centering
    \includegraphics[width=0.95\textwidth]{Evaluation/Figures/capxs_zr91_loglog.pdf}
    \caption{Capture cross section for $^{91}$Zr comparing the RPI evaluation (red) to ENDF/B-VIII.1 (blue, with uncertainty band). The binned RRR cross section (brown points) and Ohgama (2005) data (green) provide experimental constraints extending into the lower portion of the URR.}
    \label{fig:eval_zr91_capxs_comparison}
\end{figure}

The $^{91}$Zr capture cross section (\autoref{fig:eval_zr91_capxs_comparison}) benefits from more experimental constraint than $^{90}$Zr, with capture data from Ohgama (2005) extending into the lower portion of the URR and the binned RRR evaluation providing dense coverage below 220~keV. The RPI evaluation predicts systematically lower capture cross sections than ENDF/B-VIII.1, with the two evaluations diverging at higher energies where ENDF/B-VIII.1 shows an upturn. This difference reflects the updated radiation width values derived from the concurrent resolved resonance evaluation and the additional constraint provided by the natural zirconium transmission data in the simultaneous fit.

\begin{figure}[h!]
    \centering
    \includegraphics[width=0.95\textwidth]{Evaluation/Figures/zr91_capxs_uncertainty_correlation.pdf}
    \caption{Uncertainty and correlation for the $^{91}$Zr capture cross section. Left: Relative uncertainty (0.15--0.38\%). Right: Energy-energy correlation matrix showing high correlations above 0.5~MeV with reduced correlations ($\sim$0.4) at the lowest URR energies.}
    \label{fig:eval_zr91_capxs_covariance}
\end{figure}

The capture cross-section covariance (\autoref{fig:eval_zr91_capxs_covariance}) shows relative uncertainties of 0.15--0.38\%, with a characteristic shape: the uncertainty is small ($\sim$0.15\%) near the lower URR boundary where the RRR constraint is strongest, increases steadily through the URR, and shows a sharp upturn near the upper boundary at 1.2~MeV where the experimental constraints are weakest. The correlation matrix shows high correlations (0.8--1.0) above 0.5~MeV but decreases to approximately 0.4 at the lowest URR energies, mirroring the pattern observed in the total cross section. This reduced correlation near the URR lower boundary again reflects the transition from resolved resonance to statistical treatment, where the capture cross section is sensitive to both the RRR tail and the URR average parameters.



\subsection{Fit Quality Summary}
\label{ssec:eval_fit_summary}

The quantitative fit quality for all datasets is summarized in \autoref{tab:chi2_summary}.

\begin{table}[h!]
    \centering
    \caption{Summary of fit quality ($\chi^2/N$) for all transmission datasets, computed using experimental (statistical) uncertainties only.}
    \label{tab:chi2_summary}
    \begin{tabular}{l l c c}
        \hline
        \textbf{Dataset} & \textbf{Type} & \textbf{$N$} & \textbf{$\chi^2/N$} \\
        \hline
        \multicolumn{4}{c}{\textit{Enriched samples}} \\
        \hline
        Green (1973) $^{90}$Zr     & Enriched & 12 & 54.67 \\
        Musgrove (1977) $^{90}$Zr  & Enriched & 14 & 60.39 \\
        Musgrove (1977) $^{91}$Zr  & Enriched & 8 & 23.67 \\
        \hline
        \multicolumn{4}{c}{\textit{Natural samples}} \\
        \hline
        Rapp (2019) Nat-Zr, 6~cm   & Natural  & 13 & 66.93 \\
        Rapp (2019) Nat-Zr, 10~cm  & Natural  & 13 & 33.20 \\
        \hline
    \end{tabular}
\end{table}


The $\chi^2/N$ values in \autoref{tab:chi2_summary} are computed using only the reported experimental uncertainties and are therefore substantially greater than unity. This reflects the significance of the self-shielding model uncertainty (\autoref{ssec:eval_ct_uncertainty}): when the model uncertainty is included in the total uncertainty budget via \autoref{eq:total_uncertainty}, the $\chi^2/N$ values decrease toward unity, confirming that the combined experimental and model uncertainties provide a realistic estimate of the total uncertainty. The $^{91}$Zr Musgrove dataset shows the best agreement ($\chi^2/N = 23.67$), consistent with the conventional (non-doorway) URR behavior of this isotope being more straightforward to model. The enriched $^{90}$Zr datasets show comparable fit quality ($\chi^2/N \approx 55$--$60$), reflecting the challenge of simultaneously fitting the intermediate structure region across all datasets.

The natural zirconium datasets show $\chi^2/N$ values comparable to or better than the enriched samples, which is notable given the additional modeling challenge posed by the un-re-evaluated isotopes ($^{92,94,96}$Zr) that retain their ENDF/B-VIII.1 parameters. The natural data serve an important role in the fit by providing strongly self-shielded constraints that discriminate between the smooth and resonant contributions to the cross section, and by enforcing consistency of the combined $^{90}$Zr and $^{91}$Zr parameters with direct measurement on natural samples. A comprehensive re-evaluation of all stable zirconium isotopes would be required to further improve agreement with the natural zirconium data.



\subsection{Comparison to ENDF/B-VIII.1 via MCNP}
\label{ssec:eval_mcnp_comparison}

To validate the evaluation against the current ENDF/B-VIII.1 library on equal footing, MCNP simulations were performed for each experimental configuration. MCNP uses probability tables sampled from the ENDF/B-VIII.1 URR parameters to compute self-shielded transmission through the exact sample geometry, thickness, and isotopic composition of each experiment. This provides a direct comparison: both the RPI evaluation and the ENDF library are tested against the same experimental data using the same observable (transmission). The pointwise experimental data were rebinned onto the MCNP energy grid ($\sim$7--18~keV bins depending on the dataset), and $\chi^2/N$ was computed at the rebinned points, restricted to the URR energy ranges (0.8--1.78~MeV for $^{90}$Zr, 0.22--1.25~MeV for $^{91}$Zr, and up to 1.78~MeV for the natural samples). The results are summarized in \autoref{tab:mcnp_comparison}.

\begin{table}[h!]
    \centering
    \caption{Comparison of reduced $\chi^2/N$ between the RPI evaluation and ENDF/B-VIII.1 (via MCNP) for all transmission datasets, restricted to the URR energy ranges.}
    \label{tab:mcnp_comparison}
    \begin{tabular}{l c c c c}
        \hline
        \textbf{Dataset} & \textbf{$N$} & \textbf{ENDF/B-VIII.1} & \textbf{RPI Evaluation} & \textbf{Improvement} \\
        \hline
        Green (1973) $^{90}$Zr     & 97  & 2.9   & 3.2   & 0.9$\times$ \\
        Musgrove (1977) $^{90}$Zr  & 55  & 120.6 & 99.8  & 1.2$\times$ \\
        Musgrove (1977) $^{91}$Zr  & 148 & 290.1 & 132.5 & 2.2$\times$ \\
        Rapp (2019) Nat-Zr, 6~cm   & 86  & 910.6 & 298.3 & 3.1$\times$ \\
        Rapp (2019) Nat-Zr, 10~cm  & 86  & 817.7 & 309.1 & 2.6$\times$ \\
        \hline
    \end{tabular}
\end{table}

Figures~\ref{fig:mcnp_green}--\ref{fig:mcnp_rapp10} present the detailed transmission comparisons for each dataset, showing the pointwise experimental data, the rebinned experimental averages, and the calculated transmission from both the RPI evaluation and ENDF/B-VIII.1 (via MCNP). The lower panels display the $(C-E)/E$ residuals for both calculations.

\begin{figure}[h!]
    \centering
    \includegraphics[width=0.95\textwidth]{Evaluation/Figures/fit_detail_green.pdf}
    \caption{Transmission comparison for the Green (1973) enriched $^{90}$Zr dataset (0.0799~at/b, 97.7\% enriched). The RPI evaluation (solid red) captures the intermediate structure dips near 0.55, 0.7, and 0.85~MeV that the ENDF/B-VIII.1 calculation (dashed blue) misses entirely. Residuals for both calculations are comparable in the URR above 0.8~MeV.}
    \label{fig:mcnp_green}
\end{figure}

\begin{figure}[h!]
    \centering
    \includegraphics[width=0.95\textwidth]{Evaluation/Figures/fit_detail_musgrove90.pdf}
    \caption{Transmission comparison for the Musgrove (1977) enriched $^{90}$Zr dataset (0.0827~at/b, 97.7\% enriched). Both evaluations show similar performance above 1~MeV, but ENDF/B-VIII.1 exhibits oscillatory residuals in the 0.5--0.8~MeV range that are substantially reduced in the RPI evaluation.}
    \label{fig:mcnp_musgrove90}
\end{figure}

\begin{figure}[h!]
    \centering
    \includegraphics[width=0.95\textwidth]{Evaluation/Figures/fit_detail_musgrove91.pdf}
    \caption{Transmission comparison for the Musgrove (1977) enriched $^{91}$Zr dataset (0.0642~at/b, 89.2\% enriched). The ENDF/B-VIII.1 calculation shows prominent oscillatory structure near 0.4--0.8~MeV (residuals exceeding $\pm 10\%$) arising from spurious $^{90}$Zr doorway structure propagated into the $^{91}$Zr evaluation. The RPI evaluation eliminates these artifacts and produces a smooth transmission that tracks the experimental data.}
    \label{fig:mcnp_musgrove91}
\end{figure}

\begin{figure}[h!]
    \centering
    \includegraphics[width=0.95\textwidth]{Evaluation/Figures/fit_detail_rapp6.pdf}
    \caption{Transmission comparison for the Rapp (2019) natural zirconium 6~cm sample (0.2581~at/b). The ENDF/B-VIII.1 calculation (dashed blue) systematically overpredicts transmission across most of the energy range, consistent with an underestimation of the total cross section. The RPI evaluation (solid red) substantially reduces this bias.}
    \label{fig:mcnp_rapp6}
\end{figure}

\begin{figure}[h!]
    \centering
    \includegraphics[width=0.95\textwidth]{Evaluation/Figures/fit_detail_rapp10.pdf}
    \caption{Transmission comparison for the Rapp (2019) natural zirconium 10~cm sample (0.4296~at/b). The heavy self-shielding in this thick sample amplifies the systematic bias in ENDF/B-VIII.1, which overpredicts the transmission by 20--40\% in the 0.5--1.0~MeV region. The RPI evaluation provides substantially better agreement, though residual discrepancies remain, likely due to the un-re-evaluated $^{92,94,96}$Zr isotopes.}
    \label{fig:mcnp_rapp10}
\end{figure}

The RPI evaluation achieves improved agreement across four of the five datasets. The most dramatic improvements occur for the natural zirconium samples, where the $\chi^2/N$ is reduced by factors of 2.6--3.1. The enriched $^{91}$Zr dataset shows a factor of 2.2 improvement, reflecting the elimination of spurious oscillatory structure present in the ENDF/B-VIII.1 evaluation near 0.4--0.8~MeV that is not supported by the experimental transmission data.

The only dataset where ENDF/B-VIII.1 achieves comparable or slightly better performance is the Green (1973) enriched $^{90}$Zr measurement ($\chi^2/N = 2.9$ vs.\ 3.2). As visible in \autoref{fig:mcnp_green}, this thin, highly-enriched sample experiences relatively modest self-shielding, making it less sensitive to the improvements in the self-shielding treatment. The RPI evaluation trades marginal agreement with this single dataset for substantially improved performance across the broader suite, particularly the application-relevant thick natural samples.

The Musgrove (1977) $^{91}$Zr comparison (\autoref{fig:mcnp_musgrove91}) illustrates the most striking qualitative improvement. The ENDF/B-VIII.1 calculation produces oscillatory residuals of $\pm 10\%$ in the 0.4--0.8~MeV range, arising from the spurious propagation of $^{90}$Zr doorway structure into the $^{91}$Zr cross section. The RPI evaluation eliminates these artifacts entirely, producing smooth residuals that track the experimental data.

The improvement on the natural zirconium data (\autoref{fig:mcnp_rapp6} and \autoref{fig:mcnp_rapp10}) is particularly significant because the ENDF/B-VIII.1 evaluation was originally derived from natural zirconium measurements, meaning the natural samples should represent its most favorable comparison. However, the ENDF evaluation exhibits a systematic positive bias in the $(C-E)/E$ residuals across most of the energy range, consistently overpredicting transmission (underpredicting the total cross section). This pattern is consistent with the hypothesis that the ENDF cross sections were derived from natural measurements without proper self-shielding corrections applied: infinitely dilute cross sections extracted from self-shielded transmission data without correction will systematically underestimate the true cross section. The approximately 5\% higher cross section produced by the present evaluation across the full energy range (\autoref{fig:eval_zr90_totxs_comparison}) is consistent with the magnitude of self-shielding correction expected for these sample thicknesses. The thick 10~cm natural sample amplifies this effect, explaining why both evaluations show their largest residuals for this dataset, and why the RPI evaluation, which treats self-shielding self-consistently, shows the greatest relative improvement.


\section{Discussion}
\label{sec:eval_discussion}

This evaluation demonstrates the successful application of the self-shielding correction and uncertainty quantification methodologies developed in earlier chapters to a practical nuclear data evaluation problem. The work represents several methodological advances while also revealing important limitations that should be addressed in future work.


\subsection{Methodological Advances}
\label{ssec:eval_discussion_advances}

A central methodological contribution of this work is the direct fitting of URR parameters to transmission measurements rather than to pre-corrected cross-section data. Traditional URR evaluation workflows require experimentalists to apply self-shielding corrections to their transmission data before reporting ``infinitely dilute'' cross sections, which are then used by evaluators as fitting targets. This approach introduces several problems: the correction depends on the very parameters being evaluated (creating a circular dependency), different experimentalists may apply inconsistent correction procedures, and the correction uncertainty is rarely propagated to the reported data. By fitting directly in measurement space, comparing calculated transmission $T^{\mathrm{calc}} = \exp(-n\langle\sigma\rangle/C_T)$ to experimental transmission $T^{\mathrm{exp}}$, these issues are circumvented. The self-shielding correction factor $C_T$ is computed self-consistently at each iteration using the current parameter estimates, eliminating the circular dependency. This approach also enables principled propagation of the self-shielding model uncertainty into the parameter covariances, as demonstrated in Chapter~\ref{chap:uncertainty}.

A practical consequence of fitting in measurement space is improved discrimination between parameters with similar effects on the infinitely dilute cross section. The strength function $S_\ell$ and distant-level parameter $R_\ell^\infty$ both contribute to the average cross section, creating strong correlations when fitting to cross-section data. However, their effects on self-shielded transmission are distinguishable: the strength function governs the resonance fluctuations that drive self-shielding, while $R_\ell^\infty$ contributes a smooth background that experiences minimal self-shielding. Fitting to transmission data from samples of varying thickness exploits this distinction and significantly reduces the $S_\ell$--$R_\ell^\infty$ correlation.

More broadly, each measurement type included in the simultaneous fit provides distinct parameter sensitivities. Total cross-section measurements constrain the sum of all partial cross sections, providing strong sensitivity to $S_\ell$ and $R_\ell^\infty$. Transmission measurements on thick samples exhibit pronounced self-shielding, which is primarily driven by the strength function, helping break the $S_0$--$R_0^\infty$ correlation. Thin-sample transmission, conversely, experiences weaker self-shielding and is more sensitive to the background potential scattering component. Capture cross-section measurements provide direct sensitivity to $\langle\Gamma_\gamma\rangle$, though such data are scarce for these isotopes. Finally, enriched samples isolate individual isotopic contributions, while the natural zirconium samples from Rapp (2019) are strongly self-shielded and simultaneously constrain both $^{90}$Zr and $^{91}$Zr parameters, providing powerful discrimination between smooth and resonant cross-section contributions.

This evaluation also achieves simultaneous fitting of transmission measurements and cross-section constraints within a single optimization framework. Below the URR, the binned resolved resonance cross section provides a constraint that ensures continuity at the RRR/URR interface; within the URR, transmission data from multiple samples and isotopes, including both enriched and natural zirconium samples, provide complementary parameter sensitivities. To the author's knowledge, this represents the first URR evaluation to simultaneously fit both measurement types in a unified framework. The simultaneous approach ensures that the final parameters are globally optimal with respect to all constraints rather than potentially trapped in local minima that satisfy one dataset at the expense of another, and it produces a single, self-consistent parameter covariance matrix that properly accounts for the correlations induced by shared parameters across datasets.

\subsection{Limitations and Future Work}
\label{ssec:eval_discussion_limitations}

Despite the methodological advances, several significant limitations remain that should be addressed in future work.

The reported parameter and cross-section uncertainties are almost certainly underestimated. The current uncertainty quantification includes the statistical uncertainty from finite resonance sampling (the self-shielding model uncertainty) and the experimental statistical uncertainties, but several additional uncertainty sources are not fully propagated. The transmission measurements used in this evaluation have associated systematic uncertainties in sample thickness, isotopic composition, and normalization that are not fully characterized in the original publications. These systematics can produce correlated biases across energy bins that are not captured by the reported statistical uncertainties. The doorway state energies and escape widths were taken from shell-model calculations and adjusted empirically to match observed peak positions; while the spreading widths were fitted, the other doorway parameters were held fixed, and a complete uncertainty quantification would propagate uncertainties in all doorway parameters, substantially increasing the cross-section uncertainty in the energy regions dominated by doorway contributions. The URR formalism itself involves approximations (single-level Breit-Wigner vs.\ Reich-Moore, assumed statistical distributions, energy-independent average parameters within bins) whose uncertainties are difficult to quantify but are certainly non-negligible. Future work should develop a more comprehensive uncertainty budget that includes these additional sources, likely resulting in parameter uncertainties that are factors of two to three larger than currently reported.

A subtle but important issue arises from the treatment of intermediate structure in this evaluation. The doorway state model produces smooth, energy-dependent enhancements to the strength function that significantly elevate the cross section above the compound nucleus baseline, by factors of two to three near the doorway peaks. However, these enhancements are deterministic (zero variance) contributions that do not participate in the resonance fluctuations governing self-shielding. This creates a problem for transport codes that use probability tables to represent URR self-shielding. The standard practice for handling mean discrepancies between the evaluated mean and the sampled mean is multiplicative normalization, but as derived in recent work \cite{Golas2025VariancePreservation}, this multiplicative scaling inadvertently rescales the variance by the square of the same factor, distorting the self-shielding behavior. The variance-preserving transformation proposed in Ref.~\cite{Golas2025VariancePreservation} provides a solution: enforce the evaluated mean via an additive shift rather than multiplicative scaling, thereby preserving the physically sampled variance. However, implementing this correction requires changes to nuclear data processing codes and potentially to the ENDF format itself. Until such changes are adopted, the present evaluation cannot be fully utilized in production transport calculations without accepting some degree of self-shielding bias. This limitation highlights a broader tension in URR evaluation: the physics of intermediate structure produces smooth cross-section contributions that are distinct from compound nucleus fluctuations, yet the ENDF URR format and associated processing tools were designed primarily for compound nucleus statistics.

The intermediate structure model significantly improves agreement with experimental data compared to a purely statistical treatment, but residual discrepancies remain, particularly in the energy region (0.5--0.8~MeV) where the largest doorway contributions occur. Because the doorway parameters (energies and escape widths) were not varied in the final fit, any errors in these fixed parameters must be compensated by adjustments to the fitted compound nucleus parameters. Future work should explore simultaneous fitting of doorway and compound nucleus parameters, which would require a substantially larger parameter space but could reduce these residual discrepancies.

The complete absence of capture cross-section measurements within the $^{90}$Zr URR (0.8--1.78~MeV) represents the most significant experimental gap in this evaluation. The capture cross section is determined entirely by extrapolation of the radiation width from the resolved region and by consistency requirements with the transmission-derived parameters. For $^{91}$Zr, capture data from Ohgama (2005) extend into the lower portion of the URR, providing some constraint, but the upper portion remains unconstrained. New capture cross-section measurements for both isotopes at modern time-of-flight facilities would substantially improve future evaluations.

Finally, a practical outcome of this work is the demonstration that historical transmission measurements, previously considered unusable for URR evaluation due to the self-shielding correction problem, can be directly incorporated into modern evaluations. The Musgrove (1977) and Green (1973) datasets were measured decades ago, but by fitting directly to the uncorrected transmission data, this evaluation extracts information from these measurements without inheriting the biases of historical processing. This approach could be applied to legacy transmission data for other isotopes, potentially enabling re-evaluation of URR parameters for nuclides where modern measurements are unavailable. The key requirement is access to the original transmission data rather than only the corrected cross sections.


\subsection{Comparison to Previous Evaluations}
\label{ssec:eval_discussion_comparison}

The present evaluation differs substantially from ENDF/B-VIII.1 for both isotopes. For $^{90}$Zr, the inclusion of intermediate structure produces cross sections approximately 5\% higher than ENDF/B-VIII.1 across the URR, with the enhancement concentrated near the p-wave doorway peaks. For $^{91}$Zr, the smooth evaluated cross section contrasts with the fluctuating ENDF/B-VIII.1 cross section, which contains artifacts from natural zirconium data processing. The MCNP comparison (\autoref{ssec:eval_mcnp_comparison}) demonstrates that these differences translate into substantial improvements in agreement with experimental transmission data, with $\chi^2/N$ reductions of 2--3$\times$ for the natural zirconium and enriched $^{91}$Zr datasets.

These differences have implications for reactor physics applications. Zirconium is used extensively as fuel cladding in light water reactors, and accurate cross sections in the keV--MeV range affect calculations of neutron economy, spectral indices, and reactivity coefficients. The intermediate structure in $^{90}$Zr, in particular, produces enhanced neutron absorption in an energy range that overlaps with the slowing-down spectrum in thermal reactors.

Integral validation against critical benchmarks and reactor measurements would help assess whether the present evaluation or ENDF/B-VIII.1 provides better agreement with macroscopic observables. Such validation is beyond the scope of this dissertation but represents an important next step before the evaluation could be considered for adoption in future ENDF releases.
