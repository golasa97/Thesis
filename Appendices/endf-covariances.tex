\chapter{Monte Carlo Propagation of Theoretical Covariances}
\label{app:theory-cov-mc}

The gradient descent procedure in \autoref{app:gradient-descent} produces a best-fit parameter vector $\mathbf{p}^\star$ and an associated parameter covariance matrix $C_{\mathbf{p}}$. However, the quantities of interest for an evaluation are not the parameters themselves but the cross sections $\boldsymbol{\sigma}(\mathbf{p})$ evaluated on a reporting energy grid. Because the relationship between URR parameters and cross sections is nonlinear, a linear error propagation (i.e., $C_{\boldsymbol{\sigma}} = J\, C_{\mathbf{p}}\, J^T$) can be a poor approximation. This appendix describes the Monte Carlo sampling procedure used to propagate parameter covariances to cross-section covariances without linearization.

This same machinery is general-purpose: it is implemented as a module within the fitAPI framework and can be applied to any model and dataset type. In principle, one could instead \emph{include} experimental parameters such as sample temperature, thickness, and isotopic abundance in the covariance and propagate their uncertainties alongside the resonance parameters. This would produce a more complete uncertainty estimate on the experimental observable itself, accounting for systematic effects beyond resonance parameter uncertainty. That application is not pursued in this work, but the infrastructure supports it.


The goal is to generate a large number of parameter vectors that are statistically consistent with the fitted mean $\mathbf{p}^\star$ and covariance $C_\mathbf{p}$. This is done by drawing random samples from a multivariate normal distribution with mean $\mathbf{p}^\star$ and covariance $C_\mathbf{p}$.

To draw from this distribution, the covariance matrix is first decomposed into its eigenvalues $\lambda_i$ and eigenvectors. Any eigenvalues that are negative (which can occur due to numerical noise in the inverse-Hessian estimate from the fit) are set to zero. A square root matrix $L$ is then constructed from the eigenvalues and eigenvectors such that $L L^T = C_\mathbf{p}$.

A single parameter sample is generated by drawing a vector $\mathbf{z}$ of independent random numbers from a standard normal distribution (mean zero, variance one), and computing
\begin{equation}
    \mathbf{p}^{(k)} = \mathbf{p}^\star + L\, \mathbf{z}^{(k)}.
    \label{eq:mc_sample}
\end{equation}
This produces a parameter vector $\mathbf{p}^{(k)}$ that is drawn from the desired multivariate normal distribution. The procedure is analogous to the Monte Carlo sampling used throughout this work for generating resonance ladders from average parameters: a random number is drawn from a known distribution and used to produce a single realization, and the process is repeated many times to build up a statistical population.

After each draw, any parameters that fall outside their physical bounds (e.g., a strength function that has been sampled as negative) are set to the nearest allowed value.

For each sampled parameter vector $\mathbf{p}^{(k)}$, the URR model computes the corresponding cross sections on the evaluation energy grid:
\begin{equation}
    \sigma^{(k)} = \sigma(\mathbf{p}^{(k)}).
    \label{eq:mc_forward}
\end{equation}
Any samples that produce non-physical values (e.g., NaN results from parameter combinations that cause the cross-section calculation to fail) are discarded. In practice, only a small number of samples are affected.

After generating $N$ valid samples, the mean cross section at each energy point $i$ is computed as
\begin{equation}
    \bar{\sigma}_i = \frac{1}{N} \sum_{k=1}^{N} \sigma_i^{(k)},
    \label{eq:mc_mean}
\end{equation}
and the covariance between energy points $i$ and $j$ is
\begin{equation}
    (C_\sigma)_{ij} = \frac{1}{N-1} \sum_{k=1}^{N} \left(\sigma_i^{(k)} - \bar{\sigma}_i\right)\left(\sigma_j^{(k)} - \bar{\sigma}_j\right).
    \label{eq:mc_cov}
\end{equation}
The $1\sigma$ uncertainty at each energy point is then the square root of the diagonal,
\begin{equation}
    \Delta\sigma_i = \sqrt{(C_\sigma)_{ii}},
    \label{eq:mc_unc}
\end{equation}
and the energy-energy correlation coefficient between points $i$ and $j$ is
\begin{equation}
    \rho_{ij} = \frac{(C_\sigma)_{ij}}{\Delta\sigma_i\, \Delta\sigma_j}.
    \label{eq:mc_corr}
\end{equation}
